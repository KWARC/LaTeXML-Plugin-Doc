\Para{Der Euler-Operator}{}
\Def{}{}
F�r einen Punkt $w\in \Theta$ mit $\pi(w)=(p,q)$
definieren wir ein Unterb�ndel $V$ des Tangentialb�ndels $\T\Theta$ durch
$$V_w\colon=\T_w\Hom(T_pM,\T_q\widetilde M)=\pi^{-1}(\pi(w))\;.$$
$\T\Theta$ l��t sich aufspalten in die direkte B�ndelsumme
$$\T\Theta=\H\oplus \V=\T M\oplus\T\widetilde M\oplus V$$
mit $H\colon=\T M\times\widetilde M=\T M\oplus\T\widetilde M$.
Einen Vektor $W\in \T_w\Theta$ mit $\pi(w)=(p,q)$ 
kann man also eindeutig zerlegen in
$W=X+Y+Z$ mit $X\in\T M$, $Y\in \T\widetilde M$ und $Z\in \T\V$.
\Def{}{}
Wir f�hren auf $\T\Theta$ und den oben definierten Unterb�ndeln 
lineare Zusammenh�nge ein.
$$\nabla^\Theta=\nabla^\H+\nabla^\V$$
dabei ist 
$$\nabla^\H=\nabla+\widetilde\nabla$$
und $\nabla^\V$ der euklidische Zusammenhang auf 
$\T\Hom(\T_pM,\T_q\widetilde M)$.\par
Ist $\Phi\colon\Theta\to\RR$ differenzierbar so definieren wir
$$\eqalign{\Phi_\H(w)\colon=\nabla^\H\Phi(w)
                &\colon\T_{(p,q)}M\times\widetilde M\to\RR      \cr
           \Phi_Y(w)\colon=\widetilde\nabla\Phi(w)
                &\colon\T_q\widetilde M\to\RR                   \cr
           \Phi_\V(w)\colon=\nabla^\V\Phi(w)
                &\colon\T\Hom(\T_pM,\T_q\widetilde M)\to\RR\;.  \cr
      }$$
\Satz{Euler-Gleichungen}{}
{\it Ein kritischer Punkt $f$ von $\F$ erf�llt die Gleichung 
$$\Phi_Y(\theta)-C_{1,2}\nabla^M\Phi_\V(\theta_f))=0$$
auf $\T\widetilde M$.}
\Beweis
F�r ein Vektorfeld $\phi$ l�ngs $f$ ist 
$$d_p\phi\colon\T_pM\to\T_{\phi(p)}\T_{f(p)}\widetilde M
                       \cong\T_{f(p)}\widetilde M\;.$$
Damit ist dann
$$\left.{d\over dt}\phi\circ\theta_{f_t}\right|_{t=0}
     =\Phi_Y(\theta_f)(\phi)+\Phi_\V(\theta_f)(\phi)$$
denn $$\left.{d\over dt}(\Id_M,\exp_ft\phi)\right|_{t=0}=(0,\phi)\;,$$ 
$$\left.{d\over dt}(d_p\exp_ft\phi)\right|_{t=0}=d_p\phi$$ und
$$\Phi_H(theta_df)(0,\phi)=\Phi_Y(\theta_f)(\phi)\;.$$


\Def{}{\EULEROP}
Wir definieren den {\bf Euler-Operator zu $\F$} 
$$L_f\colon \C^2(M,NM)\longrightarrow \C^0(M,TM)$$ durch
$$L_f\phi\colon =\bigl[F_y\klamf-C_{X,V^*}
       (\nabla_xF_{V^*\otimes W}\klamf)\bigr]^\sharp\;.$$
\Bem{}{}
Es gilt mit diesen Bezeichnungen
$$\partial\F(f,\phi)=\int_M{g(L(f),\phi)\dvol}$$
\Bem{}{}
$L_f$ ist ein quasilinearer Differentialoperator der Ordnung 2.
\Beweis

