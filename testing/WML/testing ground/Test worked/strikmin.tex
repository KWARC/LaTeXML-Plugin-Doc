\Para{Striktes Minimum}{}
\Def{}{}
Eine {\bf ($m$-dimensionale Distribution $\theta$ auf $M$} ist ein 
Unterb�ndel $\theta$ von Rang $m$ des Tangentialraumes $\T M$. 
F�r jedes $p\in M$ ist also $\theta_p\subset \T_pM$ ein $p$-dimensionaler 
Teilraum.\par
Ist das Unterb�ndel $\theta\subset \T M$ von der Klasse $\C^k$, so hei�t
$\theta$ eine $\C^k$-Distribution.
\Def{}{}
Eine $p$-dimensionale immersierte Untermannigfaltigkeit 
$f\colon N\longrightarrow M$ hei�t {\bf Integral-Untermannigfaltigkeit zu 
$\theta$},
falls f�r alle $p\in N$ gilt $\theta(f(p))=df(\T _pN)$.\par
$\theta$ hei�t {\bf integrabel in $p\in M$}, falls es eine 
Integraluntermannigfaltigkeit $N\subset M$ zu $\theta$ mit $p\in N$.
\Lemma{}{}
{\it Ist $\theta$ eine $p$-dimensionale $\C^1$-Distribution und 
$f\colon N\to M,\widetilde f\colon\widetilde N\to M$ 
Integral-Untermannigfaltigkeiten von $\theta$ mit 
$f(p)\in f(N)\cap\widetilde f(\widetilde N)$, so gibt es offene 
Untermannigfaltigkeiten
$U$ von $N$ respektive $\widetilde U$ von $\widetilde N$, 
so da� $f(U)=\widetilde f(\widetilde U)$}
\Beweis
Siehe [CH] p. 91.
\Bem{}{}
Sind $f\colon N\to M$ und $\widetilde f\colon\widetilde\rightarrow M$ 
Integral-Untermannigfaltigkeiten von $\theta$
mit $\partial f(N)\cap\partial \widetilde f(\widetilde N)\ne\emptyset$, so ist 
$f(N)=\widetilde f(\widetilde N)$.
\Beweis
$f(N)\cap\widetilde f(\widetilde N)$ ist offensichtlich abgeschlossen  
und nach Voraussetzung nichtleer.
Aufgrund des obigen Lemmas ist $f(N)\cap\widetilde f(\widetilde N)$ auch 
offen in $f(N)$. Es ist also 
$f(N)\cap\widetilde f(\widetilde N)=f(N)$, das hei�t 
$\widetilde f(\widetilde N)\subset f(N)$.\par
Die Inklusion in der anderen Richtung zeigt man analog.\kasten

\Kor{}{\MINISTRIKT} 
{\it Mit den Voraussetzungen von {\MINIFELD} ist $M$ ein striktes starkes 
Minimum.}
\Beweis
Wir betrachten die zu der Normalen $\nu$ an $F$ duale Distribution 
$\theta$ aller Tangentialr�ume an $\Phi$. Offensichtlich ist $M$ eine 
Integraluntermannigfaltigkeit zu $\theta$.
F�r $N\ne M$ mit $\partial N=\partial M\ne\emptyset$ mu� sich das 
Normalenfeld 
$\nu_N$ an $N$ auf einer Teilmenge $\widehat N\subset N$ von $\nu\bigr |_N$
mit $\H^n(\widehat N)>0$ unterscheiden, weil $\nu_N$ stetig ist und sonst 
$N$ mit $M$ identisch w�ren.
Dort hat die Projektion von $\nu_N$ auf $\nu\bigr|_N$ eine 
L�nge <1, und deswegen gilt $\omega_\nu\bigr|_{\widehat N}<1$. In 
{\KALIFUN} gilt also die strikte Ungleichung.\kasten
\Kor{}{\EXTRESTRIKT} 
{\it Mit den Voraussetzungen von {\EXTREFELD} ist $M$ ein striktes starkes 
Minimum.}
\Beweis
Wir betrachten hier die Distribution $\theta$ der Tangentialr�ume an
$\Phi$, sie ist gegeben durch 
$$\theta(q)\colon =\Im(\partial f\bigl|_{f^{-1}(q)})\subset \T \widetilde M.$$
Nun ist $\partial f(M)=\partial g(M)$ aber $g\ne f$, deswegen kann 
$g(M)$ keine Integraluntermannigfaltigkeit von $\theta$ sein. 
Es gibt also einen Punkt $p_0\in M$ so, da� 
$$\partial f\bigl|_{f^{-1}(g(p))}\ne\partial g\bigl|_p.$$
$\E(p,g(p),\partial f\bigl|_{f^{-1}(g(p))},\partial g\bigl|_p)$
ist also auf einer ganzen Umgebung von $p_0$ gr��er als $0$.\kasten 

%%% Local Variables: 
%%% mode: plain-tex
%%% TeX-master: "arbeit"
%%% End: 
