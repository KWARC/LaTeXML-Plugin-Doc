Wir konstruieren in diesem Kapitel ein $\C^{2,\alpha}$-Immersion
$$g:M\times [-\rho,\rho]\longrightarrow \widetilde M,$$
so da� die Niveaufl�chen $M_s:=g(M\times\{s\})$ in $\widetilde M$ 
Minimalfl�chen sind, und da� gilt
$g(x,s)=\tau(s\nu(x))$ auf $\partial M$ und $g(x,0)=f(x)$ auf $M$.\par
Dazu konstuieren wir eine $\C^{2,\alpha}$-Injektion
$$u:M\times[-\rho,\rho]\longrightarrow\widetilde M,$$
so da� 
$$g(x,s):=\tau(su(x,s))=\exp_{f(x)}(su(s,x))\;.$$
Offensichtlich ist dann $g(0,x)=f(x)$, und wir m�ssen f�r alle
$s\in [-\rho,\rho]$ erreichen, da�
$$u(x,s)=\nu \hbox{ auf }\partial M.$$
Sei $s\in [-\rho,\rho]$ fest. Dann ist 
$$h(x,t):=\tau(tsu(x,s))$$
eine glatte Variation von $f$, und es gilt
$$\eqalign{h_0(x)&:=h(x,0)=\tau(0)=f(x)                   \cr
           h_1(x)&:=h(x,1)=\tau(su(x,s))=g(x,s)   \;.     \cr
        }$$
$f$ ist nach Voraussetzung minimal, deswegen mu� man das Problem
$$H(h_1)-H(h_0)=\int_0^1{{{d\over dt}H(h_t)\bigr|}_{t=\tau}d\tau}=0
\leqno(\Num{\HMINH})$$
l�sen. Nun ist aber 
$${d\over dt}H(h_t)\bigr|_{t=\tau}=L_{h_t}(su_s),$$
dabei ist $L_{h_t}$ der Jacobioperator von $h_t$.
Mit den Bezeichnungen $\L_t:=L_{h_t}$ und $\L:=\L_0$ schreibt sich
{\HMINH} als
$$s\int_0^1{\L_tu_s dt}=0\;.$$
Damit erhalten wir
$$ \eqalign{ \L u_s
      &=\int^1_0{(\L_0-\L_t)u_s dt}                                      \cr
      &=-\int^1_0{\int_0^t{{d\over d\sigma}\L_\sigma u_sd\sigma dt}} \;. \cr
         }$$
\Lemma{}{}
{\it Es gibt eine glatte Funktion $G$, so da� 
$$     -\int^1_0{\int_0^t{{d\over d\sigma}\L_\sigma u_sd\sigma dt}} 
            =sG(x,s,u,\nabla u,\nabla^2u).$$}
\Beweis
kommt noch\par
Nun ist aber der kleinste Eigenwert $\lambda$ des Jacobi-Operators $\L=L_f$
positiv, deswegen ist $\L$ nach {\LUMKEHR} invertierbar. Es gibt also genau
eine L�sung $v\in \C^{2,\alpha}(M,NM)$ der Randwertaufgabe
$$\left.\eqalign{\L(v)&=0 \hbox{ auf }M             \cr 
                     v&=\nu  \hbox{ auf }\partial M \cr}\right\}\;.$$
Setzen wir $$\tilde u:=u-v$$ und 
$$H(x,s,\tilde u,\nabla \tilde u,\nabla^2\tilde u)
        :=g(x,s,u+v,\nabla(u+v),\nabla^2(u+v)),$$
so k�nnen wir {\HMINH} in der folgenden Form schreiben
$$\left.
\eqalign{\L \tilde u&=\sigma\cdot 
          H(x,s,\tilde u,\nabla \tilde u,\nabla^2\tilde u)\hbox{ auf }M \cr
         \tilde u   &=0 \hbox{ auf } M                                  \cr}
  \right \}\Num{\HMINHH}$$
oder wegen der Invertierbarkeit von $\L$ als
$$\tilde u(x,s)=s\cdot \L^{-1}H(x,s,\tilde u(x),
       \nabla \tilde u(x),\nabla^2\tilde u(x))\;.$$                                 
Durch 
$$\T U(x,s):=s\cdot \L^{-1}H(x,s, U(x),\nabla U(x),\nabla^2U(x))$$  
definieren wir einen stetigen Operator                              
$$\T:\C^{2,\alpha}_0(M\times[-\rho,\rho],NM)\longrightarrow
\C^{2,\alpha}_0(M\times[-\rho,\rho],NM).$$
Es gilt dann
$$\eqalign{&|TU-TV|_{\C^{2,\alpha}(M\times[-\rho,\rho],NM)}           \cr
     &\leq s|\L^{-1}\bigl(H(\cdot,\cdot,U,\nabla U,\nabla^2U)
                        - H(\cdot,\cdot,V,\nabla V,\nabla^2V)\bigl)
                            |_{\C^{2,\alpha}(M\times[-\rho,\rho],NM)} \cr
     &\leq \rho \|\L^{-1}\|  
                        |H(\cdot,\cdot,U,\nabla U,\nabla^2U)
                         - H(\cdot,\cdot,V,\nabla V,\nabla^2V)
                 |_{\C^{0,\alpha}(M\times[-\rho,\rho],NM)}            \cr
     &\leq \rho \|\L^{-1}\|\omega
                 |U-V|_{\C^{2,\alpha}(M\times[-\rho,\rho],NM)}        \cr
              }$$
Dabei ist $\omega$ der Stetigkeitsmodul von $H$.
F�r hinreichend kleine $\rho>0$ ist 
der Operator $\T$ in der $\C^{2,\alpha}$-Norm eine Kontraktion.
Nach dem Banachschen Fixpunktsatz gibt es einen eindeutigen Fixpunkt
$\tilde u\in \C^{2,\alpha}(M\times[-\rho,\rho],NM)$.
$u(x,s):=\tilde u(x,s)+v(x)$ l�st dann das Problem und ist von der Klasse 
$\C^{2,\alpha}(M\times[-\rho,\rho],NM)$.
Wir haben also mit $g(x,s):=\tau(su(x,s))$ eine Funktion 
$g\in\C^{2,\alpha}(M\times[-\rho,\rho],NM)$ konstruiert, deren Niveaufl�chen
$g(\{s=\const\})$ Minimalfl�chen sind. Analog zu der Argumentation in
{\BLAETTERUNG} erhalten wir, da� die Niveaufl�chen von $g$ eine 
$\C^{2,\alpha}$-Bl�tterung ergeben.


