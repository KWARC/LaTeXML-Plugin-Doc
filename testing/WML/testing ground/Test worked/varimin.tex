\Para{Variation des Funktionals}{}
\Def{}{}
Wir betrachten die Vektorb�ndel
$$\pi\colon\bi{E}\longrightarrow M\times\widetilde M$$
$$\widetilde\pi\colon\bi{F}\longrightarrow M\times\widetilde M$$
mit den Fasern $\pi^{-1}(p,q)=\T_pM$ und 
$\widetilde\pi(p,q)=\T_q\widetilde M$. 
$$\pi\otimes\widetilde\pi\colon\Hom(\bi {E},
 \bi{F})\longrightarrow M\times\widetilde M$$
ist dann ein Vektorb�ndel vom Rang $n^2+n$ �ber $M\times\widetilde M$.
Im folgenden bezeichnen wir das B�ndel 
$\Hom(\bi{E},\bi{F})\longrightarrow M\times\widetilde M$ mit 
$\pi\colon\Theta\longrightarrow M\times\widetilde M$.\par
Ist $f\colon M\longrightarrow\widetilde M$ differenzierbar, so ist f�r jedes
$p\in M$ und $q\colon=f(p)\in \widetilde M$ 
$$d_pf\in \colon\T_pM\longrightarrow\T_q\widetilde M$$
das zugeh�rige Differential. Durch $\theta_f(p)\colon=d_pf)$ wird
dann eine Abbildung 
$$\theta\colon M\longrightarrow \Theta$$
definiert.
Wir betrachten im folgenden die Lagrangesche 
$$\Phi\colon\Theta\longrightarrow \RR$$
und das zugeh�rige Variationsintegral 
$$\F(f)\colon =\int_M{\Phi\circ\theta \dvol}\;.$$
Wir w�hlen einen endlichen Atlas 
${\cal A}=\{(x_i, U_i)\colon i\in I\}$ f�r $M$, einen
endlichen Atlas
${\cal \widetilde A}=\{(y_j \widetilde U_j)\colon j\in I\}$ f�r 
$\widetilde M$ mit $x_i\colon U_i\to V_i$, 
$y_j\colon\widetilde U_j\to\widetilde V_j$ und eine
Teilung der Eins $\{\psi_i\colon i\in I\}$ mit $\supp\psi_i\subset U_i$.
Verm�ge $dx_i$ und $dy_j$ sind dann $\T U_i\cong V_i\times\RR^n$ und
$\T \widetilde U\cong\widetilde V\times\RR^{n+1}$.
Die Zuordnung
$$d_pf\longmapsto\klamf$$
gibt einen Diffeomorphismus
$$\Theta\bigl|_{U\times\widetilde U}\cong V\times\widetilde V\times\RR^{n^2+n}\;.$$
In lokalen Koordinaten ist 
$$\F(f)=\sum_{i\in I}{\int_M{\psi_i\Phi(d_pf)\dvol}}
       =\sum_{i\in I}{\int_{V_i}{F_i(x,u_i(x),Du_i(x))dx}}\;,$$
dabei ist $u_i\colon =y_j\circ f\circ x_i^{-1}$.
\Def{}{}
Sei $f\colon M\to \widetilde M$ eine Immersion.
Eine $\C^\infty$-Abbildung 
$$F\colon (-\epsilon,\epsilon)\times M\longrightarrow \widetilde M$$
hei�t {\bf glatte Variation von f}, falls
{\parindent=1 truecm
\item{a)} F�r alle $t\in(-\epsilon,\epsilon)$ ist 
$f_t\colon =F(t,\cdot)\colon M\to \widetilde M$ eine Immersion,
\item {b)} $f_0=f$
\item {c)} F�r alle $t\in (-\epsilon,\epsilon)$ ist 
${f_t\bigl |}_{\partial M} = {f\bigl |}_{\partial M}$
}
\Def{}{}
Eine Abbildung  $\phi\colon M\to\T\widetilde M$ hei�t 
{\bf Vektorfeld l�ngs $f$}, wenn $\pi_{\widetilde M}\circ\phi=f$ gilt.
\Def{}{}
$F$ hei�t {\bf glatte Variation von f in Richtung $\phi$}, falls
$$\phi=d_pF\left({\partial\over\partial t}\right)\;.$$
\Bem{}{}
{\it Ist $\phi$ ein $\C^\infty_0$ Vektorfeld l�ngs $f$, so ist
$F(t,p)\colon =\exp_{f(p)}(t\phi)$ eine glatte Variation von $f$.}

\Def{}{}
Die {\bf erste Variation von $\F$ in Richtung $\phi$} ist definiert als
$$\partial \F(f,\phi)\colon ={d\over dt}{I(\exp_ft\phi)\bigl |}_{t=0}$$
Die {\bf zweite Variation von $\F$ in Richtung $\phi$} ist definiert als
$$\partial^2 \F(f,\phi)\colon ={d\over dt}{\partial I(\exp_ft\phi,\phi)\bigl |}_{t=0}$$
$f$ hei�t {\bf kritischer Punkt von $I$}, falls f�r alle $\C^\infty_0$ 
Vektorfelder $\phi$
$$\partial \F(f,\phi)=0.$$
\par $f$ hei�t {\bf relatives Minimum}, falls es f�r alle 
$\phi\in\C^\infty(M,T\widetilde M)$ ein $t_0(\phi)>0$ gibt, so da�
$I(f)\leq I(\exp_ft\phi)$ f�r alle $t\in(-t_0(\phi),t_0(\phi))$.
\par $f$ hei�t {\bf starkes Minimum zu festen Randwerten von $\F$}, 
falls $\F(f)<\F(g)$ f�r alle Immersionen $g$, so da� $g(M)$ in einer
Tubenumgebung von $f(M)$ liegt und $\partial f(M)=\partial g(M)$.\par
$f$ hei�t {\bf homologisches Minimum von $\F$}, falls nur in der 
Homologieklasse von $f$ minimiert wird.

