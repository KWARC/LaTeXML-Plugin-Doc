\Para{Beweis von \EABSCH}{}
Sei $G\colon =G(s,t,x)\colon =\bigl (g_{ij}(s\nu + tu)(x)\bigr )
= (\psi + s\nu + tu)_{x_i}(\psi + s\nu + tu)_{x_j}\bigl |_x$, 
$G_t\colon =\bigl ( {d\over dt} g_{ij}(s\nu + tu)\bigr )$ 
und $\|A\|\colon = \sum _{i,j=1}^n {|a_{ij}|}$ f�r $ A\in GL(n,\RR)$.

\Lemma {} {\GABSCH}
{\it Sei $|\nabla u|\leq 1$ und $|\nabla^2 u|\leq1$.
Es gibt eine Konstante $K>0$ so, da� f�r $0\leq t\leq 1$ gilt:
$$\leqalignno{
  \|G\|,        \|G^{-1}\|         & \leq K                           &a) \cr
  \|G_t\|,      \|G^{-1}_t\|       & \leq K |\nabla u|                &b) \cr
  \|G_{tt}\| ,  \|G^{-1}_{tt}\|    & \leq K |\nabla u|^2              &c) \cr
  \|G_{x_i}\|,  \|G^{-1}_{x_i}\|   
        & \leq K (1+|\nabla u||\nabla ^2u|)                           &d) \cr
  \|G_{tx_i}\|, \|G^{-1}_{tx_i}\|  
        & \leq K (|\nabla u| + |\nabla ^2u| + |\nabla u||\nabla ^2u|) &e) \cr
  \|G_{ttx_i}\|,\|G^{-1}_{ttx_i}\| 
        & \leq K (|\nabla u|^2+ |\nabla^2u|+|\nabla u||\nabla ^2u|    &f) \cr
  |(\det G)_t|                     & \leq K |\nabla u|                &g) \cr
  |(\det G)_{tt}|                  & \leq K |\nabla u|^2              &h) \cr
  |(\det G)_{x_i}|, |(\det G)_{tx_i}|,|(\det G)_{ttx_i}| &\leq K\;.   &i) \cr
   }$$}
\Beweis
Sei $\psi $ eine Karte von $M_s$, dann gilt f�r die Ableitungen der Metrik:
$$\eqalign{ g_{ij}(v\nu + tu)
      =\colon  & g(s,t) = (\psi + s\nu + tu)_{x_i}(\psi +s\nu + tu)_{x_j} \cr
      = & \psi _{x_i} \psi _{x_j} 
          + s(\psi_{x_i}\nu_{x_j} + \nu_{x_i}\psi_{x_j}) 
          + t(u_{x_i} \psi _{x_j} + \psi _{x_i} u_{x_j})    \cr
        & + s^2 \nu_{x_i}\nu_{x_j}
          + st (\nu_{x_i} u_{x_j} + u_{x_i}\nu_{x_j})
          + t^2u_{x_i}u_{x_j}                               \cr 
   {d\over dt}g_{ij}(s,t)
      = & u_{x_i} \psi_{x_j} + u_{x_j}\psi_{x_i}
          + s(\nu_{x_i}u_{x_j} + u_{x_i}\nu_{x_j})
          + 2t u_{x_i}u_{x_j}                               \cr
   {d^2\over dt^2}g_{ij}(s,t)
      = & 2u_{x_i}u_{x_j}                                   \cr 
   {d\over ds}g_{ij}(s,t)
      = & \nu_{x_i} \psi_{x_j} + \psi_{x_i}\nu_{x_j}
          + t(\nu_{x_i}u_{x_j} + u_{x_i}\nu_{x_j})
          + 2s \nu_{x_i}\nu_{x_j}                           \cr
   {d^2\over dtds}g_{ij}(s,t)
      = & \nu_{x_i}u_{x_j} + u_{x_i}\nu_{x_j}               \cr 
   {d^3\over dt^2ds}g_{ij}(s,t)
      = & 0 \;.                                             \cr 
         }$$
$M$ und damit $M_s$ sind kompakt. Deswegen sind $u$, $\psi$, $\psi_{x_i}$ 
und $\psi_{x_ix_j}$ f�r jede Kartenwahl beschr�nkt. 
Daraus kann man  die linken Teile der Behauptungen $a)$ bis $f)$ und die Punkte 
$g)$ und  $h)$ direkt ablesen.
\par \noindent Die Inversionsabbildung $F\colon A\mapsto A^{-1}$ ist eine stetige
Abbildung von Gl$(n,\RR)$ auf sich. $G([0,1]^2\times M)$ ist 
kompakt, und deshalb ist $F$ auf $G([0,1]^2\times M)$ beschr�nkt. 
Dies ist die Aussage der Behauptung $a)$.
\def\Gm{G^{-1}}
\par \noindent F�r $r\in \{t,x_1,\ldots,x_n\}$ gilt 
$0=(G\Gm)_r=G_r\Gm+G\Gm_r$ und damit $ \Gm_r=\Gm G_r\Gm$.
Also:
$$\leqalignno{ 
    \Gm_t       & =\Gm G_t \Gm                                  &b)\cr
    \Gm_{tt}    & = \bigl ( \Gm_t\bigr )_t 
                    =\bigl ( \Gm G_t\Gm\bigr )_t                &c)\cr
                & = \Gm_t G_t \Gm + \Gm G_{tt} \Gm 
                    + \Gm G_t \Gm_t                             &  \cr
    \Gm_{x_i}   & =\Gm G_{x_i} \Gm                              &d)\cr
    \Gm_{tx_i}  & = \bigl ( \Gm_t\bigr )_{x_i} 
                    =\bigl ( \Gm G_t\Gm\Bigr )_{x_i}            &e)\cr
                &= \Gm_{x_i} G_t \Gm + \Gm G_{tx_i} \Gm + 
                    \Gm G_t \Gm_{x_i}                           &  \cr
    \Gm_{ttx_i} &= \bigl ( \Gm_{tx_i}\bigr )_t                  &f)\cr
                &= \Gm_{x_it} G_t \Gm + \Gm_{x_i} G_{tt} \Gm 
                    + \Gm_{x_i} G_t \Gm_t                       &  \cr
                &+ \Gm_t G_{x_it} \Gm + \Gm G_{ttx_i}\Gm + 
                    + \Gm G_{tx_i} \Gm_t                        &  \cr
                &+ \Gm_t G_t \Gm_{x_i} + \Gm G_{tt} \Gm_{x_i}
                    + \Gm G_t \Gm_{x_it}                        &  \cr
              }$$
Durch sukzessives Einsetzen der schon gewonnenen Absch�tzungen gewinnt man
die Behauptungen $b)$ bis $f)$ wenn man ausn�tzt, da� die jeweils kleineren
Potenzen von $|\nabla u|$ und $|\nabla^2 u|$ die gr��eren dominieren.
\par \noindent Die Determinantenfunktion $\det G (s,t,x)$ ist ein Polynom in 
$t$ und $x_i$. Deswegen ist sie und ihre Ableitungen beschr�nkt auf 
$[0,1]^2\times M$, dies impliziert $i)$.\kasten
\Para{Beweis der ersten Ungleichung:}{}
$$E(u)=\int ^1_0 {(1-t){d^2 H^\bot \over dt^2} (s,t)dt}$$
Es reicht zu zeigen, da� f�r $0\leq t\leq 1$ gilt 
$$\Bigl |{d^2 H^\bot \over dt^2}(s,t)\Bigl | 
          \leq K \bigl ( |\nabla u|^2 + |\nabla u| |\nabla ^2 u| \bigr )$$
Dazu reicht es, ${d^2 \over dt^2} H_i$ abzusch�tzen. Ausdifferenzieren
ergibt:
$$\eqalign{
    {d^2 \over dt^2}H_1 (s,t)
         = &{2tg^2_t(s,t)-tg(s,t)g_{tt}(s,t)-3g(s,t)g_t(s,t) 
               \over 4g^3(s,t)}
             g_{x_i}(s,t)g^{ij}(s,t)u^\bot_{x_j}                           \cr
           & +{2g(s,t)-tg_t(s,t) \over 2g^2(s,t)} 
             \Bigl (
             g_{tx_i}(s,t)g^{ij}(s,t) + g_{x_i}(s,t)g^{ij}_t(s,t)
             \Bigr )
             u^\bot_{x_j}                                                \cr
           & +{t\over 2g(s,t)}
             \Bigl (
             g_{ttx_i}(s,t)g^{ij}(s,t) + g_{x_i}(s,t)g^{ij}_{tt}(s,t)
             \Bigr ) 
             u^\bot_{x_j}                                                \cr
           & +{t\over g(s,t)}g_{tx_i}(s,t)g^{ij}_t(s,t)u^\bot_{x_j}         \cr
    {d^2 \over dt^2}H_2 (s,t)
         = & 2g^{ij}_{tx_j}(s,t)u^\bot_{x_j} 
             + tg^{ij}_{ttx_j}(s,t) u^\bot_{x_j}                          \cr
    {d^2 \over dt^2}H_3 (s,t)
         = & g^{ij}_{tt}(s,t)(\psi^\bot + s\nu)_{x_ix_j}                  \cr
    {d^2 \over dt^2}H_4 (s,t)
         = & 2g^{ij}_t(s,t)u^\bot_{x_ix_j}
             + tg^{ij}_{tt}(s,t)u^\bot_{x_ix_j} \;.                       \cr
          }$$
Wenn man die Absch�tzungen aus Lemma {\GABSCH} einsetzt, ergibt sich:
$$\eqalign{
    \Bigl | {d^2 \over dt^2}H_1 (s,t)\Bigr |
           \leq & K\bigl ( |\det G_t(s,t)||\nabla u| 
                  +|\det G_{tt}(s,t)||\nabla u|                         \cr
                & + \|G_t^{-1}(s,t)\| |\nabla u|
                  + \|G^{-1}_{tt}(s,t)\| |\nabla u| \bigr )             \cr
           \leq & K \bigl ( |\nabla u|^2 |\nabla u|^3 
                  +|\nabla u|^2+|\nabla u|^3 \bigr )                   \cr
           \leq & K|\nabla u|^2                                        \cr
    \Bigl | {d^2 \over dt^2}H_2(s,t)\Bigr |
           \leq & K\bigl ( \|G_{tx_i}^{-1}(s,t)\| |\nabla u|
                  + \|G^{-1}_{ttx_i}(s,t)\| |\nabla u| \bigr )          \cr
           \leq & K \bigl ( |\nabla u|^2 + |\nabla u||\nabla^2u|
                  + |\nabla u|^2|\nabla^2 u|                           \cr
                & + |\nabla u|^3 + |\nabla u||\nabla^2 u| 
                  + |\nabla u|^2 |\nabla ^2u| \bigr )                  \cr
           \leq & K\bigl ( |\nabla u|^2 + |\nabla u||\nabla^2u| \bigr )\cr
    \Bigl | {d^2 \over dt^2}H_3 (s,t)\Bigr |
           \leq & K\|G_{tt}^{-1}(s,t)\| \leq K |\nabla u|^2             \cr
    \Bigl | {d^2 \over dt^2}H_4 (s,t)\Bigr |
           \leq & K\bigl ( \|G_t^{-1}(s,t)\| |\nabla^2 u|
                  + \|G^{-1}_{tt}(s,t)\| |\nabla^2 u| \bigr )           \cr
           \leq & K \bigl ( |\nabla u||\nabla^2 u| 
                  + |\nabla u|^2 |\nabla ^2u| \bigr )                  \cr
           \leq & K|\nabla u||\nabla ^2u|  \;.                         \cr
     }$$
Daraus folgt die Behauptung.\kasten
\Para{Beweis der zweiten Ungleichung}{}
$$\eqalign{ \lpnorm{E(u)-E(v)} 
            & = \Bigl |\int ^1_0{(1-t)\bigl ( 
                {d^2 \over dt^2} H^\bot (tu)-{d^2 \over dt^2} H^\bot (tv)
                \bigr ) dt} \Bigr |_{L^p}                                \cr
            & \leq \Bigl | {d^2 \over dt^2} H^\bot (tu)-
                           {d^2 \over dt^2} H^\bot (tv) \Bigr |_{L^p}\;. \cr
              }$$
Die Behauptung ergibt sich aus der folgenden Rechnung f�r die einzelnen 
Summanden.
$$\eqalign{ \Bigl | {d^2 \over dt^2} H^\bot_3 (tu)-
                    {d^2 \over dt^2} H^\bot_3 (tv)   \Bigr |                 
            \leq &  K  \bigl | g^{ij}_{tt}(tu)-g^{ij}_{tt}(tv) \bigr |   \cr
            \leq &  K  \bigl | |\nabla u|^2 - |\nabla v|^2     \bigr |   \cr
            \leq &  K  \bigl | |\nabla u| - |\nabla v|         \bigr |^2 \cr  
            \leq &  K  |\nabla (u-v)|^2                                  \cr
            \leq &  Kr |\nabla (u-v)|       \;.                          \cr
        }$$
Die Metrik und ihre Ableitungen sind punktweise Polynome in 
$u_{x_i},\ldots,u_{x_n}$, deswegen $\C^1$ und damit Lipschitz-stetig.
Dasselbe gilt auch f�r die Determinate der Metrik und ihrer Ableitungen.
$Q$ sei die gemeinsame Lipschitz-Konstante. Dann gilt
$$\eqalign{ \Bigl | {d^2 \over dt^2} H^\bot_2 (tu)-
                    {d^2 \over dt^2} H^\bot_2 (tv) \Bigr |
            \leq & 2 \bigl | g^{ij}_{tx_i}(tu)u^\bot_{x_i}-
                             g^{ij}_{tx_i}(tv)v^\bot_{x_i} \bigr |       \cr
                 & + t \bigl | g^{ij}_{ttx_i}(tu)u^\bot_{x_i}-
                             g^{ij}_{ttx_i}(tv)v^\bot_{x_i} \bigr |      \cr
         }$$
und folglich
$$\eqalign{& \bigl | g^{ij}_{tx_i}(tu)u^\bot_{x_i}
                   - g^{ij}_{tx_i}(tu)v^\bot_{x_i}
                   + g^{ij}_{tx_i}(tu)v^\bot_{x_i}
                   - g^{ij}_{tx_i}(tv)v^\bot_{x_i} \bigr |               \cr
      \leq &   \bigl | g^{ij}_{tx_i}(tu)
                       (u^\bot_{x_i} - v^\bot_{x_j})\bigr |
             + \bigl | (g^{ij}_{tx_i}(tv) - g^{ij}_{tx_i}(tv))
                       v^\bot_{x_i} \bigr |                              \cr
      \leq &   | g^{ij}_{tx_i}(tu) | |\nabla (u-v)|
             + |\nabla v| |g^{ij}_{tx_i}(tv) - g^{ij}_{tx_i}(tv)|        \cr
      \leq &   | g^{ij}_{tx_i}(tu) | |\nabla (u-v)|
             + |\nabla v|Q |\nabla (u - v)|                              \cr
      \leq & \bigl (K(|\nabla u|^2 + |\nabla ^2u| + |\nabla u||\nabla^2u|)
             + Q |\nabla v| \bigr )
               |\nabla (u-v)|                                            \cr
      \leq & Kr |\nabla (u-v)|.                                          \cr
         }$$
Der zweite Term und die Absch�tzungen f�r $H_1$ gehen analog.

$$\eqalign{\Bigl | {d^2 \over dt^2} H^\bot_4 (tu)- 
                    {d^2 \over dt^2} H^\bot (tv) \Bigr |
            \leq & 2 \bigl | g^{ij}_t(tu)u^\bot_{x_ix_j}-
                             g^{ij}_t(tv)v^\bot_{x_ix_j} \bigr |         \cr
                 & + t \bigl | g^{ij}_{tt}(tu)u^\bot_{x_ix_j}-
                               g^{ij}_{tt}(tv)v^\bot_{x_ix_j} \bigr |    \cr
          }$$
Der erste Term wird jetzt exemplarisch abgesch�tzt
$$\eqalign{& \bigl | g^{ij}_t(tu)u^\bot_{x_ix_j}
                   - g^{ij}_t(tu)v^\bot_{x_ix_j}
                   + g^{ij}_t(tu)v^\bot_{x_ix_j}
                   - g^{ij}_t(tv)v^\bot_{x_ix-j} \bigr |                 \cr
      \leq &   \bigl | g^{ij}_t(tu)
                       (u^\bot_{x_ix_j} - v^\bot_{x_jx_j})\bigr |
             + \bigl | (g^{ij}_t(tv) - g^{ij}_t(tv))
                       v^\bot_{x_ix_j} \bigr |                           \cr
      \leq &   | g^{ij}_t(tu) | |\nabla^2 (u-v)|
             + |\nabla^2v|Q |\nabla (u - v)|                             \cr
      \leq & K|\nabla u| |\nabla^2 (u-v)| + Q |\nabla v||\nabla (u-v)|   \cr
      \leq & Kr (|\nabla (u-v)| + |\nabla^2 (u-v)|).                     \cr 
         }$$\kasten


