\def\CTeXPreproc{Created by ctex v0.2.12, don't edit!}
\def\CTeXPreproc{Created by ctex v0.2.12, don't edit!}
\documentclass[11pt]{article}
\usepackage{mathrsfs}
\usepackage{amssymb}
\usepackage{amsfonts}
\usepackage{color,xcolor}
\usepackage{graphicx}
\usepackage{manfnt}
\usepackage{graphicx}
\textwidth  165mm \textheight 225mm \topmargin  -15mm
%\voffset  -65mm
\oddsidemargin  0mm
\renewcommand\baselinestretch{1.0}

\newtheorem{theorem}{Theorem}[section]
\newtheorem{corollary}{Corollary}[section]
\newtheorem{lemma}{Lemma}[section]
\newtheorem{proposition}{Proposition}[section]
\newtheorem{conjecture}{Conjecture}[section]
\newtheorem{definition}{Definition}[section]
\newtheorem{remark}{Remark}[section]
\newtheorem{example}{Example}[section]
\renewcommand{\theequation}{\thesection.\arabic{equation}}

\def\HS{{\mathrm{\tiny HS}}}
\def\eps{\varepsilon}\def\t{\tau}\def\vr{\varrho}\def\th{\theta}
\def\a{\alpha}\def\om{\omega}\def\Om{\Omega}\def\b{\beta}\def\p{\partial}
\def\d{\delta}\def\g{\gamma}\def\l{\lambda}\def\la{\langle}\def\ra{\rangle}
\def\[{{\Big[}}\def\]{{\Big]}}\def\<{{\langle}}\def\>{{\rangle}}\def\({{\Big(}}
\def\){{\Big)}}\def\bz{{\mathbf{z}}}\def\by{{\mathbf{y}}}\def\bx{{\mathbf{x}}}
\def\tr{{\rm tr}}\def\W{{\mathcal W}}\def\Ric{{\rm Ricci}}\def\Cap{\mbox{\rm Cap}}
\def\sgn{\mbox{\rm sgn}}\def\mathcalV{{\mathcal V}}\def\Law{{\mathord{{\rm Law}}}}
\def\dif{{\mathord{{\rm d}}}}\def\dis{{\mathord{{\rm \bf d}}}}\def\Hess{{\mathord{{\rm Hess}}}}
\def\min{{\mathord{{\rm min}}}}\def\Vol{\mathord{{\rm Vol}}}\def\bbbn{{\rm I\!N}}
\def\no{\nonumber}\def\={&\!\!=\!\!&}
\def\cA{{\mathcal A}}\def\cB{{\mathcal B}}\def\cC{{\mathcal C}}\def\cD{{\mathcal D}}
\def\cE{{\mathcal E}}\def\cF{{\mathcal F}}\def\cG{{\mathcal G}}\def\cH{{\mathcal H}}
\def\cI{{\mathcal I}}\def\cJ{{\mathcal J}}\def\cK{{\mathcal K}}\def\cL{{\mathcal L}}
\def\cW{{\mathcal L}}\def\cM{{\mathcal M}}\def\cN{{\mathcal N}}\def\cO{{\mathcal O}}
\def\cP{{\mathcal P}}\def\cQ{{\mathcal Q}}\def\cR{{\mathcal R}}\def\cS{{\mathcal S}}
\def\cT{{\mathcal T}}\def\cU{{\mathcal U}}\def\cV{{\mathcal V}}\def\cW{{\mathcal W}}
\def\cX{{\mathcal X}}\def\cY{{\mathcal Y}}\def\cZ{{\mathcal Z}}
\def\mA{{\mathbb A}}\def\mB{{\mathbb B}}\def\mC{{\mathbb C}}\def\mD{{\mathbb D}}
\def\mE{{\mathbb E}}\def\mF{{\mathbb F}}\def\mG{{\mathbb G}}\def\mH{{\mathbb H}}
\def\mI{{\mathbb I}}\def\mJ{{\mathbb J}}\def\mK{{\mathbb K}}\def\mL{{\mathbb L}}
\def\mM{{\mathbb M}}\def\mN{{\mathbb N}}\def\mO{{\mathbb O}}\def\mP{{\mathbb P}}
\def\mQ{{\mathbb Q}}\def\mR{{\mathbb R}}\def\mS{{\mathbb S}}\def\mT{{\mathbb T}}
\def\mU{{\mathbb U}}\def\mV{{\mathbb V}}\def\mW{{\mathbb W}}\def\mX{{\mathbb X}}
\def\mY{{\mathbb Y}}\def\mZ{{\mathbb Z}}
\def\bP{{\mathbf P}}\def\bB{{\mathbf B}}
\def\sA{{\mathscr A}}\def\sB{{\mathscr B}}\def\sC{{\mathscr C}}\def\sD{{\mathscr D}}
\def\sE{{\mathscr E}}\def\sF{{\mathscr F}}\def\sG{{\mathscr G}}\def\sH{{\mathscr H}}
\def\sI{{\mathscr I}}\def\sJ{{\mathscr J}}\def\sK{{\mathscr K}}\def\sL{{\mathscr L}}
\def\sM{{\mathscr M}}\def\sN{{\mathscr N}}\def\sO{{\mathscr O}}\def\sP{{\mathscr P}}
\def\sQ{{\mathscr Q}}\def\sR{{\mathscr R}}\def\sS{{\mathscr S}}\def\sT{{\mathscr T}}
\def\sU{{\mathscr U}}\def\sV{{\mathscr V}}\def\sW{{\mathscr W}}\def\sX{{\mathscr X}}
\def\sY{{\mathscr Y}}\def\sZ{{\mathscr Z}}
\def\fM{{\mathfrak M}}\def\fA{{\mathfrak A}}
\def\geq{\geqslant}\def\leq{\leqslant}
\def\c{\mathord{{\bf c}}}\def\div{\mathord{{\rm div}}}\def\iint{\int^t_0\!\!\!\int}
\def\sb{{\mathfrak b}}
\def\bH{{\mathbf H}}\def\bW{{\mathbf W}}\def\bP{{\mathbf P}}
\def\bA{{\mathbf A}}\def\bT{{\mathbf T}}\def\tr{{\mathrm t}}
\begin{document}
\title{\bf
Existence of stochastic entropy solutions for stochastic scalar
balance laws with Lipschitz vector fields\thanks{This work was
partially supported by NNSF of China
       (Grant No. 11171122)}}
\author{Jinlong  Wei$^1$, Liang Ding$^2$, Bin Liu$^3$\thanks{Corresponding author, E-mail:
\ binliu@mail.hust.edu.cn,  Fax: + 86 27 87543231} \\
$^1$School of Statistics and Mathematics, Zhongnan University of
Economics  \\ and Law, Wuhan 430073, Hubei, P.R.China
\\ $^2$Department of Basis Education
Dehong Vocational College \\ Dehong, Yunnan, 678400, P.R.China
\\
$^3$School of Mathematics and Statistics, Huazhong University of
Science \\ and Technology, Wuhan 430074, Hubei, P.R.China}
\date{}
 \maketitle
\noindent{\hrulefill}
\vskip1mm\noindent
 {\bf Abstract}  In this paper, we consider a scalar stochastic balance law and gain
the existence for stochastic entropy solutions. Our proof relies on
the BGK approximation and the generalized It\^{o} formula. Moreover,
as an application, we derive the existence of stochastic entropy
solutions for stochastic Buckley-Leverett type equations.
  \vskip2mm\noindent
{\bf Keywords:}  Existence; Stochastic entropy solution; BGK
approximation; It\^{o} formula
  \vskip2mm\noindent
{\bf MSC (2010):} 35L65; 60H15; 60H30
 \vskip0mm\noindent{\hrulefill}
\section{Introduction}\label{sec1}\setcounter{equation}{0}
\setcounter{equation}{0} In this paper, we study the first-order
scalar balance law with Stratonovich type perturbations:
\begin{eqnarray}\label{1.1}
\partial_t\rho(t,x)+\mbox{div}_x(B(\rho))
+\partial_{x_i}\rho(t,x)\circ\dot{M}_i(t)=A(t,\rho), \quad \mbox{in}
\ \ \Omega\times (0,\infty)\times \mR^d,
\end{eqnarray}
where $M_i(t)=\int^t_0\sigma_{i,j}(s)dW_j(s)$, ($1\leq i,j\leq d$)
and $\sigma_{i,j}\in L^2_{loc}([0,\infty))$. The flux function $B$
is assumed to be of class $\cC^1: B\in \cC^1(\mR;\mR^d)$ and the
forcing $A$ is supposed to be Lipschitz: $A\in
L^1_{loc}([0,\infty);W^{1,\infty}(\mR))$ and $A(t,0)=0$. To
formulate the Cauchy problem, we presume
\begin{eqnarray}\label{1.2}
\rho(t,x)|_{t=0}=\rho_0(x), \quad  \mbox{in} \ \  \mR^d,
\end{eqnarray}
here $\rho_0$ is a non-random function.
  \vskip1mm\par
The Cauchy problem for $(1.1), (1.2)$ in the case of
$\sigma_{i,j}=0$ has been studied in Kru\u{z}kov [1]. Without the
hypothesis $A(t,0)=0$, he gains the uniqueness of entropy solutions
as well as the existence.
  \vskip1mm\par
In the case of the perturbations are quasi-linear dependent, i.e.
\begin{eqnarray}\label{1.3}
\partial_t\rho(t,x)+\partial_{x_i}B_i(\rho)\circ\dot{W}_i(t)=A(t,\rho), \quad
 \mbox{in}
\ \  \Omega\times (0,\infty)\times \mR^d,
\end{eqnarray}
Lions, Perthame and Souganidis [2] develop a pathwise theory for
weak $L^1\cap L^\infty$-solutions for which $A$ vanishes and $B\in
\cC^2(\mR;\mR^d)$.
 \vskip1mm\par
When the stochastic perturbations in $(1.1)$ is replaced by a
multiplicative noise, which in our present setting takes the form
\begin{eqnarray}\label{1.4}
\partial_t\rho(t,x)+\mbox{div}_x(B(\rho))
=A(t,\rho)\dot{W}(t), \quad  \mbox{in} \ \
\Omega\times(0,\infty)\times \mR^d,
\end{eqnarray}
it has been studied by many other researchers [3-5]. For example, in
[3], Chen, Ding and Karlsen concern with $(1.4), (1.1)$ with
$A(t,\rho)=A(\rho)$, then they supply an existence theory of
stochastic entropy solutions for $\cC^2$ flux.
   \vskip1mm\par
In above  papers, existence of (stochastic) entropy solutions are
proved via approximation by (stochastic) parabolic equations. Using
a differential philosophy, Hofmanov\'{a} in [6] proceeds a
stochastic BGK approximation, and the existence of stochastic
entropy solution is proved for $\cC^{4,\alpha}$ ($\alpha>0$) flux
and Lipschitz forcing.
 \vskip1mm\par
All above mentioned works for stochastic balance laws are
concentrated on $\cC^2$ (or $\cC^{4,\alpha}$) flux, there are
relatively few research works for well-posedness of $(1.1), (1.2)$
for $\cC^1$ flux. Our purpose now is to raise a well-posedness
theory for $L^1\cap L^\infty$-solutions to $(1.1). (1.2)$ on
stochastic entropy solutions to the level of Kru\v{z}kov theory. Our
present work is a fellow-up of Wei and Liu's work [7]. In [7], the
authors put forward the following result:
\begin{lemma} \textbf{(Stochastic kinetic formula)} Assume that
$\rho_0\in L^1(\mR^d)\cap L^\infty(\mR^d)$. (i) Let $\rho$ be a
stochastic entropy solution of $(1.1), (1.2)$ and set
\begin{eqnarray}\label{1.5}
u(t,x,v)=\chi_{\rho(t,x)}(v)=\left\{
\begin{array}{ll} \ \ 1, \ \ \mbox{when} \ 0<v<\rho, \\
-1, \ \  \mbox{when} \ \rho<v<0, \\
\ \ 0, \  \ \mbox{otherwise},
\end{array}
\right.
\end{eqnarray}
then
\begin{eqnarray}\label{1.6}
u\in L^\infty(\Omega;L^\infty_{loc}([0,\infty);L^\infty({\mathbb
R}^d_x;L^1({\mR}_v ))))\cap
\cC([0,\infty);L^1({\mR}^d_x\times{\mR}_v\times \Omega)),
\end{eqnarray}
and it is a stochastic weak solution of the linear stochastic
transport problem
\begin{eqnarray}\label{1.7}
\left\{
  \begin{array}{ll}
\partial_tu+b(v)\cdot \nabla_{x} u
+\partial_{x_i}u\circ \dot{M}_i(t)
 +A(t,v)\partial_
vu=\partial_vm, \ \
 \mbox{in}
\ \  \Omega\times (0,\infty)\times \mR^d_x\times\mR_v,\\
u(t,x,v)|_{t=0}=\chi_{\rho_0}(v), \ \  \mbox{in} \ \
\mR^d_x\times\mR_v,
  \end{array}
\right.
\end{eqnarray}
where $b=B^\prime$, $0\leq m\in
L^1(\Omega;\cD^{\prime}([0,\infty)\times {\mR^d_x}\times \mR_v))$,
satisfying, for any $T>0$ and for almost all $\omega\in \Omega$, $m$
is bounded on $[0,T]\times \mR^d_x\times \mR_v$, supported in
$[0,T]\times \mR^d_x\times [-K,K]$
($K=\|\rho\|_{L^\infty((0,T)\times \mR^d\times \Omega)}$),
continuous in $t$, here the continuous is interpreted
\begin{eqnarray}\label{1.8}
 m([0,s]\times \mR^d_x\times\mR_v)\rightarrow m([0,t]\times \mR^d_x\times\mR_v), \ \ \mbox{as} \ \
 s\rightarrow t.
\end{eqnarray}
 \vskip1mm\par
(ii) Let $u\in L^\infty(\Omega;L^\infty_{loc}([0,\infty);L^\infty(
\mR^d_x;L^1(\mR_v))))\cap\cC([0,\infty);L^1({\mR}^d_x\times
\mR_v\times \Omega))$ be a stochastic weak solution of $(1.7)$ (i.e.
$u$ yields $(1.7)$ in the sense of distributions), with $m$ meeting
the properties stated in $(i)$. Then
\begin{eqnarray}\label{1.9}
\rho (t,x)=\int_{\mR}u(t,x,v)dv \in L^\infty(\Omega;
L^\infty_{loc}([0,\infty);L^\infty(\mR^d)))\cap
\cC([0,\infty);L^1({\mR}^d\times\Omega)),
\end{eqnarray}
and it is a stochastic entropy solution of $(1.1), (1.2)$.
 \end{lemma}
Then they derive the uniqueness of stochastic entropy solutions for
$(1.1), (1.2)$. Whence, to establish the well-posedness for $(1.1),
(1.2)$, it remains to found the existence of solutions, and it is
the main interest of the paper here.
  \vskip1mm\par
This paper is organized as follows. In Section 2, we review some
notions. Section 3 is devoted to the existence on stochastic entropy
solutions for $(1.1), (1.2)$.
   \vskip1mm\par
As usual, the notation used here is mostly standard. $\cD(\mR^d)$ is
the space of all $\varphi\in \cC^\infty(\mR^d)$ with compact
support. Correspondingly, $\cD_+(\mR^d)$ represents the non-negative
elements in $\cD(\mR^d)$. $\mN$ denotes the set consisting of all
natural numbers. $W(t)=(W_1(t), W_2(t), _{\cdots}, W_d(t))^\top$
stands for the standard $d$-dimensional Wiener procss on the
classical Wiener space ($\Omega, \cF,P,(\cF_{t})_{t\geq 0}$), i.e.
$\Omega$ is the space of all continuous functions from $[0,\infty)$
to $\mR^d$ with locally uniform convergence topology, $\cF$ is the
Borel $\sigma$-field, $P$ is the Wiener measure, $(\cF_t)_{t\geq 0}$
is the natural filtration generated by the coordinate process
$W_t(\omega)=\omega_t$. The stochastic integration with a notation
$\circ$ is interpreted in Stratonovich sense and the others is
It\^{o}'s. The $C(T)$ denotes a positive constant depends only on
$T$, whose values may change in different places. For a given
measurable function $f$, $f^+$ is its positive portion, defined by
$1_{f\geq0}f$, and $f^-=[-f]^+$. The summation convention is
enforced throughout this article, wherein summation is understood
with respect to repeated indices.
\section{Stochastic entropy solutions}\label{sec2}
\setcounter{equation}{0} In this short section, we introduce some
notions and review the stochastic kinetic formula for future use.
\begin{definition} Let $\rho_0\in L^\infty(\mR^d)\cap L^1(\mR^d)$. $\rho\in L^\infty(\Omega;
L^\infty_{loc}([0,\infty);L^\infty(\mR^d)))\cap
\cC([0,\infty);L^1(\mR^d\times \Omega))$ is a stochastic weak
solution of $(1.1), (1.2)$, if for any $\varphi\in {\mathcal
D}(\mR^d)$, with probability one, the below identity
\begin{eqnarray}\label{2.1}
&&\int_{\mR^d}\varphi(x)\rho(t,x)dx -\int^t_0\int_{{\mathbb
R^d}}B(\rho)\cdot\nabla_x\varphi(x)dxds-\int^t_0M_i(\circ
ds)\int_{{\mR^d}}\partial_{x_i}\varphi(x)\rho(s,x)dx \cr \cr &= &
\int_{\mR^d}\varphi(x)\rho_0(x)dx
+\int^t_0\int_{\mR^d}A(s,\rho)\varphi(x)dxds,
\end{eqnarray}
holds true, for all $t\in [0,\infty)$. The stochastic weak solution
is call a stochastic entropy solution, if for any  $\eta(\rho)\in
\Xi$,
\begin{eqnarray}\label{2.3}
\partial_t\eta(\rho)+\mbox{div}(Q(\rho))+
\partial_{x_i}\eta(\rho) \circ \dot{M}_i(t) \leq h(t,\rho), \
P-a.s. \ \omega\in\Omega,
\end{eqnarray}
in the sense of distributions, where
\begin{eqnarray}\label{2.4}
Q(\rho)=\int^\rho \eta^\prime(v)b(v)dv, \
h(t,\rho)=A(t,\rho)\eta^\prime(\rho),
\end{eqnarray}
and
$$
\Xi=\{c_0\rho+\sum_{k=1}^nc_k|\rho-\rho_k|, \ \ c_0, \rho_k, c_k\in
\mR \ \ \mbox{are constants}\}, \ \ n\in \mN.
$$
\end{definition}
\section{Existence of stochastic entropy solutions}\label{sec3}
\setcounter{equation}{0}  In this section, we intend to found the
fundamental existence results on stochastic entropy solutions to
$(1.1), (1.2)$. Inspiring by Lemma 2.1 stated in introduction, it
suffices to establish the existence on stochastic weak solutions for
$(1.7)$. Initially, we give a remark.
\begin{remark}
$m$ is continuous in $t$ (see $(1.8)$), so $u\in
L^\infty(\Omega;L^\infty_{loc}([0,\infty);L^\infty(
\mR^d_x;L^1(\mR_v))))\cap\cC([0,\infty);L^1({\mR}^d_x\times
\mR_v\times \Omega))$ is a stochastic weak solution of $(1.7)$ if
and only if, for any $\phi(x,v)\in \cD(\mR^d_x\times\mR_v)$, with
probability one,
\begin{eqnarray*}
&&\int_{\mR^d_x\times\mR_v}\phi(x,v)u(t,x,v)dxdv
-\int^t_0\int_{\mR^d_x\times\mR_v}b(v)\cdot\nabla_x\phi(x,v)dxdvds
 \cr \cr &=& \int_{\mR^d_x\times\mR_v}\phi(x,v)u_0(x,v)dxdv
+\int^t_0\int_{\mR^d_x\times\mR_v}\partial_v[A(s,v)\phi(x,v)]u(s,x,v)dxdvds
\cr\cr&& - \int^t_0\int_{\mR^d_x\times\mR_v}\partial_v\phi(x,v)
m(dx,dv,ds) +\int^t_0M_i(\circ
ds)\int_{\mR^d_x\times\mR_v}\partial_{x_i}\phi(x,v)u(s,x,v)dxdv
\end{eqnarray*}
is legitimate, for all $t\in [0,\infty)$.
\end{remark}
   \vskip0mm\par
 We are now in a position to give our main result.
\begin{theorem} \textbf{(Existence)} \ Let $B$, $\sigma$ and $A$
yield the conditions stated in introduction. If $\rho_0\in
L^\infty(\mR^d)\cap L^1(\mR^d)$, then there exists a stochastic
entropy solution of the Cauchy problem $(1.1), (1.2)$.
 \end{theorem}
\textbf{Proof.} The conclusion will be reached in three steps.
  \vskip2mm\noindent
$\bullet$ \textbf{Step 1:} $\sigma=0$. Now $(1.7)$ becomes to
\begin{eqnarray}\label{3.1}
\left\{\begin{array}{ll}\partial_tu(t,x,v)+b(v)\cdot\nabla_xu(t,x,v)+
A(t,v)\partial_vu(t,x,v)
=\partial_vm, \ \ \mbox{in} \ \ (0,\infty)\times \mR^d_x\times \mR_v, \\
u(t,x,v)|_{t=0}=\chi_{\rho_0(x)}(v),  \ \ \mbox{in} \ \
\mR^d_x\times \mR_v.
  \end{array}
\right.
\end{eqnarray}
 \vskip1mm\par
We begin with building the existence of weak solutions for $(3.1)$
by using the BGK approximation, i.e., for $\varepsilon>0$, we regard
$(3.1)$ as the $\varepsilon\rightarrow 0$ limit of the
integro-differential equation
\begin{eqnarray}\label{3.2}
\left\{\begin{array}{ll} \partial_
tu_\varepsilon(t,x,v)+b(v)\cdot\nabla_xu_\varepsilon+A(t,v)
\partial_vu_\varepsilon= \frac{1}{\varepsilon}
{\big[} \chi_{\rho_\varepsilon(t,x)} -u_\varepsilon{\big]}, \ \ \
\mbox{in} \ \ (0,\infty)\times\mR^d_x\times \mR_v, \\
u_\varepsilon(t,x,v)|_{t=0}=\chi_{\rho_0(x)}(v), \ \ \ \mbox{in} \ \
\mR^d_x\times \mR_v,
\end{array}
\right.
\end{eqnarray}
where $\rho_\varepsilon(t,x)=\int_{\mathbb
R}u_\varepsilon(t,x,v)dv.$
  \vskip2mm\noindent
$\bullet$ \textbf{Assertion 1:} $(3.2)$ is well-posed in $
L^\infty_{loc}([0,\infty);L^\infty(\mR^d_x\times\mR_v))\cap
{\mathcal C}([0,\infty);L^1(\mR^d_x\times\mR_v))$.
 \vskip2mm\par
Clearly, $(3.2)_1$ grants an equivalent presentation
\begin{eqnarray*}
\partial_tZ_\varepsilon+b(v)\cdot\nabla_xZ_\varepsilon+A(t,v)
\partial_vZ_\varepsilon=
\frac{1}{\varepsilon}e^{\frac{t}{\varepsilon}}
\chi_{e^{-\frac{t}{\varepsilon}}\tilde{\rho}_\varepsilon}(v),
\end{eqnarray*}
here
$$
Z_\varepsilon(t,x,v)=e^{\frac{t}{\varepsilon}}u_\varepsilon(t,x,v),
\ \tilde{\rho}_\varepsilon=\int_{\mR}Z_\varepsilon(t,x,v)dv.
$$
Due to the assumptions
$$
B\in \cC^1(\mR;\mR^d), \ A\in
L^1_{loc}([0,\infty);W^{1,\infty}(\mR)),
$$
there is a unique global solution to the ODE
\begin{eqnarray}\label{3.3}
\frac{d}{dt}(X(t,x,v), V(t,v))^\top=(b(V),  A(t,V))^\top,  \ \
\mbox{with} \ \ (X(t,x,v), V(t,v))^\top|_{t=0}=(x,v)^\top,
\end{eqnarray}
for any $(x,v)\in \mR^d_x\times\mR_v$.
 \vskip1mm\par
Therefore, along the direction $(3.3)$,
$$
Z_\varepsilon(t,X(t),V(t))=\frac{1}{\varepsilon}\int^t_0e^{\frac{s}
{\varepsilon}}\chi_{e^{-\frac{s}{\varepsilon}}\tilde{\rho}_\varepsilon(s,X(s,x,v))}
(V(s,v))ds+ \chi_{\rho_0(x)}(v),
$$
i.e.
\begin{eqnarray*}
u_\varepsilon(t,X(t),V(t))=\frac{1}{\varepsilon}\int^t_0e^{\frac{s-t}{\varepsilon}}
\chi_{\rho_\varepsilon(s,X(s,x,v))}(V(s,v))ds+
e^{-\frac{t}{\varepsilon}}\chi_{\rho_0(x)}(v).
\end{eqnarray*}
 \vskip1mm\par
Define $J(t,V)=|\partial_vV(t,v)|$, thanks to Euler's formula, then
\begin{eqnarray}\label{3.4}
\exp(-\int^t_0[\partial_vA(s,V(s))]^-ds)\leq J(t,V)\leq
\exp(\int^t_0[\partial_vA(s,V(s))]^+ds),
\end{eqnarray}
whence the inverse of the mapping $(x,v)^\top \mapsto
 (X,V)^\top$ exists and it forms a flow of homeomorphic. We thus have
\begin{eqnarray}\label{3.5}
u_\varepsilon(t,x,v)=\frac{1}{\varepsilon}\int^t_0e^{\frac{s-t}
{\varepsilon}}
\chi_{\rho_\varepsilon(s,X_{t,s}(x,v))}(V_{t,s}(v))ds+
e^{-\frac{t}{\varepsilon}}\chi_{\rho_0(X_{t,0}(x,v))}(V_{t,0}(v)),
\end{eqnarray}
where
$(X_{t,s}(x,v),V_{t,s}(v))^\top=[(X_{s,t}(x,v),V_{s,t}(v))^\top]^{-1}$,
i.e.
\begin{eqnarray*}
\left\{
  \begin{array}{ll}
 \frac{d}{dt}(X_{s,t}(x,v),V_{s,t}(v))^\top=(b(V_{s,t}),
A(t,V_{s,t}))^\top, \ t\geq s, \\ \\ (X_{s,t}(x,v),
V_{s,t}(v))^\top|_{t=s}=(X(s,x,v),V(s,v))^\top,
  \end{array}
\right.
\end{eqnarray*}
and
$(X_{s,t}(x,v),V_{s,s}(v))^\top=(X(t,X(s,x,v),V(s,v)),V(t,V(s,v)))^\top$.
 \vskip2mm\par
For any $u\in
L^\infty_{loc}([0,\infty);L^\infty(\mR^d_x\times\mR_v))\cap
\cC([0,\infty);L^1(\mR^d_x\times\mR_v))$, we define a mapping
$S_\varepsilon$ by $(3.5)$:
\begin{eqnarray}\label{3.6}
(S_\varepsilon u)(t,x,v)=\frac{1}{\varepsilon}\int^t_0e^{\frac{s-t}
{\varepsilon}} \chi_{\rho^u(s,X_{t,s}(x,v))}(V_{t,s}(v))ds+
e^{-\frac{t}{\varepsilon}}\chi_{\rho_0^u(X_{t,0}(x,v))}(V_{t,0}(v)),
\end{eqnarray}
here
$$
\rho^u(t,x)=\int_{\mR}u(t,x,v)dv, \ \ \ \rho_0^u(x)=\int_{{\mathbb
R}}u(0,x,v)dv=\rho_0(x).
$$
We claim that $S_\varepsilon$ is well-defined in
$L^\infty_{loc}([0,\infty);L^\infty(\mR^d_x\times\mR_v))\cap
\cC([0,\infty);L^1(\mR^d_x\times\mR_v))$ and locally (in time)
contractive in $\cC([0,\infty);L^1(\mR^d_x\times\mR_v))$.
  \vskip1mm\par
Initially, we collate that $(3.6)$ is well-defined. Indeed,
\begin{eqnarray}\label{3.7}
\|S_\varepsilon u\|_{L^\infty([0,T]\times \mR^d_x\times\mR_v)}\leq
1,
\end{eqnarray}
and for any $0<T<\infty$,
\begin{eqnarray}\label{3.8}
&&\sup_{0\leq t\leq T}{\Big |}
\frac{1}{\varepsilon}\int^t_0e^{\frac{s-t}
{\varepsilon}}ds\!\!\int_{\mR^d_x\times \mR_v}
\!\!\chi_{\rho^u(s,X_{t,s}(x,v))}(V_{t,s}(v))dxdv +
e^{-\frac{t}{\varepsilon}}\!\!\int_{\mR^d_x\times \mR_v}\!\!
\chi_{\rho_0^u(X_{t,0}(x,v))}(V_{t,0}(v))dxdv{\Big |} \cr \cr &=&
\sup_{0\leq t\leq T}{\Big |}
\frac{1}{\varepsilon}\int^t_0e^{\frac{s-t}
{\varepsilon}}ds\int_{\mR^d_x\times \mR_v} \chi_{\rho^u(s,x)}(v)
\exp(\int^t_s\partial_v A(r,V_{s,r}(v))dr) dxdv \cr \cr &&   + \
e^{-\frac{t}{\varepsilon}}\int_{\mR^d_x\times {\mathbb R}_v}
\chi_{\rho_0^u(x)}(v) \exp(\int^t_0\partial_v
A(r,V_{0,r}(v))dr)dxdv{\Big |} \cr \cr &\leq&
\exp(\int^T_0\|[\partial_vA]^+\|_{L^\infty(\mR)}(t)dt){\Big
[}(1-e^{-\frac{T}{\varepsilon}})\|u\|_{\cC([0,T]; L^1(\mR^d_x\times
\mR_v))}+\|\rho_0^u\|_{L^1(\mR^d_x\times \mR_v)}{\Big]},
\end{eqnarray}
thus $(3.6)$ is meaningful.
 \vskip1mm\par
For any $f,g\in L^\infty_{loc}([0,\infty);L^\infty(\mR^d_x\times
\mR_v))\cap \cC([0,\infty);L^1(\mR^d_x\times\mR_v))$, an analogue
calculation of $(3.8)$ also leads to
\begin{eqnarray}\label{3.9}
 && \|S_\varepsilon f-S_\varepsilon g\|_{\cC([0,T];
L^1(\mR^d_x\times \mR_v))}
 \cr \cr
&\leq& \sup_{0\leq t\leq T}{\Big |}
\frac{1}{\varepsilon}\int^t_0e^{\frac{s-t}
{\varepsilon}}ds\int_{\mR^d_x\times \mR_v}
|\chi_{\rho^f(s,X_{t,s}(x,v))}(V_{t,s}(v))-
\chi_{\rho^g(s,X_{t,s}(x,v))}(V_{t,s}(v))| dxdv
 \cr \cr &&+ \
e^{-\frac{t}{\varepsilon}}\int_{\mR^d_x\times \mR_v}
|\chi_{\rho_0^f(X_{t,0}(x,v))}(V_{t,0}(v))-
\chi_{\rho_0^g(X_{t,0}(x,v))}(V_{t,0}(v))|dxdv{\Big |} \cr \cr &=&
\sup_{0\leq t\leq T}{\Big |}
\frac{1}{\varepsilon}\int^t_0e^{\frac{s-t}
{\varepsilon}}ds\int_{\mR^d_x\times \mR_v}
|\chi_{\rho^f(s,x)}(v)-\chi_{\rho^g(s,x)}(v)|
\exp(\int^t_s\partial_v A(r,V_{s,r})dr) dxdv \cr \cr && + \
e^{-\frac{t}{\varepsilon}}\int_{\mR^d_x \times \mR_v}
|\chi_{\rho_0^f(x)}(v)-\chi_{\rho_0^g(x)}(v) |
\exp(\int^t_0\partial_v A(r,V_{0,r}(v))dr)dxdv{\Big |} \cr \cr
&\leq& \exp(\int^T_0\!\!\|[\partial_vA]^+\|_{L^\infty(\mR)}dt){\Big
[}(1\!-\!e^{-\frac{T}{\varepsilon}})\|f-g\|_{\cC([0,T];
L^1(\mR^d_x\times \mR_v))}\!+\!\|f_0-g_0\|_{L^1(\mR^d_x\times
\mR_v)}{\Big]}. \
\end{eqnarray}
In particular, if $f_0=g_0=\chi_{\rho_0}$, from $(3.9)$,
$$
\|S_\varepsilon f-S_\varepsilon g\|_{\cC([0,T]; L^1(\mR^d_x\times
\mR_v))}\leq \exp(\int^T_0\|[\partial_vA]^+\|_{L^\infty(\mR)}(t)dt)
(1-e^{-\frac{T}{\varepsilon}})\|f-g\|_{{\mathcal C}([0,T];
L^1(\mR^d_x\times \mR_v))},
$$
for any $T>0$.
 \vskip1mm\par
Given above $T>0$ we select $T_1>0$ so small that
$\exp(\int^T_0\|[\partial_vA]^+\|_{L^\infty(\mR)}(t)dt)
(1-e^{-\frac{T_1}{\varepsilon}})<1$. Then we apply the Banach fixed
point theorem to find a unique $u_\varepsilon \in
\cC([0,T_1];L^1(\mR^d_x\times\mR_v))$ solving the Cauchy problem
$(3.2)$. By $(3.7)$, $u_\varepsilon \in
L^\infty([0,T];L^\infty(\mR^d_x\times\mR_v))$, so
$u_\varepsilon(T_1)\in L^1(\mR^d_x\times\mR_v))\cap
L^\infty(\mR^d_x\times\mR_v)$. We then repeat the argument above to
extend our solution to the time interval $[T_1,2T_1]$. Continuing,
after finitely many steps we construct a solution existing on the
interval $(0,T)$ for any $T>0$. From this, we demonstrate that there
exists a unique $u_\varepsilon \in
\cC([0,\infty);L^1(\mR^d_x\times\mR_v))\cap
L^\infty_{loc}([0,\infty);L^\infty(\mR^d_x\times\mR_v))$ solving the
Cauchy problem $(3.2)$.
 \vskip2mm\noindent
$\bullet$ \textbf{Assertion 2: (Comparison principle)}. For any
$\rho_0,\tilde{\rho_0}\in  L^\infty(\mR^d)\cap L^1(\mR^d)$, the
allied solutions $u_\varepsilon$ and $\tilde{u}_\varepsilon$ of
$(3.2)$ satisfy
\begin{eqnarray}\label{3.10}
\|[u_\varepsilon(t)-\tilde{u}_\varepsilon(t)]^+\|_{L^1(\mR^d_x\times
\mR_v)} &\leq&
\exp(\int^t_0\|[\partial_vA]^+\|_{L^\infty(\mR)}(s)ds)\|[\chi_{\rho_0}-\chi_{\tilde{\rho}_0}]^+\|_{L^1(\mR^d_x\times
\mR_v)} \cr\cr&=&
\exp(\int^t_0\|[\partial_vA]^+\|_{L^\infty(\mR)}(s)ds)
\|[\rho_0-\tilde{\rho}_0]^+\|_{L^1(\mR^d)},
\end{eqnarray}
\begin{eqnarray}\label{3.11}
\|\rho_\varepsilon(t)-\tilde{\rho}_\varepsilon(t)\|_{L^1(\mR^d)}\leq
\exp(\int^t_0\|[\partial_vA]^+\|_{L^\infty(\mR)}(s)ds)
\|\rho_0-\tilde{\rho}_0\|_{L^1(\mR^d)},
\end{eqnarray}
\begin{eqnarray}\label{3.12}
\|\rho_\varepsilon(t)\|_{L^\infty(\mR^d)}\leq
\exp(\int^t_0\|[\partial_vA]^+\|_{L^\infty(\mR)}(s)ds)
\|\rho_0\|_{L^\infty(\mR^d)}.
\end{eqnarray}
Furthermore, if $\rho_0\leq \tilde{\rho}_0$, for for almost all
$(t,x,v)\in (0,\infty)\times \mR^d_x\times \mR_v$, and almost all
$(t,x)\in (0,\infty)\times \mR^d$,
\begin{eqnarray}\label{3.13}
u_\varepsilon(t,x,v)\leq\tilde{u}_\varepsilon(t,x,v),\quad
\rho_\varepsilon(t,x)\leq \tilde{\rho}_\varepsilon(t,x).
\end{eqnarray}
 \vskip1mm\par
$(3.13)$ holds mutatis mutandis from $(3.10)$ and $(1.5)$, it is
sufficient to show $(3.10)-(3.12)$. Since the calculation for
$(3.11)$ and $(3.12)$ is analogue of $(3.10)$, we only show $(3.10)$
here. Let
$\lambda_\varepsilon=[u_\varepsilon-\tilde{u}_\varepsilon]^+$, by a
tedious approximation argument, it meets
\begin{eqnarray}\label{3.14}
\partial_t\lambda_\varepsilon(t,x,v)+b(v)\cdot\nabla_x\lambda_\varepsilon+A(t,v)
\partial_v\lambda_\varepsilon=
\frac{1}{\varepsilon} {\big[}
\chi_{\rho_\varepsilon(t,x)}-\chi_{\tilde{\rho}_\varepsilon(t,x)}
-(u_\varepsilon-\tilde{u}_\varepsilon){\big]}\mbox{sign}
\lambda_\varepsilon,
\end{eqnarray}
in $(0,\infty)\times\mR^d_x\times\mR_v$, with the initial data
\begin{eqnarray}\label{3.15}
\lambda_\varepsilon|_{t=0}=[\chi_{\rho_0(x)}(v)-\chi_{\tilde{\rho}_0(x)}(v)]^+,
\quad \mbox{in} \ \ \mR^d_x\times\mR_v.
\end{eqnarray}
 \vskip1mm\par
Notice that,
\begin{eqnarray}\label{3.16}
{\big[}
\chi_{\rho_\varepsilon(t,x)}-\chi_{\tilde{\rho}_\varepsilon(t,x)}
-(u_\varepsilon-\tilde{u}_\varepsilon){\big]}\mbox{sign}
\lambda_\varepsilon={\big[}
\chi_{\rho_\varepsilon(t,x)}-\chi_{\tilde{\rho}_\varepsilon(t,x)}]\mbox{sign}
\lambda_\varepsilon -\lambda_\varepsilon,
\end{eqnarray}
and
\begin{eqnarray}\label{3.17}
\int_{\mR}{\big[}
\chi_{\rho_\varepsilon(t,x)}(v)-\chi_{\tilde{\rho}_\varepsilon(t,x)}(v)]\mbox{sign}
\lambda_\varepsilon(t,x,v)dv\leq \int_{\mR}
\lambda_\varepsilon(t,x,v)dv.
\end{eqnarray}
 \vskip1mm\par
Indeed, when $\rho_\varepsilon\leq \tilde{\rho}_\varepsilon$,
$(3.17)$ is nature and reversely,
\begin{eqnarray*}
\int_{\mR}{\big[}
\chi_{\rho_\varepsilon(t,x)}(v)-\chi_{\tilde{\rho}_\varepsilon(t,x)}(v)]\mbox{sign}
\lambda_\varepsilon(t,x,v)dv &\leq& \int_{\mR}{\big[}
\chi_{\rho_\varepsilon(t,x)}(v)-\chi_{\tilde{\rho}_\varepsilon(t,x)}(v)]dv
\cr \cr &=&\int_{\mR} [u_\varepsilon-\tilde{u}_\varepsilon]dv \cr
\cr &\leq&\int_{\mR} \lambda_\varepsilon(t,x,v)dv.
\end{eqnarray*}
By $(3.16), (3.17)$, from $(3.14)$ it follows that
\begin{eqnarray*}
\partial_t\int_{\mR}\lambda_\varepsilon(t,x,v)dv+
\int_{\mR}b(v)\cdot\nabla_x\lambda_\varepsilon dv\leq
\int_{\mR}\partial_vA(t,v) \lambda_\varepsilon dv\leq
\|[\partial_vA(t)]^+\|_{L^\infty(\mR)}\int_{\mR}\lambda_\varepsilon
dv,
\end{eqnarray*}
which suggests that for any $\varphi\in \cD(\mR^d)$,
\begin{eqnarray*}
\frac{d}{dt}\int_{\mR^d_x\times\mR_v}\lambda_\varepsilon(t,x,v)\varphi(x)dxdv
\leq
\int_{\mR^d_x\times\mR_v}b(v)\cdot\nabla_x\varphi\lambda_\varepsilon
dxdv +
\|[\partial_vA(t)]^+\|_{L^\infty(\mR)}\int_{\mR^d_x\times\mR_v}
\lambda_\varepsilon\varphi dxdv.
\end{eqnarray*}
For any $k\in \mN$, we can choose $\varphi$ such that for any $0\leq
|x|\leq k$, $\varphi(x)=1$, then by letting $k$ tend to infinity,
one deduces
\begin{eqnarray}\label{3.18}
\frac{d}{dt}\int_{\mR^d_x\times\mR_v}\lambda_\varepsilon(t,x,v)dxdv
\leq \|[\partial_vA(t)]^+\|_{L^\infty(\mR)}\int_{\mR^d_x\times\mR_v}
\lambda_\varepsilon (t,x,v)dxdv.
\end{eqnarray}
   \vskip1mm\par
On account of the fact: for any $\alpha_1,\alpha_2\in
\mR$,
\begin{eqnarray}\label{3.19}
\int_{\mR}[\chi_{\alpha_1}(v)-\chi_{\alpha_2}(v)]^+dv=[\alpha_1-\alpha_2]^+,
\end{eqnarray}
from $(3.18)$, by $(3.15)$ and a Gr\"{o}nwall type argument, one
arrives $(3.10)$
 \vskip2mm\noindent
$\bullet$ \textbf{Assertion 3:} With locally uniform convergence
topology, $\{u_\varepsilon\}$ is pre-compact in
$\cC([0,\infty);L^1(\mR^d_x\times\mR_v))$ and $\{\rho_\varepsilon\}$
is pre-compact in $\cC([0,\infty);L^1(\mR^{d}))$.
 \vskip2mm\par
From $(3.10)$ (with a slight change), we have for any
$(\tilde{x},\tilde{v})\in \mR^d_x\times\mR_v$, $t\in (0,\infty)$,
\begin{eqnarray*}
&&\|u_\varepsilon(t,\tilde{x}+\cdot,\tilde{v}+\cdot)-
u_\varepsilon(t,\cdot,\cdot)\|_{L^1(\mR^d_x\times \mR_v))} \cr\cr
&\leq& \frac{1}{\varepsilon}\int^t_0e^{\frac{s-t}
{\varepsilon}}\|u_\varepsilon(s,\tilde{x}+\cdot,\tilde{v}+\cdot)-
u_\varepsilon(s,\cdot,\cdot)\|_{L^1(\mR^d_x\times \mR_v)}
\exp(\int^t_s\|[\partial_vA]^+\|_{L^\infty(\mR)}(r)dr)ds \cr \cr &&
+ e^{-\frac{t}{\varepsilon}}\int_{\mR^d_x \times \mR_v}
|\chi_{\rho_0(x+\tilde{x})}(v+\tilde{v})-\chi_{\rho_0(x)}(v)|dxdv
\exp(\int^t_0\|[\partial_vA]^+\|_{L^\infty(\mR)}(r)dr).
\end{eqnarray*}
Thus
\begin{eqnarray*}
&&\|u_\varepsilon(t,\tilde{x}+\cdot,\tilde{v}+\cdot)-
u_\varepsilon(t,\cdot,\cdot)\|_{L^1(\mR^d_x\times \mR_v)}
\cr\cr&\leq&\int_{\mR^d_x \times \mR_v}
|\chi_{\rho_0(x+\tilde{x})}(v+\tilde{v})-\chi_{\rho_0(x)}(v)|dxdv\exp(\int^t_0
\|[\partial_vA]^+\|_{L^\infty(\mR)}(s)ds).
\end{eqnarray*}
With the aid of $(3.19)$, then for $\tilde{v}=0$, it follows that
\begin{eqnarray*}
&&\|\rho_\varepsilon(t,\tilde{x}+\cdot)-
\rho_\varepsilon(t,\cdot)\|_{L^1(\mR^d)}\cr\cr
&=&\int_{\mR^d_x}{\Big|} \int_{\mR_v}
u_\varepsilon(t,\tilde{x}+x,v)dv-\int_\mR u_\varepsilon(t,x,v)dv
{\Big|}dx \cr \cr &\leq& \int_{\mR^d_x\times\mR_v}
|u_\varepsilon(t,\tilde{x}+x,v)- u_\varepsilon(t,x,v)|dxdv \cr \cr
&\leq& \int_{\mR^d_x\times\mR_v}
|\chi_{\rho_0(x+\tilde{x})}(v)-\chi_{\rho_0(x)}(v)|dxdv\exp(\int^t_0\|[\partial_vA]^+\|_{L^\infty(\mR)}(r)dr),
\end{eqnarray*}
which implies for any $0<T<\infty$, $\{u_\varepsilon\}$ is contained
in a compact set of $\cC([0,T];L^1_{loc}(\mR^d_x\times\mR_v))$,
$\{\rho_\varepsilon\}$ is pre-compact in
$\cC([0,T];L^1_{loc}(\mR^{d}))$. Hence by appealing to the
Arzela-Ascoli theorem, with any sequence $\{\varepsilon_k\}$,
$\varepsilon_k\rightarrow 0$ as $k\rightarrow \infty$, is associated
two subsequences (for ease of notation, we also denote them by
themselves) $\{u_{\varepsilon_k}\}$ and $\{\rho_{\varepsilon_k}\}$,
such that
$$
u_{\varepsilon_k}\longrightarrow u\in {\mathcal
C}([0,T];L^1_{loc}(\mR^d_x\times\mR_v)), \ \
\rho_{\varepsilon_k}\longrightarrow \rho\in {\mathcal
C}([0,T];L^1_{loc}(\mR^d)), \ \ \mbox{as} \ \ k\rightarrow \infty.
$$
 \vskip1mm\par
On the other hand, by $(3.7)$ and the lower semi-continuity,
\begin{eqnarray*}
&&u \in L^\infty_{loc}([0,\infty);L^\infty(\mR^d_x\times\mR_v))\cap
{\mathcal C}([0,\infty);L^1(\mR^d_x\times\mR_v)), \cr\cr &&
 \rho \in
L^\infty_{loc}([0,\infty);L^\infty(\mR^d))\cap {\mathcal
C}([0,\infty);L^1(\mR^d)). \end{eqnarray*}
  \vskip2mm\noindent
$\bullet$ \textbf{Assertion 4:} $\frac{1}{\varepsilon} {\big[}
\chi_{\rho_\varepsilon}-u_\varepsilon{\big]}=\partial_v
m_\varepsilon$, where $m_\varepsilon\geq 0$ is continuous in $t$ and
bounded uniformly in $\varepsilon$.
 \vskip1mm\par
Let $(t,x)\in (0,\infty)\times \mR^d$ be fixed, assuming without
loss of generality that $\rho_\varepsilon\geq0$, define
\begin{eqnarray*}
 m_\varepsilon(t,x,v)= \frac{1}{\varepsilon}
\int_{-\infty}^v{\big[}
\chi_{\rho_\varepsilon(t,x)}(r)-u_\varepsilon(t,x,r){\big]}dr.
\end{eqnarray*}
In view of $(3.5)$,
$$
u_{\varepsilon}(t,x,r)\in \cases{ \ \ [0,1], \ \ \mbox{when} \ r>0,
\cr  \ [-1,0], \ \mbox{when} \ r<0.}
$$
Hence $m_\varepsilon(t,x,v)$ is nondecreasing on
$(-\infty,\rho_\varepsilon)$ and nonincreasing on
$[\rho_\varepsilon,\infty)$. On the other side,
$m_\varepsilon(t,x,-\infty)=m_\varepsilon(t,x,\infty)=0$, we
conclude $m_\varepsilon\geq 0$.
 \vskip1mm\par
Since $\rho_0\in L^\infty(\mR^d)\cap L^1(\mR^d)$, owing to $(3.4),
(3.5)$, $(3.12)$ and condition $A\in
L^1_{loc}([0,\infty);W^{1,\infty}(\mR))$,
$$
\mbox{supp}m_\varepsilon\subset [0,T]\times \mR^d_x\times [-K,K],
$$
where $K=\|\rho\|_{L^\infty((0,T)\times
\mR^d)}\exp(\int^T_0\|\partial_vA(t)\|_{L^\infty(\mR)}ds)$.
 \vskip1mm\par
For above fixed $T>0$,
\begin{eqnarray*}
&& \int^T_0dt\int_{\mR^d}dx\int_{
\mR}m_\varepsilon(t,x,v)dv\cr\cr&=&\int^T_0dt\int_{\mR^d}dx
\int_{-K}^Kdv\int_{-K}^v[\partial_tu_\varepsilon+b(r)\cdot\nabla_x
u_\varepsilon+A(t,r)
\partial_r u_\varepsilon]dr \cr \cr &=&
\int^T_0dt\int_{\mR^d}dx\int_{-K}^Kdv\int_{-K}^v[
\partial_tu_\varepsilon(t,x,r)+A(t,r)
\partial_r u_\varepsilon]dr \cr \cr &\leq&
2K[\|u_\varepsilon(T)\|_{L^1(
\mR^d_x\times\mR_v)}+\|u_\varepsilon(0)\|_{L^1(\mR^d_x\times\mR_v)}]+
\int^T_0dt\int_{\mR^d}dx \int_{-K}^KA(t,v)u_\varepsilon(t,x,v) dv
\cr \cr && -
\int^T_0dt\int_{\mR^d}dx\int_{-K}^Kdv\int_{-K}^v\partial_r A(t,r)
u_\varepsilon(t,x,r) dr.
\end{eqnarray*}
 \vskip1mm\noindent
Combing $(3.12)$, we arrive at
\begin{eqnarray*}
&&\int^T_0dt\int_{\mR^d}dx\int_{
\mR}m_\varepsilon(t,x,v)dv\cr\cr&\leq & 4K^2
\|\rho_0\|_{L^1(\mR^d)}+(1+2K)\int^T_0dt\int_{\mR^d}dx
\int_{-K}^K\|\partial_vA(t)\|_{W^{1,\infty}(\mR)}
|u_\varepsilon(t,x,v)| dv.
\end{eqnarray*}
Whence $m_\varepsilon$ is bounded uniformly in $\varepsilon$.
  \vskip1mm\par
By extracting a unlabeled subsequence, one achieves
$$
m_\varepsilon\rightarrow m \geq 0 \ \mbox{in} \ \ {\mathcal
D}^\prime ([0,\infty)\times \mR^d_x\times\mR_v).
$$
In order to show $m$ yields the properties stated in Lemma 1.1, it
suffices to test that it is continuous in $t$, and by a translation,
it remains to demonstrate the continuity at zero. But this fact is
obvious, so the required result is completed.
   \vskip2mm\noindent
$\bullet$ \textbf{Assertion 5:} $u(t,x,v)=\chi_{\rho(t,x)}(v)$ and
$\rho$ solves $(1.1), (1.2)$. In addition, for any
$\rho_0,\tilde{\rho_0}\in L^\infty(\mR^d)\cap L^1(\mR^d)$, the
related solutions $u$ and $\tilde{u}$ of $(3.1)$ fulfill
\begin{eqnarray}\label{3.20}
\|[u(t)-\tilde{u}(t)]^+\|_{L^1(\mR^d_x\times \mR_v)}&\leq&
\exp(\int^t_0\|[\partial_vA]^+\|_{L^\infty(\mR)}(s)ds)\|[\chi_{\rho_0}-\chi_{\tilde{\rho}_0}]^+\|_{L^1(\mR^d_x\times
\mR_v)}\cr\cr&=&\exp(\int^t_0\|[\partial_vA]^+\|_{L^\infty(\mR)}(s)ds)
\|[\rho_0-\tilde{\rho}_0]^+\|_{L^1(\mR^d)},
\end{eqnarray}
\begin{eqnarray}\label{3.21}
\|\rho(t)-\tilde{\rho}(t)\|_{L^1(\mR^d}\leq
\exp(\int^t_0\|[\partial_vA]^+\|_{L^\infty(\mR)}(s)ds)
\|\rho_0-\tilde{\rho}_0\|_{L^1(\mR^d)},
\end{eqnarray}
\begin{eqnarray}\label{3.22}
\|\rho(t)\|_{L^\infty(\mR^d)}\leq
\exp(\int^t_0\|[\partial_vA]^+\|_{L^\infty(\mR)}(s)ds)
\|\rho_0\|_{L^\infty(\mR^d)}.
\end{eqnarray}
Furthermore, if $\rho_0\leq \tilde{\rho}_0$, for almost all
$(t,x,v)\in (0,\infty)\times\mR^d_x\times \mR_v$, and almost all
$(t,x)\in (0,\infty)\times \mR^d$,
\begin{eqnarray}\label{3.23}
&& u(t,x,v)\leq\tilde{u}(t,x,v), \quad \rho(t,x)\leq
\tilde{\rho}(t,x) .
\end{eqnarray}
In particular, if
$\rho_0\geq 0$, then $u\geq 0$, $\rho\geq 0$.
 \vskip1mm\par
Observing that $u_\varepsilon\rightarrow u$,
$\rho_\varepsilon\rightarrow \rho$ and $m_\varepsilon\rightarrow m$,
so $u_\varepsilon(t,x,v)-\chi_{\rho_\varepsilon}(v)\rightarrow 0$
and then $u=\chi_{\rho(t,x)}(v)$. Moreover $\rho$ is a weak solution
of $(3.1)$.
  \vskip1mm\par
With the help of $(3.10)-(3.13)$, the rest of the assertion is
clear.
 \vskip2mm\noindent
\textbf{Step 2:} Existence of stochastic weak solutions to $(1.7)$.
 \vskip1mm\par
Before handling the general $\sigma$, we review some notions. For
any $a\in\mR^d$, set $\tau_a$ by
$$
\tau_a\varphi(x)=\varphi(x+a), \ \ \mbox{for any} \ \varphi\in
\cC(\mR^d),
$$
and the pullback mapping of $m$ by $\tau^{*}_a$ is defined by
$$
\tau^{*}_am(\tilde{\phi})=m(\tau_{-a}\tilde{\phi})=\int_0^\infty
dt\int_{\mR^d}dx\int_{\mR}\tilde{\phi}(t,x-a,v)dv,
$$
for any $\tilde{\phi}\in {\mathcal D}([0,\infty)\times \mR^d_x
\times \mR_v).$
  \vskip2mm\par
Let us consider the below Cauchy problem
\begin{eqnarray}\label{3.24}
\left\{
  \begin{array}{ll}
\partial_t\tilde{u}(t,x,v)+b(v)\cdot\nabla_x\tilde{u}+
A(t,v)\partial_v\tilde{u}(t,x,v) =\tau^{*}_{M(t)}\partial_vm, \ \
\mbox{in} \ \ (0,\infty) \times \mR^d_x\times\mR_v, \\
 \tilde{u}(t,x,v)|_{t=0}=\chi_{\rho_0(x)}(v), \ \ \mbox{in} \ \
 \mR^d_x\times\mR_v.
  \end{array}
\right.
\end{eqnarray}
 \vskip1mm\par
The arguments employed in $(3.1)$ for $\partial_vm$ adapted to
$\tau^{*}_{M(t)}\partial_vm=\partial_v\tau^{*}_{M(t)}m$ in $(3.24)$
now, produces that there is a $\tilde{u}(\omega)\in
L^\infty_{loc}([0,\infty);L^\infty(\mR^d_x\times\mR_v))\cap
\cC([0,\infty);L^1(\mR^d_x\times\mR_v))$ solving $(3.24)$. Besides,
by Assertion 5, $\tilde{u}\in
L^\infty(\Omega;L^\infty_{loc}([0,\infty);L^\infty(
\mR^d_x\times\mR_v)))\cap
\cC([0,\infty);L^1(\mR^d_x\times\mR_v\times \Omega))$.
   \vskip1mm\par
Hence upon using It\^{o}-Wentzell's formula (see [8]) to
$$
G(y)=\int_{\mR^d_x\times\mR_v}\tilde{u}(t,x,v)\phi(x+y,v)dxdv,
$$
for any $\phi\in \cD(\mR^d_x\times\mR_v)$, one gains
\begin{eqnarray*}
&&\int_{\mR^d_x\times \mR_v}\tilde{u}(t,x,v)\phi(x+M_t,v)dxdv
-\int_{\mR^d_x\times \mR_v}\chi_{\rho_0(x)}(v)\phi(x,v)dxdv \cr
\cr&=& \int_{0}^{t}ds\int_{\mR^d_x\times \mR_v}\tilde{u}
b(v)\cdot\nabla_x
\phi(x+M_s,v)dxdv+\int_{0}^{t}ds\int_{\mR^d_x\times \mR_v}\tilde{u}
\partial_v[A(s,v)\phi(x+M_s,v)]dxdv \cr \cr &&  +
\int_0^tM_i(\circ ds)\int_{\mR^d_x\times \mR_v}
\tilde{u}(s,x,v)\partial_{x_i}\phi(x+M_s,v)dxdv -
\int_0^t\int_{\mR^d_x\times \mR_v}\partial_v\phi(x,v) m(ds,dx,dv).
\end{eqnarray*}
Let $u(t,x,v)=\tilde{u}(t,x-M_t,v)$, then $\tilde{u} \in
L^\infty(\Omega;L^\infty_{loc}([0,\infty);L^\infty(
\mR^d_x\times\mR_v)))\cap
\cC([0,\infty);L^1(\mR^d_x\times\mR_v\times \Omega))$, and
\begin{eqnarray}\label{3.25}
&&\int_{\mR^d_x\times \mR_v}u(t,x,v)\phi(x,v)dxdv
-\int_{\mR^d_x\times \mR_v}\chi_{\rho_0(x)}(v)\phi(x,v)dxdv \cr
\cr&=& \int_{0}^{t}ds\int_{\mR^d_x\times \mR_v}u(s,x,v)
b(v)\cdot\nabla_x
\phi(x,v)dxdv+\int_{0}^{t}\!\!ds\!\!\int_{\mR^d_x\times
\mR_v}u(s,x,v)
\partial_v[A(s,v)\phi(x,v)]dxdv \cr \cr && +\int_0^t
M_i(\circ ds)\int_{\mR^d_x\times
\mR_v}u(s,x,v)\partial_{x_i}\phi(x,v)dxdv-
\int_0^t\int_{\mR^d_x\times \mR_v}\partial_v\phi(x,v) m(ds,dx,dv).
\end{eqnarray}
Thanks to $(3.25)$ and Remark 3.1, hence there exists a stochastic
weak solution to $(1.7)$.
   \vskip2mm\noindent
\textbf{Step 3:} Existence of stochastic entropy solutions to
$(1.1), (1.2)$.
   \vskip1mm\par
Due to Step 2, one claims that
$$
u(t,x,v)=\chi_{\rho(t,x)}(v) \ \mbox{and}  \ \rho\in
L^\infty(\Omega; L^\infty_{loc}([0,\infty);L^\infty(\mR^d)))\cap
\cC([0,\infty);L^1(\mR^d\times \Omega)).
$$
Lemma 1.1 (ii) applies, $\rho$ is a stochastic entropy solution of
$(1.1), (1.2)$.
 \vskip2mm\par
With the help of Theorem 3.1 [7], and Assertion 5, we have:
\begin{corollary}
Let $\rho_0$, $B$, $\sigma$ and $A$ meet the conditions mentioned in
Theorem 3.1, then there is a unique stochastic entropy solution of
$(1.1), (1.2)$. Furthermore, if $\rho_0\geq 0$, then the unique
stochastic entropy solution $\rho \geq 0$.
\end{corollary}
\begin{remark}
(i)  Our proof for Theorem 3.1 is inspired by the proof for Theorem
1 in [9] and Lemma 2.1 in [10]. For more details, one can see [9-10]
and the references cited up there.
 \vskip1mm\par
(ii) When $A(t,\rho)=\xi(t)\rho(t,x)$, then an analogue calculation
of $(3.21), (3.22)$ also yields that
\begin{eqnarray*}
\|\rho(t)\|_{L^\iota(\mR^d)}\leq
\exp(\int^t_0\xi(s)ds)\|\rho_0\|_{L^\iota(\mR^d)},
 \quad \mbox{for} \ \ t\in [0,\infty) \ \ \mbox{and} \ \ \iota=1, \infty.
\end{eqnarray*}
Whence for any $p\in [1,\infty]$,
\begin{eqnarray}\label{3.26}
\|\rho(t)\|_{L^p(\mR^d)}\leq C\exp(\int^t_0\xi(s)ds).
\end{eqnarray}
If there is a positive real number $c>0$ such that $\xi\leq -c$,
then with probability one, the unique stochastic entropy solution
$\rho$ is exponentially stable. If for some real number
$\alpha_1,r_1>0$, $\xi$ possesses the below form
\begin{eqnarray*}
\xi(t)=\left\{
 \begin{array}{ll} \ -\frac{\alpha_1}{t}, \ \
\mbox{when} \ \  t\in (r_1,\infty),
\\ \ \xi_1(t), \ \ \mbox{when} \ \  t\in [0,r_1], \end{array} \right.
\end{eqnarray*}
where $\xi_1\in L^1([0,r_1])$, then from $(3.26)$,
\begin{eqnarray*}
\|\rho(t)\|_{L^p(\mR^d)}\leq \frac{C}{t^{\alpha_1}},  \ \ t\in
[0,\infty),
\end{eqnarray*}
which implies  $\rho$ is asymptotically stable.
 \vskip1mm\par
(iii) $(1.7)$ can be solvable by the stochastic characteristic
method for general forcing $A=A(t,x,\rho)$ and flux $B=B(t,x,\rho)$,
when $b(t,x,v)=\partial_vB(t,x,v)$ and $A$ are Lipschitz. But in
Theorem 3.1, this method is not adapted, for one may establish the
fundamental existence results on weak solutions for a large scale on
$b$ and $A$. Then the renormalization technique applies, one may
obtain the existence of stochastic weak solutions for $(1.7)$ on
weaker assumptions on $b$ and $A$, especially for the case of
non-Lipschitz vector fields (for example one can see [11] and [12]
for the linear stochastic transport equation).
 And now if the weak solution is unique, it can be
given by: for any $\phi\in \cD(\mR^d_x\times\mR_v)$,
\begin{eqnarray*}
&&\int_{\mR^d_x\times \mR_v}u(t,x,v)\phi(x,v)dxdv \cr \cr
&=&\int_{\mR^d_x\times
\mR_v}\chi_{\rho_0(x)}(v)\phi(X_{0,t},V_{0,t})\exp(\int^t_0
\partial_v
A(r,X_{0,r}V_{0,r})dr)dxdv \cr \cr &&- \int_0^t\int_{\mR^d_x\times
\mR_v}\frac{\partial}{\partial
v}[\phi(X_{s,t},V_{s,t})\exp(\int^t_s\partial_v
A(r,X_{s,r},V_{s,r})dr)] m(ds,dx,dv),
\end{eqnarray*}
where
\begin{eqnarray*}
\left\{
  \begin{array}{ll}
 \frac{d}{dt}(X_{s,t}(x,v),V_{s,t}(v))^\top=(b(t,X_{s,t},V_{s,t})+\dot{M}(t),
A(t,X_{s,t},V_{s,t}))^\top, \ t\geq s, \\ \\ (X_{s,t}(x,v),
V_{s,t}(v))^\top|_{t=s}=(X(s,x,v),V(s,x,v))^\top,
  \end{array}
\right.
\end{eqnarray*}
and $M(t)=(M_1(t),M_2(t),_{\cdots},M_d(t))$.
   \vskip1mm\par
(iv) All above restrictions on $B$ and $A$ seem to be rigid, but
there are many models in statistic physics and fluid mechanics,
satisfying all the hypotheses, we intend to illustrate it by
displaying a typical example now.
  \end{remark}
\begin{example} This example is concerned with the
Buckley-Leverett equation (see [13]), which provides a simple model
for the rectilinear flow of immiscible fluids (phases) through a
porous medium. To be simple, nevertheless, to capture some of the
qualitative features, here we consider the case of two-phase flows
(such as oil and water) in one space dimension. In this issue, the
Buckley-Leverett equation, with an external force, and a stochastic
perturbation reads
\begin{eqnarray}\label{3.27}
\left\{
 \begin{array}{ll}
\partial_t\rho(t,x)+\partial_xB(\rho)+\partial_x\rho(t,x)\circ\dot{M}(t)
=\mu A(t,\rho), \ \ \mbox{in} \ \ \Omega\times(0,\infty) \times \mR, \\
 \rho(t,x)|_{t=0}=\rho_0(x), \ \  \mbox{in} \ \ \mR,
  \end{array}
\right.
\end{eqnarray}
where
\begin{eqnarray*}
B(\rho)=\left\{
 \begin{array}{ll} \ \ \ \ \ 0, \ \ \ \ \ \  \ \mbox{when} \ \ \rho<0,
\\ \frac{\rho^2}{\rho^2+(1-\rho)^2}, \ \ \mbox{when} \ \
0\leq\rho\leq 1, \\  \ \ \ \  \ \  1,\ \ \ \ \ \  \mbox{when} \ \
\rho>1, \end{array} \right.
 \ \ M(t)=\int^t_0\vartheta(t)dW(s),\ \ \
A(t,\rho)=\frac{\theta(t)\rho^2}{1+\rho^2},
\end{eqnarray*}
$\mu\geq 0$ is a constant, $W$ is a 1-dimensional standard Wiener
process, $\theta\in L^1_{loc}([0,\infty))$. Then $B\in \cC^1$ and
$A\in L^1_{loc}([0,\infty);W^{1,\infty}(\mR)), A(0)=0$.
\end{example}
      \par
Applying the Corollary 3.1, we obtain
\begin{corollary} Assume that $\vartheta\in
L^2_{loc}([0,\infty))$, $\rho_0\in L^\infty(\mR)\cap L^1(\mR)$. Then
there exists a unique stochastic entropy solution $\rho$ of
$(3.27)$. Moreover, if $\rho_0\geq 0$, then $\rho\geq 0$.
\end{corollary}

\begin{thebibliography}{00}
 \bibitem{1} S.N. Kru\u{z}kov, First-order quasilinear equations in several
independent variables, Sbornik: Mathematics 10 (1970) 217-243.

 \bibitem{2} P.L. Lions, B. Perthame, P.E. Souganidis, Scalar conservation laws
with rough (stochastic) fluxes, Stochastic partial differential
equations: analysis and computations, 1(4) (2013) 664-686.

 \bibitem{3} G.Q. Chen, Q. Ding, K.H. Karlsen, On nonlinear stochastic balance
laws, Archive for Rational Mechanics and Analysis, 204(3) (2012)
707-743.

\bibitem{4}  J. Feng, D. Nualart, Stochastic scalar conservation laws, Journal
of Functional Analysis, 255(2) (2008) 313-373.

\bibitem{5} A. Debussche, J. Vovelle, Scalar conservation laws with stochastic
forcing, Journal of Functional Analysis,  259(4) (2010) 1014-1042.

\bibitem{6} M. Hofmanov\'{a}, A bhatnagar-gross-krook approximation to
stochastic scalar conservation laws, arXiv:1305.6450, 2013.

\bibitem{7} J.L. Wei, B. Liu,  Uniqueness of stochastic entropy solutions for stochastic balance
laws with Lipschitz fluxes, arXiv:1405.2614, 2014.

\bibitem{8} N.V. Krylov, On the It\^{o}-Wentzell formula for distribution-valued
processes and related topics, Probability Theory and Related Fields,
50 (2011) 295-319.

\bibitem{9} B. Perthame, E. Tadmor, A kinetic equation with kinectic entropy
functions for scalar conservation laws, Communications in
Mathematical Physics, 136 (1991) 501-517.

\bibitem{10} P. Catuogno, C. Olivera, $L^p$-solutions of the stochastic transport
equation,  Random Operators and Stochastic Equations, 21(2) (2013)
125-134.

\bibitem{11} J.L. Wei, B. Liu, $L^p$-solutions of Fokker-Planck equations,
Nonlinear Analysis 85 (2013) 110-124.

\bibitem{12} F. Flandoli, M. Gubinelli, E. Priola,  Well-posedness of the transport
equation by stochastic perturbation, Inventiones mathematicae,
180(1) (2010) 1-53.

\bibitem{13} C.M. Dafermas, Hyperbolic conservation laws in continuum physics,
Grundlehren der mathematischen Wissenschaften 325,  Springer-Verlag,
New York, 2010.
\end{thebibliography}
\end{document}
