\Para{Felder von Extremalen}{}
Sei im Folgenden $\Phi=\{f_s:M\longrightarrow \widetilde M\}$ ein Feld zu 
$f_0$.
\Def{}{}
Wir definieren das Differential von  $\Phi$ an der Stelle $q\in\widetilde M$ 
durch 
$$P(q):=\partial f_s\bigl|_p\hbox{ \quad\quad falls }f_s(p)=q.$$
\Def{}{}
Wir definieren ein neues Funktional
$$F^*(p,g(p),A\bigl|_p)
     :=F_{q_{i,\alpha}}(p,g(p),P(g(p)))
          \cdot(A_{i,\alpha}\bigl|_p-P(g(p)))
       + F(p,g(p),P(g(p)))$$
und 
$$\F^*(g):=\int_M{F^*(p,g(p),\partial g\bigl|_p)dvol}$$
\Lemma{}{}
{\it $\partial\F^*(g,\phi)=0$ f�r alle $\phi\in\C^\infty_0(M,\N M)$.}
\Beweis
Offensichtlich gilt $F^*_{p^\alpha_i}=F_{p^\alpha_i}$.
Die Bl�tter $f_s\in\Phi$ sind Extremale zu $F$, sie sind schwache
L�sungen der Eulergleichungen von $\F$, das hei�t
$$\eqalign{0=
  &F_{y^\alpha}(p,f_s(p),\partial f_s\bigl|_p)
   -{\partial\over\partial x_i}F_{p^\alpha_i}(p,f(p),\partial f_s\bigl|_p)\cr
 =&F_{y^\alpha}(p,f_s(p),P(f(p)))
   -{\partial\over\partial x_i}F_{p^\alpha_i}(p,f(p),P(f(p)))             \cr
 =&F_{y^\alpha}(p,f_s(p),P(f(p)))
   -F_{p^\alpha_ix_i}(p,f(p),P(f(p)))                                     \cr
  &-F_{p^\alpha_iy_\beta}(p,f(p),P(f(p)))
    \cdot{\partial f_a\over\partial x_i}                                  \cr
  &-F_{p^\alpha_ip^\beta_j}(p,f(p),P(f(p)))                               
    \cdot(P^\beta_{x_i}-P^j_{y_\alpha}P^i)\;.                             \cr
            }$$

\Lemma{}{\INVARIANT}
{$\F^*(f)=\F^*(g)$ f�r alle Immersionen $f$ und $g$ mit 
$\partial f(M)=\partial g(M)$ und $f(M),g(M)\subset \O$.}
\Bem{}{}
Wegen der Eigenschaft {\INVARIANT} hei�t $\F^*$ {\bf Hilbert invariantes
Integral}.                    
\Def{}{}
Wir definieren die {\bf Weierstra�-Funktion} zu $F$
$$\E:M\times\widetilde M\times \Hom(\T M,\T \widetilde M)
         \times\Hom(\T M,\T \widetilde M)\longrightarrow \RR$$
durch
$$\E(p,z,A,B):=F(x,z,B)-F(p,z,A)-(B-A)F_P(p,z,A).$$
\Satz{}{\EXTREFELD}
{\it $f:M\longrightarrow \widetilde M$ sei kompakte orientierte 
immersierte Extremale zu $\F$ und 
$\Phi$ ein $\C^2$-Feld zu $M$ in $\widetilde M$, so da� $\widetilde M$
eine Tubenumgebung ${\cal T}_\epsilon(f)$ von $M$ enth�lt.\par
Ist dann weiterhin $\E(p,z,A,B)>0$ f�r alle $p\in M$, $z\in {\cal T}_\epsilon(f)$
und $A\ne B\in \Hom(\T M,\T \widetilde M)$, 
so ist $f$ starkes homologisches Minimum von $\F$ zu 
eigenen Randwerten.}
\Beweis
Wir betrachten eine andere Immersion $g:M\longrightarrow \widetilde M$ mit
$\partial f(M)=\partial g(M)$.  
$$\eqalign{\F(g)-\F(f)=&\F(g)-\F^*(f)=\F(g)-\F^*(g)                       \cr
    \int_M&{F(p,g(p),\partial g\bigl|_p)
            -F(p,g(p),\partial f\bigl|_{f^{-1}(g(p))})}                   \cr
          &-F_p(p,g(p),\partial f\bigl|_{f^{-1}(g(p))})
            \cdot(\partial g\bigl|_p-\partial f\bigl|_{f^{-1}(g(p))})dvol \cr
          =\int_M&{\underbrace{
            \E(p,g(p),\partial f\bigl|_{f^{-1}(g(p))},\partial g\bigl|_p)
                            }_{\geq 0}}   \;.                             \cr
           }$$
Damit ist dann $\F(g)\geq\F(f)$.\kasten
