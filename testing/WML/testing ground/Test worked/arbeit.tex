   \magnification=\magstep1
   \hsize=15true cm\vsize=22true cm
   \tolerance=1000
   \frenchspacing
   \input mathmac.def
\drafttrue
   \kapno =0\nummer =0
   \scrollmode
\input abstract \vfill\eject

  \openin 0 = {inhaver}
  \ifeof 0\else
\centerline{\ganzgross Inhaltsverzeichnis}
\bigskip
\input inhaver \vfill\eject
  \fi
  \closein 0
  \openin 0 =mathmac.aux
  \ifeof 0\else
\input mathmac.aux
  \fi
  \closein 0

  \newwrite\inhaltno
  \immediate\openout\inhaltno= inhaver.tex
\input einfuehr
\Kap{Notation und Grundlagen}{\BLA}
\input diffgeo1
\input diffgeo2
\input diffop
\input fktanal
\Kap{Variationsformeln und Jacobi-Operator}{\BAL}
\input varimin
\input eulerneu
\input jacobiop
\input kleigenw\vfill\eject
\Kap{Variationsformeln f�r den Fl�cheninhalt}{}
\input flaevari\vfill\eject
\Kap{Kalibrierungen, Felder und starke Minimalit�t}{}
\input minimum
\input kalifeld
\input hilbert
\input strikmin\vfill\eject
\Kap{Feldkonstruktion}{}
\input hjet
\input fixpunkt
\input feld\vfill\eject
\Kap{Absch�tzungen f�r $E_s$}{}
\input eabsch\vfill\eject
\Kap{Simultane Feldkonstruktion}{}
\input feldzwei\vfill\eject
\Kap{Literaturliste}{}\bigskip
\input litlis
   \closeout\makrono
   \closeout\inhaltno
   \vfill
   \end


%%% Local Variables: 
%%% mode: plain-tex
%%% TeX-master: t
%%% End: 
