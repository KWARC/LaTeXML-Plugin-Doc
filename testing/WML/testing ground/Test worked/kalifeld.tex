\Para{Felder von Minimalfl�chen}{}
\Def {}{}
Sei $\widetilde M$ eine Riemannsche Mannigfaltigkeit, 
$M$ eine $n$-dimen\-sio\-na\-le 
orientierte Untermannigfaltigkeit.
Eine geschlossene $n$-Form $\phi$ auf $\widetilde M$ hei�t eine {\bf Ka\-li\-brie\-rung 
von $M$ in $\widetilde M$} wenn f�r alle $p\in \widetilde M$ und f�r alle orthonormalen 
$v_1,\ldots,v_n \in \T _p\widetilde M$ gilt:
$$\phi_p (v_1,\ldots,v_n) \leq 1$$
und 
$$\phi_p(x_1,\ldots,x_n)=1$$
falls $(x_1,\ldots,x_n)$ positiv orientiert. 
\Satz {Fundamentalsatz f�r Kalibrierungen}{\KALIFUN}
{\it Sei $\widetilde M$ eine vollst�ndige Riemannsche Mannigfaltig\-keit, 
$M$ eine kompakte, orientierbare, n-dimensionale 
Unter\-man\-nig\-fal\-tig\-keit von $\widetilde M$
und $\phi $ ein Ka\-li\-brie\-rung von $M$ in $\widetilde M$. 
Dann ist $M$ homologisch minimierend in $\widetilde M$ zu eigenen Randwerten,
das hei�t, f�r alle zu $M$ homologen Untermannig\-fal\-tig\-kei\-ten $N$
 mit $\partial M = \partial N$ gilt ${\cal H}^n(N) \geq {\cal H}^n(M)$.}
\Beweis
(Federers Differentialformenargument)
$${\cal H}^n(N) =    \int _N {1\, d{\cal H}^n} 
                \geq \int _N {\phi}
                =    \int _M {\phi} 
                =    \int _M {1\, d{\cal H}^n}
                =    {\cal H}^n(M)$$\kasten
\Bem {}{}
Hat $\widetilde M$ triviale $n$-te Homologie, so ist $M$ minimierend zu eigenem 
Rand.
\Bem {}{\RANDMIN} 
Ist f�r alle Untermannigfaltigkeiten
$N\subset \widetilde M$ die orientierte Vereinigung $N\cup(-M)$ Rand eines 
Kompaktums, so ist $M$ ein Minimum in $\widetilde M$ zu
eigenen Randwerten.
\Satz {Solomon}{\SOLOMON}
{\it Sei $U\subset \RR^{n+1}$ offen. Dann ist das Einheitsnormalenfeld
$\nu$ an eine $\C^0$-Bl�tterung von $U$ in minimale Hyperfl�chen sogar
Lipschitz-stetig.}
\Kor {}{\FELDDIFF}
{\it Ist $\Phi$ ein Feld zu $M$ in $\widetilde M$, 
so ist das Einheitsnormalenfeld
an $\Phi$ fast �berall differenzierbar.}
\Beweis 
mit dem Satz von Rademacher {\RADEMACHER}.\kasten
\Satz {}{\MINIFELD}
{\it Ist $M$ eine kompakte, orientierbare Minimalfl�che mit Rand 
$\partial M\ne \emptyset$, und 
ist $\Phi$ ein $\C^0$-Feld zu $M$ in $\widetilde M$, so da� $\widetilde M$ eine 
Tubenumgebung von $M$ enth�lt,
so ist $M$ ein starkes homologisches
 Minimum des Fl�\-chen\-in\-halts.}
\Beweis
Wir konstruieren eine Kalibrierung von $M$ in  $\widetilde M$.
\par Sei $\nu$ das Einheitsnormalenfeld an $\Phi$ und seien
in einer Umgebung von $p\in \widetilde M$ die orthonormalen Vektorfelder
$\{ e_0,\ldots ,e_n \}$ gegeben. Es gibt dann Koeffizientenfunktionen 
$\alpha_i$, so da� $\nu = \sum _{i=0}^n{\alpha_ie_i}$ auf $V$.
Mit 
$$\omega \colon = de_0 \wedge\ldots\wedge de_n $$
ist dann f�r jede positiv orientierte Orthonormalbasis $ \{v_0,\ldots,v_n\} $
von $\T _pS$ ist $$\omega (v_0,\ldots,v_n) = 1. $$
Definiere nun  auf $\widetilde M$ die $n$-Form $\omega _\nu$ durch
$$\eqalign{\omega_\nu (w_1,\ldots ,w_n)\colon = 
                & \omega (\nu,w_1,\ldots,w_n)                            \cr
             =  & \sum^n_{i=0}{\alpha_i\omega (e_i,w_1,\ldots,w_n)}      \cr
             =  & \sum^n_{i=0}{(-1)^i\alpha_i
                                   de_0 \wedge\ldots\wedge \widehat {de_i}
                                        \wedge\ldots\wedge de_n  
                                   (w_1,\ldots,w_n)} \;.                 \cr
     }$$
Also ist
$$\eqalign{\omega_\nu & = \sum^n_{i=0}{(-1)^i\alpha_i
                           de_0 \wedge\ldots\wedge \widehat {de_i}
                                \wedge\ldots\wedge de_n}       \cr
          d\omega_\nu & = \sum^n_{i=0}
                           {{\partial\alpha_i\over\partial x_i}
                            de_0\wedge\ldots\wedge de_n}           \cr
                      & = \div (\nu)\omega       \;.               \cr
                    }$$
Um zu zeigen, da� $\omega_\nu$ geschlossen ist, zeigen wir da� $\div(\nu)=0$.
In einer Umgebung um $q=f_s(p)$ kann man ohne Einschr�nkung annehmen, 
da� $f_s$ ein Graph in $\RR^{n+1}$ ist, das hei�t,
$f_s(x)=(x,F(x))$. Dann ist die Normale in $x$
$$\nu={1\over\sqrt{1+|\nabla F|^2}}(\nabla F,-1)\;.$$
Die Minimalfl�chengleichung f�r die minimale Immersion $f_s$ hat dann die
Form
$$\div_n\left({\nabla F\over\sqrt{1+|\nabla F|^2}}\right)=0\;.$$
Wir haben also
$$\div_{n+1}(\nu)=\div_n\left({\nabla F\over\sqrt{1+|\nabla F|^2}}\right)
                 +{\partial\over\partial x_{n+1}}
                   \left({-1\over\sqrt{1+|\nabla F|^2}}\right)=0\;,$$

denn der rechte Term h�ngt nicht von $x_{n+1}$ ab.\par
F�r eine positiv orientierte Orthonormalbasis $\{x_1,\ldots,x_n\} $ von 
$\T M$ ist $$\omega_\nu (x_1,\ldots,x_n)=1,$$
denn $\{\nu,x_1,\ldots,x_n\}$ ist eine positiv orientierte Orthonormalbasis 
von $\T _p\widetilde M$.
\par F�r alle orthonormalen $\{v_1,\ldots,v_n\}$ gibt es genau
 ein $v_0$, so da� $\{v_0,\ldots,v_n\}$ eine positiv orientierte 
Orthonormalbasis bildet.
Dann ist $\omega (v_0,\ldots,v_n)=1$.
\par $\nu$ l��t sich �berall so aufspalten, da� $\nu = v + \sigma v_1$ wobei 
$v\in \hbox{span}(v_0,\ldots,v_n)$ und $\sigma \leq 1$. Also gilt
$$\eqalign{ \omega_\nu (v_1,\ldots,v_n) 
                       & = \omega (\nu ,v_1,\ldots,v_n)           \cr
                       & = \omega (v,v_1,\ldots,v_n)
                         + \omega (\sigma v_0,v_1,\ldots,v_n)
                         = \sigma \leq 1\;.                       \cr
                }$$
$\omega_\nu$ kalibriert also $M$ in $\widetilde M$.
Die Behauptung folgt jetzt aus dem Fundamentalsatz f�r Kalibrierungen.
\kasten
\Bem {}{\MINIBEM}
Wenn sich eine Minimalfl�che $M\subset\RR^{n+1}$ als Graph schreiben l��t, 
ist sie ein starkes Minimum des Fl�cheninhalts zu eigenen Randwerten.
\Beweis
Sei $M_0=\{(x,f(x))\}$. Dann erf�llt 
$\Phi\colon =\{M_s \colon  s\in \RR\}$ mit $M_s = \{ (x,f(x)+s) \} $ die 
Voraussetzungen von Satz {\MINIFELD}.\kasten

