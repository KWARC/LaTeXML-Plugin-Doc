\Kap{Einf�hrung}{}
In dieser Arbeit betrachten wir das Variationsintegral
$$\F(f)=\int_M{\Phi(d_pf)\dvol}$$
mit einer Lagrangeschen
$$\Phi\colon \Hom(\T M,\T\widetilde M)\longrightarrow \RR$$
der Klasse $\C^2$. Dabei ist $f\colon M\longrightarrow\widetilde M$
eine $\C^1$-Abbildung von einer kompakten $n$-dimesnionalen 
Mannigfaltigkeit $M$ in eine glatte $n+1$-dimensionale Mannigfaltigkeit 
$\widetilde M$. $\F$ schreibt sich in lokalen  Koordinaten als ein 
Funktional
$$I(u)=\int_\Omega{F(x,u(x),Du(x))dx}$$
mit einer $\C^2$-Lagrangeschen
$$F\colon \Omega\times\RR^n\times\RR^N\longrightarrow \RR$$ 
und $N=n+1$
und verallgemeinert deswegen $I$.\par
Ein lokales Minimum von $\F$ in der Klasse der Funktionen mit gleichem 
Rand ist ein kritischer Punkt $f$ von $\F$ und deswegen eine L�sung der 
Euler-Lagrange-Gleichungen 
$$L(f)=0\;.$$
Wir geben eine hinreichende Bedingung daf�r, da�
eine L�sung $f$ der Euler-Lagrange-Gleichungen ein 
Minimum von $\F$  ist: \par
Ist der kleinste Eigenwert $\lambda$ der zweiten Variation von $\F$ an der 
Stelle $f$ positiv, so ist $f$ ein starkes homologisches Minimum von $\F$ 
zu eigenen Randwerten. Ist dagegen $\lambda<0$, so ist $f$ kein lokales 
Minimum von $\F$.\par
F�r den Fall von nichtparametrisch gegebenen Extremalen, das hei�t, f�r
Graphen von Funktionen $f\colon \Omega\longrightarrow\RR$  
stammt das hier angegebene Resultat im wesentlichen von L. Lichtenstein
[LL], der f�r $\dim \Omega=2$ Funktionale mit
analytischem Integranden $F$ betrachtet hat. Die �bertragung auf den Fall
mehrdimensionaler Variationsprobleme stammt von C. B. Morrey [M]; einige 
Vereinfachungen wurden von S. Hildebrandt [HS1],[HS2] und X. LI [LX]
angegeben. Die Verallgemeinerung auf den Fall von  
kompakten immersierten Hyperfl�chen der Kodimension 1 in glatten 
Riemannschen Mannigfaltigkeiten wurde von mir ausgef�hrt.\par
Um starke homologische Minimalit�t zu zeigen, zeigen wir die Existenz 
eines Feldes, das hei�t einer ausreichend glatten Bl�tterung aus Extremalen.
Dieses Feld konstruieren wir im allgemeinen Fall mit einer Methode, wie sie
S. Hildebrandt in [HS2] f�r den Fall nichtparametrischer Extremaler 
benutzt. Den Fall von immersierten Minimalfl�chen im $\RR^{n+1}$ 
behandeln wir gesondert.
Wir konstruieren das Feld, indem wir die Schar der Parallelfl�chen durch 
geeignete kleine St�rungen in ein Feld von Minimalfl�chen 
Feld von Minimalfl�chen `verbiegen'. N. Smale hat in [SN] eine 
Minimalfl�che erhalten indem er eine `Fast-Minimalfl�che' in eine 
Minimalfl�che `verbogen' hat.\par
Ich danke $\ldots$ blabla.




