\Satz{Erste Variationsformel}{\ERSTEVAR}
{\it Sei $\phi $ ein normales Variationsfeld, dann ist
$$\partial I(f,\phi)=-\int_M{<H,\phi>dV(f)}.$$}
\Beweis
$$\partial I(f,\phi)={d\over dt}I(\exp_ft\phi)
={d\over dt}\int_M{dV(\exp_ft\phi)}$$
man mu� also zeigen, da�
$${d\over dt}dV(\exp_ft\phi)=-<K,\phi>dV(f).$$
Wir w�hlen jetzt einen ON Rahmen $\{v_i\}$ auf einer Umgebung von $p$ mit
$\nabla_{v_i}v_i=0$. Dies kann man zum Beispiel erreichen, indem man eine
orthonormale Basis von $\T _pM$ parallel entlang der von $p$ ausgehenden geod�tischen
Strahlen verschiebt. Setzt man nun $f_t\colon=\exp_ft\phi$, dann erhlten wir f�r
die induzierte Metrik
$g_{ij}(t)\colon=<\partial f_t\cdot v_i,\partial f_t\cdot v_j>$ und
$g(t)\colon=\det(g_{ij}(t))$, und danmit
$$dV(f_t)=\sqrt{g(t)}dx_1\wedge\ldots\wedge dx_n=\sqrt{g(t)}dV(f)$$
$$\eqalign{{d\over dt}dV(f_t)&={d\over dt}\sqrt{g(t)}dV(f)          
                              ={1\over 2}{dg\over dt}dV(f)          \cr
                             &={1\over 2}\Spur\bigl(g_{ij}(t)\bigr) 
                              ={1\over 2}\sum_{i=1}^n{d\over dt}g_{ii}(t)\cr
    }$$
Setzt man nun $v_i(t)\colon=\partial f_t\cdot v_i$ dann erh�t man
$$\eqalign{{d\over dt}g_{ii}\Bigl|_{t=0} 
    =&{d\over dt}<v_i(t),v_i(t)>\Bigl|_{t=0}                         \cr
    =& 2<\widetilde\nabla_\phi v_i(t),v_i(t)>\Bigl|_{t=0}            \cr
    =& 2<\widetilde\nabla_{v_i(t)}\phi,v_i(t)>\Bigl|_{t=0} 
       +2<\underbrace{[\phi,v_i(t)]}_{=0},v_i>                       \cr
    =& 2v_i(t)\underbrace{<\phi,v_i>}_{=0}
       -2<\phi,\widetilde\nabla_{v_i(t)}v_i(t)>                      \cr
    =& -2<(\widetilde\nabla_{v_i}v_i)^N,\phi>
       -<\underbrace{(\widetilde\nabla_{v_i}v_i)^T}_{=0},\phi>       \cr
       }$$
Insgesamt ist also ${d\over dt}dV(f_t)=-<H,\phi>$. \kasten
\Bem{}{}
$H$ ist der Euleroperator zu $I$.
\Lemma{}{}
{\it Ein kritischer Punkt $f$ von $I$ erf�llt die lineare elliptische 
Differtialgleichung
$$ H(f)=0$$}
F�r der zweite Variation des Fl�cheninhalts betrachten wir die folgenden
Differentialoperatoren
$$\eqalign{{\cal K}(\phi)   \colon=&\sum_{i=1}^n{\widetilde R(v_i,\phi)v_i} \cr
           \triangle^N(\phi)\colon=& \Spur\nabla^N\nabla^N                  \cr
      }$$
wobei $v_i$ ein lokaler orthonormaler Rahmen von $\T M$ sei.\par

\Lemma{}{}
{\it ${\cal K}$ und $\triangle^N$ sind lineare selbstadjungierte 
Differentialoperatoren zweiter Ordnung.}
\Beweis
Symmetrien\par
Die zweite Fundamentalform $\B$ ist ein Schnitt in 
$\T ^*M\otimes \T ^*M\otimes \N M$, deswegen ist 
$$\B_p\in \T _p^*M\otimes \T _p^*M\otimes \N M 
      \cong \Hom(\T _p^*M\otimes \T _p^*M,\N M). $$
Wir betrachten den Differentialoperator 
$$B\colon=\B^t\circ\B\;,$$
dabei ist $\B^t$ der zu $\B$ adjungierte Homomorphismus 
$$\B_p^t\in\Hom(\N _pM,\T _p^*M\otimes \T _p^*M). $$
\Lemma{}{}
{\it $B$ ist linearer selbstadjungierter Differentialoperator nullter 
Ordnung.}
\Beweis
$$\eqalign{<B(\nu),\mu>
    &=<\B^t(\nu),\B^t(\mu)>                                               \cr
    &=\sum^n_{i,j=1}{<\B^t(\nu),e_i\otimes e_j><\B^t(\nu),e_i\otimes e_j>}\cr
    &=\sum^n_{i,j=1}{<\B(e_i,e_j),\nu><\B(e_i,e_j),\mu>}  \;,             \cr
         }$$
insbesondere ist $B$ selbstadjungiert.
\Satz{Zweite Variationsformel}{\ZWEITEVAR}
{\it Ist $f\colon M\longrightarrow \widetilde M$ eine minimale Immersion,
$\phi$ ein normales Vektorfeld l�ngs $f$,
so gibt einen linearen selbstadjungierten Differentialoperator der zweiter 
Ordnung, so da�
$$\partial^2I(f,\phi)=
  \int_M{<-\triangle^N\phi+{\cal K}\phi-B\phi,\phi)>}.$$}
\Beweis
F�r die mittlere Kr�mmung $H(t)$ der variierten Immersionen 
$f_t=\exp_ft\phi$ gilt mit den Bezeichnungen von {\ERSTEVAR}
$$\eqalign{H(t)&=\Spur(\B(t)                                          \cr
               &=\sum_{i,j=1}^n{g^{ij}(t)\B(v_i(t),v_j(t))}           \cr
               &=\sum_{i,j=1}^n{g^{ij}(t)
                      \bigl(\widetilde\nabla_{v_i(t)}v_j(t)\bigr)^N}  \cr
         }\;.$$
Damit ergibt sich
$$\eqalign{{d\over dt}<H(t),\phi>\bigl|_{t=0}
    = &\sum_{i,j=1}^n{{dg^{ij}\over dt}(0)
      <\bigl(\widetilde\nabla_{v_i(t)}v_i(t)\bigr)^N,\phi>\bigl|_{t=0}}   \cr
    = &\sum_{i,j=1}^n{{dg^{ij}\over dt}(0)<\widetilde\nabla_{v_i}v_i,\phi>}
      +\sum_{ij=1}^n{g^{ij}(0){\partial \over \partial t}
      <\widetilde\nabla_{v_i}v_i,\phi>}                                  \cr
    = &\sum_{i,j=1}^n{{dg^{ij}\over dt}(0)<\widetilde\nabla_{v_i}v_i,\phi>}
      +\sum_{i=1}^n{<\widetilde\nabla_\phi\widetilde\nabla_{v_i}v_i,\phi>}\cr
   }\;.$$
Wir berechnen nun die beiden Summanden,
und verwenden dabei die Formeln
$${d\over dt}(g^{ij}(t)g_{ij}(t))=0$$ 
$$\widetilde\nabla_\phi v_i-\widetilde\nabla_{v_i}\phi=[v_i,\phi]=0$$ 
$$<\widetilde\nabla_{v_i}\phi,v_i>+<\phi,\widetilde\nabla_{v_i}v_i>
       =v_i<v_i,\phi>=0\;.$$
Es ist also
$$\eqalign{{dg^{ij}\over dt}(0) 
            = & -{\partial\over\partial t}<v_i,v_j>                 \cr
            = & -<\widetilde\nabla_\phi v_i,v_j>
                -<v_i,\widetilde\nabla_\phi v_j>                    \cr
            = & <\widetilde\nabla_{v_i}v_j,\phi> 
               +<\widetilde\nabla_{v_j}v_i,\phi>                    \cr
            = & 2<\widetilde\nabla_{v_i}v_j,\phi>
               +\underbrace{<[v_i,v_j],\phi>}_{\quad =0} \;.        \cr
     }$$
\def\nab{\widetilde\nabla}
F�r den linken Summanden berechnen wir
$$\eqalign{ <\nab_\phi\nab_{v_i},\phi>                                 
           =&<\nab_{v_i}\nab_\phi v_i,\phi>
             +<\widetilde R(\phi,v_i)v_i,\phi>
             +\underbrace{<\nab_{[v_i,\phi]}v_i,\phi>}_{\quad =0}   \cr
           =&v_i<\nab_\phi v_i,\phi>
             -<\nab_\phi v_i,\nab_{v_i}\phi>
             -<\widetilde R(v_i,\phi)v_i,\phi>        \;.           \cr
     }$$
Dabei ist
$$\eqalign{<\nab_{v_i}\phi,\nab_\phi v_i>
             =&<\nab_{v_i}\phi,\nab_{v_i}\phi>
             =\bigl|\nab_{v_i}\phi\bigr|^2                          \cr
             =&\bigl|(\nab_{v_i}\phi)^N\bigr|^2            
             + \bigl|(\nab_{v_i}\phi)^T\bigr|^2                     \cr
             =&\bigl|\nabla^N_{v_i}\phi\bigr|^2
             + \sum_{j=1}^n{<\nab_{v_i}\phi,v_j>^2}                 \cr
             =&\bigl|\nabla^N_{v_i}\phi\bigr|^2
             + \sum_{j=1}^n{<\nab_{v_i}v_j,\phi>^2}                 \cr
             =&\bigl|\nabla^N_{v_i}\phi\bigr|^2
             + \sum_{j=1}^n{<(\nab_{v_i}v_j)^N,\phi>^2} \;.         \cr
     }$$
Insgesamt ist also, wenn man alle Rechnungen einsetzt
$$\eqalign{&{d\over dt}<K(t),\phi>\bigr|_{t=0}                        \cr
         = &\sum_{i,j=1}^n{<\B(v_i,v_j),\phi><\phi,\B(v_i,v_j)>}      \cr
           &+\quad\sum_{i=1}^n{\bigl|\nabla^N_{v_i}\phi\bigl|^2}\quad
            -\quad\sum_{i=1}^n{<\widetilde R(v_i,\phi)v_i,\phi>}  \;. \cr
        }$$
Aus $\int_M{\bigl|\nabla^N \phi\bigr|^2}=-\int_M{<\triangle^N\phi,\phi>}$ 
folgt nun die Behauptung.\kasten
\Para{Der Jacobi-Operator}{}
\Lemma{}{}
{\it $$L_f\colon=-\triangle^N+{\cal K}-B$$
ist ein stetiger linearer selbstadjungierter Differentialoperator zweiter 
Ordnung und es gilt
$$L_f(\phi)={d\over dt}H(\exp_ft\phi)\Bigr |_{t=0}.$$}
\Beweis
$$\eqalign{\int_M{<L_f(\phi),\phi>} 
           = & \partial^2I(f,\phi)                                       \cr
           = & {d\over dt}{\partial I(\exp_ft\phi,\phi)\bigl |}_{t=0}    \cr
           = & - { \int_M{{d\over dt}<H(\exp_ft\phi),\phi>}\Bigl|}_{t=0} \cr
           = & - \int_M{<\widetilde\nabla_{F_*{\partial\over \partial t}}
               H(\exp_ft\phi),\phi>}\Bigl |_{t=0}                        \cr
             & - \int_M{<H(\exp_ft\phi),
               \widetilde\nabla_{F_*{\partial\over\partial t}}\phi> 
               \Bigl |} _{t=0}                                           \cr
           = & -\int_M{<{d\over dt}H(\exp_ft\phi)\bigl | _{t=0},\phi>}\;.\cr
       }$$
Mit dem Fundamentallemma der Variationsrechnung folgt jetzt die Behauptung.
\kasten

