Der Titel der Diplomarbeit wird voraussichtlich sein
"Eine hinreichend Bedingung f�r die starke minimalit�t von 
kompaktenMinimalfl�chen".Eine Minimalfl�che $M$, das hei�t eine 
$n$-dimensionale Hyperfl�che mit Rand, die die notwendige Bedingung, 
da� die die mittlereKr�mmung $H$ auf $M$ identisch verschwindet, ist 
im allgemeinen keinMinimum des Fl�cheninhales in der Kalasse aller 
Funktionen, die den gleichen Rand haben wie $M$. In dieser Arbeit 
aknn ich zeigen, da� $M$ ein starkes Minimum des Fl�cheninhaltes ist, 
falls der Jacobioperator von $M$ nur echt positive Eigenwerte hat. 
Ich f�hre dabei die starke minimalit�t von $M$ auf die Existenz eine 
Kalibrierung auf einer offenen Tubenumgebung von $M$ und dann auf die 
Existenz einer Bl�tterung dieser Tubenumgebung in Minimalfl�chen 
zur�ck. Diese Bl�tterung wird dann im zweiten Teil der Arbeit unter 
der oben gennanten Voraussetzung an den Jacobioperator konstruiert. 
Ich greife in dieser Arbeit eine Methode von Nathan Smale auf, mit 
der Minimalfl�chen in $\RR^{n+1}$ durch kleine St�rungen von Fl�chen 
mit 'kleiner' mittlerer Kr�mmung konstruiert werden, und 
verallgemeinere sie auf den Fall von Hyperfl�chen in glatten 
Mannigfaltigkeiten. Das Problem $H\equiv 0$ auf $M$ wind 
umgeschrieben in ein Fixpunktproblem der Form $T(x)=x$, dann werden 
durch Absch�tzungen f�r die Operatoren $H$ und $T$ auf gewissen 
kompakten Mengen ${\cal K}(\sigma,\alpha)$ die Voraussetzungen des 
Banachschen Fixpunktsatzes nachgewiesen. F�r die Verallgemeinerung 
auf Untermannigfaltigkeiten von glatten Mannigfaltigkeiten mu�te ich 
eine globale Darstellung der mittleren Kr�mmung finden und f�r diese 
�hnliche Absch�tzungen erhalten, die dann die Konvergenz des 
Banachschen Fixpunktsatzes f�hren. 

Zur Konstruktion einer Bl�tterung von Minimalfl�chen wird dann die 
Schar der Parallelfl�chen von $M$ in eine Schar von Minimalfl�chen 
'verbogen'. Aus dem Maximumsprinzip f�r lineare partielle 
Diffenzialoperatoren ergeben sich dann die Eigenschaften f�r die 
Bl�tterung.

Ein gro�er Reiz an dieser Arbeit war, da� ish hier Methoden und 
ergebnisse aus mehreren verschieden Gebieten der Mathematik zu einer 
Art globalen Variationsrechnungr verbinden konnte.

