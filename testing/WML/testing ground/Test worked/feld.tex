\Para{${\cal F}$ ist eine $\C^0$-Bl�tterung}{\BLAETTERUNG}
\Satz{}{}
{\it Die Bl�tter von ${\cal F}$ schneiden sich nicht.}
\Beweis
Angenommen, es gibt ein $p\in \MM_s\cap\MM_t$ mit $s<t$, dann ist 
$u_s(p)-u_t(p)=0$ und deswegen $\max(u_s-u_t)\geq 0$.
Auf dem Rand von $M$ ist aber $u_s-u_t=s-t<0$, damit ergibt sich ein 
Widerspruch zum Maximumprinzip {\MAXPRINZ}. \kasten
\medskip
\par Sei $V\colon = \{x\in \O \colon  x\not\in \MM _\sigma \quad
\forall\quad |\sigma |< s_1 \} $.
\Lemma {}{\UELIMES}
{\it Der Rand jeder Zusammenhangskomponente von $V$ ist von der Form
$\MM _s \cup \widehat \MM _s$. Dabei ist $\widehat \MM _s $ Minimalfl�che,
 und es gilt $\partial \widehat \MM _s = \partial \MM _s = \partial M_s$.}
\Beweis
Sei $p\in V$ so, da� $p = \rho \nu (x)$ und ohne Einschr�nkung $\rho >0$.
$s\nu + u_s(x)$ ist streng monoton in $s$, weil $F$ keine Schnitte hat. 
Deswegen existiert entweder $\min \{ s \in \RR \colon   s\nu + u_s(x)>\rho \}$ oder 
$\max \{ s \in \RR \colon  s\nu + u_s(x) < \rho \}$.
\par {\bf Fall 1:} $\sigma \colon = \min \{ s \in \RR \colon  s\nu + u_s(x) > \rho \} $ 
  existiert.
\par $ \MM_\sigma$  ist also Teil des Randes der Zusammenhangskomponente.
\par Sei jetzt $\widehat u _\sigma (x) \colon = \sup_{s<\sigma} u_s(x)$ und 
$(s_n)_{n\in \NN}$ mit $s_n \nearrow \sigma$, dann konvergiert $u_{s_n}$
punktweise gegen $u_\sigma$. Aber es gilt 
$|u_{s_n}|_{\C^2} \leq C |u_{s_n}|_{\C^0} \leq C |\widehat u_\sigma |_{\C^0}$, 
weil $\MM_{s_n}$ Minimalfl�chen sind. 
Die Folge $u_{s_n}$ ist also auch in $\C_0^2(\MM,\N \MM)$ 
beschr�nkt und konvergiert dort gegen $\widehat u_\sigma$.
\par $H^\bot$  ist auf $\C_0^2(\MM,\N \MM)$ stetig und deswegen ist 
$H^\bot (u_\sigma)=0$. $\widehat \MM_\sigma \colon = M_{\widehat u_\sigma}$ ist also 
Minimalfl�che.
\par Offensichtlich ist $\partial \MM _\sigma = \partial \widehat \MM_\sigma$.
Deswegen ist $\MM_\sigma \cup \widehat \MM_\sigma$ der Rand der 
Zusammenhangskomponente von $V$, in der $p$ liegt.
\par {\bf Fall 2 } analog.\kasten
\Lemma{}{}
{\it $V=\emptyset$}
\Beweis
$H(\widehat u_\sigma)=H(u_\sigma)=0$ und deswegen ist
f�r $v\colon =(\widehat u_\sigma - u\sigma)$ auch $H(v)=0$ und 
$v\bigl |_{\partial M}=0$. Nach dem Maximumprinzip {\MAXPRINZ} ist dann
$v\leq 0$ auf $M$. Mit demselben Argument erh�lt man auch $-v\leq 0$ 
auf $M$. Aus $\widehat u_\sigma=u_\sigma$ erh�lt man nun die Behauptung. 
\kasten
Die Abbildung 
$$\eqalign{v\colon M\times[-\rho,\rho]
            &\longrightarrow {\cal O}\subset \widetilde M     \cr
   (x,s)    &\longmapsto     \tau\circ v_s(x)                 \cr
    }$$
ist ein Hom�omorphismus. $M\times[-\rho,\rho]$ ist kompakt und 
$\rho>0$, deswegen ist 
$\inf_{(x,s)\in M\times[-\rho,\rho]}|v_s(x)|=\colon \epsilon>0$. 
${\cal O}$ enth�lt also die Tubenumgebung ${\cal T}_\epsilon (f)$.
Die Niveaufl�chen $\{v=\const\}=M_{v_s}=\colon \MM_s$ von $v$ sind nach der 
Konstruktion immersierte Minimalfl�chen.
\Lemma{}{}
{\it ${\cal F}\colon =\{\MM_s\colon s\in [-\rho,\rho]\}$ ist $\C^0$-Bl�tterung}
\Beweis
Es ist nur noch $4)$ in {\BLAETTERDEF} nachzuweisen. Ist $x\in{\cal O}$,
$U\subset \N M$ eine Umgebung von $v^{-1}(x)$ und 
$\phi\colon U\to V\times[-\rho,\rho]$ eine lokale Trivialisierung des
Normalenb�ndels, so ist
$$h\colon =v\circ\phi^{-1}\colon V\times[-\rho,\rho]\to v(U)$$ 
ein Hom�omorphismus, und es gilt f�r alle $s\in [-\rho,\rho]$
da� $h(V\times\{s\})=v_s(\pi(U))\subset \MM_s$.\kasten
\Satz{}{}
{\it Ist $M$ kompakte, orientierbare $n$-dimensionale Mannigfaltigkeit und
$f\colon M\longrightarrow \widetilde M$ isometrische, minimale, stabile Immersion,
so gibt es ein Feld ${\cal F}$ zu $M$ in $\widetilde M$.}
\Kor {Hauptsatz}{} 
{\it Ist $f\colon M\longrightarrow \widetilde M$ wie oben, so ist $M$ ein 
starkes homologisches Minimum des Fl�cheninhalts in der Klasse der Fl�chen
mit gleichem Rand. }

