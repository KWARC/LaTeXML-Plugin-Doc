\newif\ifshort\shorttrue
\documentclass{llncs}
\usepackage{local}
\usepackage{verbatim}
\usepackage[show]{ed}
\usepackage{url}
%\usepackage[eso-foot,today]{svninfo}
\usepackage{svninfo}
\usepackage{lstom}
%\usepackage[final]{svninfo}
\svnInfo $Id: OM2009.tex 592 2009-07-14 21:34:38Z jhd $
\svnKeyword $HeadURL: https://svn.kwarc.info/repos/kohlhase/tex/papers/mkm/mmlstrict/OM2009.tex $

\bibliographystyle{alpha}

\def\forallin{\forall_{\kern-.25em i}}
\def\forallcond{\forall_{\kern-.25em c}}
\def\forallincond{\forall_{\kern-.25em ic}}
\def\existsin{\exists_{\kern-.01em i}}
\def\existscond{\exists_{\kern-.01em c}}
\def\existsincond{\exists_{\kern-.01em ic}}
 
\title{Quantifiers and Big Operators in OpenMath}
\author{James H.~Davenport\inst{1} \and Michael Kohlhase\inst{2}}
\institute{Department of Computer Science\\
University of Bath, Bath BA2 7AY, United Kingdom\\
{\texttt{J.H.Davenport@bath.ac.uk}}
\and
 School of Engineering \& Science, Jacobs University Bremen\\
Campus Ring 12,
 D-28759 Bremen, Germany\\
{\texttt{m.kohlhase@jacobs-university.de}}} 
\begin{document}
\maketitle
\begin{abstract}\noindent
  The effort to align MathML 3 and OpenMath has led to a realisation that (pragmatic)
  MathML's {\element{condition}} and {\element{domainofapplication}} elements, when used
  with quantifiers, do not have a {\emph{neat}} expression in OpenMath.

  This paper analyzes the situation focusing on quantifiers and proposes a solution, via
  six new symbols.
Two of them fit completely within the existing OpenMath structure, and we 
place them in the associated {\symbol{quant3}} CD. The others require a
generalization of {\symbol{OMBIND}}.
\par We also
  propose, logically separately but in the same area, a {\symbol{quant2}} CD with
  {\symbol{existsuniquely}}, commonly written $\exists!$, and the `fusion'
symbol {\symbol{existsuniquelyin}}.

  In a second step we generalize the solution to the phenomenon of big operators that
  MathML 2 implicitly provides but which do not have a direct counterpart in the OpenMath
  CDs.
\end{abstract}

\section{Introduction}
The effort to align MathML 3 \cite{WorldWideWebConsortium2008a} and OpenMath
\cite{OpenMath2004a} has led to a realisation that (pragmatic) MathML 3's
{\element{condition}} and {\element{domainofapplication}} elements, when used with
quantifiers, does not have a {\emph{neat}} expression in OpenMath. We have described the
MathML 3/OpenMath 3 and the general representational issues
in~\cite{DavenportKohlhase2009a}, which we will assume as background reference. But a
central part of the alignment effort remains unsolved there: the provisioning of the
OpenMath content dictionaries that supply the necessary symbols. We will describe what
needs to be done on this front in this paper, which should be seen as a companion paper
to~\cite{DavenportKohlhase2009a}.

We start out by analyzing the situation focusing on quantifiers and then generalizing the
solution to the phenomenon of big operators that MathML 2 implicitly provides but which do
not have a direct counterpart in the OpenMath CDs.

\section{Existential and Universal Quantifiers}

``$\forall n\in\N\ldots$'' is a very natural piece of mathematical notation, even though it
tends not to be formally defined. MathML (Content) can represent this via the
{\element{condition}} element, but OpenMath\footnote{If restricted only to the content
  dictionaries the authors know about.} has hitherto had no {\emph{direct}} way of doing
so, relying on the following equivalences:
\begin{eqnarray}
\label{eq:forall}
\forall  v\in S. p(v)&\Leftrightarrow \forall v.(v\in S)\Rightarrow p(v)\\
\label{eq:exists}
\exists  v\in S. p(v)&\Leftrightarrow \exists v.(v\in S)\land p(v)
\end{eqnarray}
However, in practice\footnote{A referee objected to \cite{DavenportKohlhase2009a} on the
  grounds that it had stated that such shorthand was not necessary, writing `one might [as well]
  write ``not $p$'' as``$p$ implies false'''.}, it would be better to have a convenient
shorthand for these, hence this proposal.

The challenge is the syntax of OpenMath. One represents $\forall n.p(n)$ as
\begin{lstlisting}
<OMBIND>
  <OMS name="forall" cd="quant1"/>
  <OMBVAR><OMV name="n"/></OMBVAR>
  <OMA><OMV name="p"/><OMV name="n"/></OMA> 
</OMBIND>
\end{lstlisting}

Where is the ``$\in\N$'' going to fit in this syntax? In \cite{DavenportKohlhase2009a} we
have argued that there are actually two representation styles that need to be supported:
Figure~\ref{fig:ex1} gives both styles the {\emph{first-order}} style on the left makes
use of expressions with bound variables whereas the {\emph{higher-order}} style directly
uses sets.
\begin{figure}[ht]
\lstset{frame=none,numbers=none,aboveskip=-.7em,belowskip=-1.2em,mathescape}
\begin{tabular}{|p{5.9cm}|p{5.9cm}|}\hline
\begin{lstlisting}
<OMBIND>
  <OMS name="forallcond" cd="quant3"/>
  <OMBVAR>
    <OMV name="n"/>
  </OMBVAR>
  <OMA>
    <OMS name="in" cd="set1"/>
    <OMV name="n"/>
    <OMS name="N" cd="setname1"/>
  </OMA>
  $\psom{p(n)}$
</OMBIND>
\end{lstlisting}
&
\begin{lstlisting}
<OMBIND>
  <OMA>
    <OMS name="forallin" cd="quant3"/>
    <OMS name="N" cd="setname1"/>
  </OMA> 
  <OMBVAR>
    <OMV name="n"/>
  </OMBVAR>
  $\psom{p(n)}$
</OMBIND>
\end{lstlisting}
\\\hline
\multicolumn{2}{|p{11.8cm}|}{here and in the following we use boxed mathematical notation to
  represent the obvious {\openmath} counterparts}\\\hline
\end{tabular}
\caption{Two representations of quantifiers with domains}\label{fig:ex1}
\end{figure}
The latter is nearer to our example, and we propose to use the fact that the first child
of an {\element{OMBIND}} need not be an atom and propose a symbol {\symbol{forallin}}
which acts as a binding constructor, i.e. an operator that takes a set as an argument and
returns a binding operator, which can then be used in the {\element{OMBIND}}. The
first-order style of representation could be glossed in Mathematics as $\forallcond
n:n\in\N.p(n)$; it makes the bound variable that was implicit in the higher-order
construction explicit in a condition expression. So we would contend that the higher-order
style is closer to our original example. But the first-order style is more in other
situations, e.g. $\forall x:(1/x\;\mbox{is irrational}).p(x)$, which can be represented as
$\psom{\forall x:(\forall n,d.n/d\ne x).p(x)}$.

\begin{lstlisting}[caption=A natural case for a quantifier with a condition,label=fig:irrational,mathescape]
<OMBIND>
  <OMS name="forallcond" cd="quant3"/>
  <OMBVAR><OMV name="x"/></OMBVAR>
  <OMBIND>
    <OMS name="forall" cd="quant1"/>
    <OMBVAR><OMV name="n"/><OMV name="d"/></OMBVAR>
    <OMA><OMS cd="relation1" name="ne"/>
      <OMA><OMS cd="arith1" name="divide">
        <OMV name="n"/>
        <OMV name="d"/>
      </OMA>
      <OMV name="x"/>
    </OMA>
  </OMBIND>
  $\psom{p(x)}$
</OMBIND>
\end{lstlisting}
Note that for the first-order representations on the left of Figure~\ref{fig:ex1} and in
Listing~\ref{fig:irrational} we need the extension of binding objects to allow multiple
scopes proposed in~\cite{DavenportKohlhase2009a}. We consider the examples in this paper
and representation possibilities they enable to be a good reason for this extension.

The symbols {\texttt{forallin}} and {\texttt{forallcond}} are defined in the proposed
{\cdname{quant3}} CD, with Formal Mathematical Properties corresponding to equations
(\ref{eq:forall}) and (\ref{eq:exists}), at least in the single-variable case. They are
related by the following FMPs:
\[\begin{array}{rlcl}
  \forall P,Q.&[\forallcond x.Q(x) P(x)] &\Leftrightarrow& [\forallin(\lambda{z}.P(z))]x.Q(x)\\
  \forall Q,S.&[\forallin(S)]x.Q(x) & \Leftrightarrow & \forall x. (x\in
S)\Rightarrow Q(x) 
\end{array}\]
where we use $\forallcond$ for
{\symbol{forallcond}} and $\forallin$ for $\symbol{forallin}$. We should note what {\symbol{forallin}} does, and does not, encode.
\begin{description}
\item[does]$\forall n\in\N$, $\forall m,n\in\N$, $\forall x\in[0,1]$, $\forall
  x\in(0,\infty)$ (but not the equivalent $\forall x>0$).
\item[does not]$\forall n\in\N,x \in\R$ (this needs two nested bindings of
  {\symbol{forallin}}\footnote{At first sight it seems that this could be represented as
    $\forallin{\N\times\R}.$ but we contend that these are different. We can encode
    $\forall z\in(\N\times\R)$, and {\emph{later}} destructure $z$, but 
{\openmath} doesn't have a
    destructuring bind.}), $\forall n>2$ (though we {\emph{can}} encode $\forall n \in
  (2,\infty)$), $\forall m<n\in\N$ (though we {\emph{can}} encode $\forall
  n\in(-\infty,n)$ and use {\symbol{integer\_interval}}).
\end{description}

\cite[section 4.2.5.1]{WorldWideWebConsortium2003b} gives an example for the formula
\[\forall{x\in\N}.\exists{p,q\in\P}.p+q=2x\]
where $\P$ stands for the set of prime numbers. While this would succumb to forall/implies
and exists/and encodings, it is a better tribute to the power of our extended quantifiers
to use the following encoding:
\begin{lstlisting}[mathescape]
<OMBIND>
  <OMA>
    <OMS name="forallin" cd="quant3"/>
    <OMS name="N" cd="setname1"/>
  </OMA>
  <OMBVAR> <OMV name="x"/> </OMBVAR>
  <OMBIND>
    <OMA>
      <OMS name="existsin" cd="quant3"/>
      <OMS name="P" cd="setname1"/>
   </OMA>
    <OMBVAR><OMV name="p"/><OMV name="q"/></OMBVAR>
    $\psom{p+q=2x}$
 </OMBIND>
</OMBIND>
\end{lstlisting}

Note that in some cases, we naturally combine these two: We often see something like the
following $\forall x,y\in\R:x-y\ne0.{\frac{1}{x-y}\in\R}$. This would be 
suitably encoded as
\begin{lstlisting}[mathescape]
<OMBIND>
  <OMA>
    <OMS name="forallincond" cd="quant3"/>
    $\psom{\R}$
  </OMA> 
  <OMBVAR><OMV name="x"/><OMV name="y"/></OMBVAR>
  $\psom{\frac{1}{x-y}\in\R}$
  $\psom{x-y\ne0}$
</OMBIND>
\end{lstlisting}
using the a symbol {\texttt{forallincond}} that we propose to add to {\texttt{quant3}}
content dictionary together with the {\texttt{FMP}}
\[\forallincond{(S)}x:{C(x)}.D \Leftrightarrow\forallin(S).C(x)\Rightarrow D\]
The existential variants {\texttt{existin}}, {\texttt{existscond}}, and
{\texttt{existsincond}} of all three have analogous FMPs in the {\texttt{quant3}} content
dictionary.

\section{Unique Existence}

Although the notation $\exists!$ is relatively new\footnote{It was not in use when the
  first author was a student, and the earliest we can trace it to is
  \cite[p. xiii]{Barendregt1984}.}, it is convenient. It would be easy enough to introduce
into OpenMath, as a symbol, say {\symbol{existsuniquely}} in the {\symbol{quant2}} CD,
with one property, which could be held to define it.
\begin{equation}\label{eq:eu}
\exists! x.p(x) \Leftrightarrow (\exists x.p(x))\land
\left((p(x)\land p(y))\Rightarrow x=y\right).
\end{equation}


One might naturally ask: ``what about $\exists!x\in \N:\ldots$, and similar
constructs''. There is obviously no difficulty (other than the length of the
name!) in adding {\symbol{existsuniquelyin}} to the {\symbol{quant3}}
CD. The real challenge is what semantics does one want. There are two options.
\begin{description}
\item[``It's unique {\it and\/} it's in $\N$'']
\begin{equation}\label{eq:eu-broad}
\exists! x\in\N.p(x) \Leftrightarrow (\exists x.(x\in\N\land p(x))\land 
\left((p(x)\land p(y))\Rightarrow x=y\right).
\end{equation}
This is what one would get by applying equation (\ref{eq:exists}) to the
$\exists$ on the right-hand side of equation (\ref{eq:eu}).
\item[``It's unique {\it within\/} $\N$'']
\begin{equation}\label{eq:eu-narrow}
\exists! x\in\N.p(x)\Leftrightarrow (\exists x.(x\in\N\land p(x)))\land
\left(((p(x)\land x\in\N)\land (p(y)\land y\in\N))\Rightarrow x=y\right).
\end{equation}
This is what one would get by applying equation (\ref{eq:eu}) to equation
(\ref{eq:exists}).
\end{description}
While both have their part to play, it appears that (\ref{eq:eu-narrow}) is
the one more commonly met, and we propose to add this meaning to
{\symbol{quant3}}.

\section{Other Similar cases: Big Operators}

Note that in the analysis above, the fact that we are dealing with quantifiers is
secondary, the interesting part is the fact that, the ``operators take qualifiers''
{\element{condition}} and {\element{domainofapplication}} in MathML 2. The full list of
these operators can be grouped into three parts: the special operators {\element{int}},
{\element{sum}}, {\element{product}}, {\element{root}}, {\element{diff}},
{\element{partialdiff}}, {\element{limit}}, {\element{log}}, {\element{moment}},
{\element{forall}}, {\element{exists}}, the binary/$n$ary operators {\element{plus}},
{\element{times}}, {\element{max}}, {\element{min}}, {\element{gcd}}, {\element{lcm}},
{\element{mean}}, {\element{sdev}}, {\element{variance}}, {\element{median}},
{\element{mode}}, {\element{and}}, {\element{or}}, {\element{xor}}, {\element{union}},
{\element{intersect}}, {\element{cartesianproduct}}, {\element{compose}}, as well as the
relational operators {\element{eq}}, {\element{leq}}, {\element{lt}}, {\element{geq}},
{\element{gt}}. The use of qualifiers with relational operators is being deprecated in
MathML 3, so we will not concern ourselves with these.

The reason binary operators can take qualifiers is that they can be ``lifted'' to their
respective ``big operator form'', for instance $\bigcup$ for $\cup$. Note that if we look
at the special operators in the first group, then we can see that {\element{sum}},
{\element{product}}, {\element{forall}}, and {\element{exists}} are the conventionalized
``big operators'' for {\element{plus}}, {\element{times}}, {\element{and}}, and
{\element{or}}. We have covered the quantifiers above, so let us look whether
{\element{sum}} and {\element{product}} show the same behavior. If they do, the
quantifiers may give a good model for the other ``big operators''. Here we can directly
see that all cases occur, witnessed e.g. by the Lagrange base polynomial $\displaystyle
L(x)\colon=\Pi_{i=0,i\ne j}^k\frac{x-x_i}{x_j-x_i}$, which would need a symbol
{\symbol{productincond}} in {\texttt{arith2}}.

\section{Other Operators that take Qualifiers}
There are other operators that take qualifiers in MathML; these are amenable to the same
treatment: Consider the naive set $\{x^2|x<1\}$, which could be represented in MathML 2 as
\begin{lstlisting}[language=MathML2]
<set>
  <bvar><ci>x</ci></bvar>
  <condition>
    <apply><lt/><ci>x</ci><cn>1</cn></apply>
  </condition>
  <apply><power/><ci>x</ci><cn>2</cn></apply>
</set>
\end{lstlisting}
We propose a {\texttt{suchthatcond}} binding constructor that would allow us to write
\begin{lstlisting}
<OMBIND>
  <OMS cd="set4" name="suchthatcond"/>
  <OMBVAR><OMV name="x"/></OMBVAR>
  <OMA>
    <OMS cd="relation1" name="lt"/>
    <OMV name="x"/>
    <OMI>1</OMI>
  </OMA>
  <OMA>
    <OMS cd="arith1" name="power"/>
    <OMV name="x"/>
    <OMI>2</OMI>
  </OMA>  
</OMBIND>
\end{lstlisting}

% \section{Remaining Special Operators}

% We have dealt with the ``big operators'' above, for {\element{int}}, {\element{diff}}, and
% {\element{partialdiff}}, see~\cite{DavenportKohlhase2009d}, so we only have to 
% \begin{todo}{think about this: JHD: What's the citation to?}
%   deal with the operators{\element{root}}, {\element{limit}}, {\element{log}},
%   {\element{moment}}.
% \end{todo}

\section{Conclusion}\label{sec:concl}

We have studied the representation of extended quantifiers like $\forall x\in S.$ or
$\forall x: p(x).$ commonly found in informal mathematical texts. While it is possible to
encode these in principle with the quantifiers and connectives provided by the
{\texttt{quant1}} and {\texttt{logic1}} content dictionaries from the MathML CD group. The
encoding loses the surface structure of the original mathematical expressions and does not
support the same intuitive modes of reasoning. Therefore we propose to augment the
{\openmath} society's set of standard CDs with a new content dictionary {\texttt{quant3}}
with six new symbols and {\texttt{FMP}}s that relate them to the classical quantifiers.

By the same concerns for structural adequacy we propose to introduce a symbol for unique
existence to the existing {\texttt{quant2}} content dictionary and various operators for
``lifted operators'' that MathML 2 implicitly supports by allowing them to ``take
qualifiers''.

We feel that the proposed symbols will make the use of {\openmath} more intuitive and thus
make {\openmath} more attractive as a whole for the working mathematician. At the same
time, people who prefer the classical quantifiers can refrain from using the new symbols.

It should be noted $\exists!$, $\existsin$ and $\forallin$ do {\emph{not}} require any
changes to {\openmath}: the rest require an extension of the concept of binding, or the
acceptance of mathematically meaningless `gluing' operators to allow for the fact that
bding should take place over both the body and the predicate.

Hence our proposals, in increasing order of scope, are as follows.
\begin{enumerate}
\item Adopt {\symbol{existsuniquelyin}}, i.e. $\exists!$.
\item Adopt {\symbol{existsin}}, i.e. $\existsin$, and{\symbol{forallin}},
  i.e. $\forallin$.
\item Accept that {\symbol{OMBIND}} should be able to bind over more than one
  child.\label{contro}
\item Adopt {\symbol{forallcond}} etc.
\item Adopt {\symbol{suchthatcond}}, and other related symbols.
\end{enumerate}
Should \ref{contro} prove a bridge too far, the subsequent proposals {\emph{could}} be
adopted with a mathematically meaningless gluing operator, as
in~\cite{DavenportKohlhase2009a}.
\bibliography{../../../../jhd}
\ifshort
\end{document}
\fi
\newpage
\begin{appendix}
  \section{The Proposed Quant2 Content Dictionary}
  \lstinputlisting[language=omCD]{quant2.ocd}
\newpage
  \section{The Proposed Quant3 Content Dictionary}
  \lstinputlisting[language=omCD]{quant3.ocd}
\end{appendix}
\end{document}

% LocalWords:  domainofapplication quant existsuniquely OMBIND forall cd OMBVAR
% LocalWords:  OMV OMA forallin forallcond setname FMPs rlcl forallincond arith
% LocalWords:  productincond bvar ci csymbol definitionURL eq cn existscond OMI
% LocalWords:  existsin existsuniquelyin lt suchthatcond jhd ne destructure FMP
% LocalWords:  destructuring existin existsincond partialdiff gcd lcm sdev leq
% LocalWords:  cartesianproduct geq
