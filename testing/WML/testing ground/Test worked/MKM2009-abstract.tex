% \fullfalse generates the paper for the journal, without tables, but with
% cross-references to the full paper
% \fulltrue generates the full paper
% Note the bibliography file uses this, to decide whether to generate an entry
% for the full paper (\fullfalse). This is the last entry, to prevent the
% numbers changing between versions
\newif\iffull\fulltrue
%\newif\iffull\fullfalse
\iffull
\else 
\fi
\def\myfig#1{(Figure \ref{#1}\iffull\else{} of \cite{Full}\fi)}
\def\mylst#1{(Listing \ref{#1}\iffull\else{} of \cite{Full}\fi)}
\def\I{{\cal I}}
\def\N{{\bf N}}
\def\Z{{\bf Z}}
\def\Q{{\bf Q}}
\def\C{{\bf C}}
\def\R{{\bf R}}
\def\CL{\mathop{\rm CL}}
\def\IDA{{}_{\rm DA}\int}
\def\DDA{D_{\rm DA}}
\def\cDA{constant${}_{\rm DA}$}
\def\fracDDA#1#2{\frac{{\rm d}#1}{{\rm d}_{\rm DA}#2}}
\def\Ied{{}_{\epsilon\delta}\int}
\def\Ded{D_{\epsilon\delta}}
\def\fracDed#1#2{\frac{{\rm d}#1}{{\rm d}_{\epsilon\delta}#2}}
%\documentclass{article}   % Without notes
\documentclass{llncs}
\usepackage{verbatim}
\usepackage[show]{ed}
\usepackage{url}
\usepackage{lstom}
\usepackage[eso-foot,today]{svninfo}
%\usepackage[final]{svninfo}
\svnInfo $Id: MKM2009-abstract.tex 494 2009-03-08 05:49:10Z kohlhase $
\svnKeyword $HeadURL: https://svn.kwarc.info/repos/kohlhase/tex/papers/mkm/mmlstrict/MKM2009-abstract.tex $
\bibliographystyle{alpha}

\newtheorem{Rule}{Rule}

% MK had "Meta" here: JHD thinks we're really talking about the math itself
\title{Unifying Math Ontologies: A tale of two standards}
\author{James H.~Davenport\inst{1} \and Michael Kohlhase\inst{2}}
\institute{Department of Computer Science\\
  University of Bath, Bath BA2 7AY, United Kingdom\\
  {\tt J.H.Davenport@bath.ac.uk}
  \and
  School of Engineering \& Science, Jacobs University Bremen\\
  Campus Ring 12,
  D-28759 Bremen, Germany\\
  {\tt m.kohlhase@jacobs-university.de}} 

\begin{document}
\maketitle
\begin{abstract}\noindent
  One of the fundamental and seemingly simple aims of mathematical knowledge management
  (MKM) is to develop and standardize formats that allow to ``represent the meaning of the
  objects of mathematics''. The open formats OpenMath and MathML address this from a
  content markup perspective, but subtly differ in syntax, rigor, and structural
  viewpoints (notably over calculus). To avoid fragmentation and smooth out interoperability obstacles effort is
  under way to align them into a joint format OpenMath/MathML3. We illustrate the
  conceptual and practical issues that come up in such an alignment by looking at three
  main areas: conditions, calculus (which also relates to the previous) and ``lifted'' $n$-ary
  operators.
\end{abstract}
\end{document}
\begin{quotation}\noindent\em
  Whenever anyone says ``you know what I mean'', you can be pretty sure that {\emph{he}}
  does not know what he means, for if he did, he would tell you.  \hfill{\rm{--- Anon.}}
\end{quotation}

\section{Introduction}

One of the fundamental and seemingly simple aims of mathematical knowledge management
(MKM) is to develop and standardize formats that allow to ``represent the meaning of the
objects of mathematics''.

 The open formats OpenMath and MathML address this from a content
markup perspective, but subtly differ in syntax, rigor, and structural
viewpoints (notably over calculus). To
avoid fragmentation and smooth out interoperability obstacles effort is under way to align
them into a joint format OpenMath/MathML3. We illustrate the conceptual and practical
issues that come up in such an alignment by looking at three main areas: conditions,
calculus (which relates to the previous) and ``lifted'' $n$-ary operators.

\begin{newpart}{we have to say something about the history probably, but maybe not here}
  The specification of version 1.01 of the MathML format was released in July 1999 and version
  2.0 appeared in February 2001. In October 2003, the second edition of MathML Version 2.0
  was published as the final release by the W3C math working group. In June 2006 the W3C
  has rechartered the MathML Working Group to produce a MathML 3 Recommendation until
  February 2008 and in November 2008 extended the charter to April 2010.

  OpenMath has been developed in a long series of workshops and (mostly European) research
  projects that began in 1993 and continues through today. The OpenMath 1.0 Standard was
  released in February 2000, and revised as OpenMath 1.1 in October 2002. Two years later,
  the OpenMath 2.0 Standard was released in June 2004. OpenMath 1 fixed the basic language
  architecture, while OpenMath2 brought better XML integration, structure sharing and
  liberalized the notion of OpenMath Content dictionaries.
\end{newpart}

MathML's Content markup has ambitious goals.
\begin{quotation}\noindent\em
  The intent of the content markup in the Mathematical Markup Language is to provide an
  explicit encoding of the {\it underlying mathematical structure\/} of an expression,
  rather than any particular rendering for the expression.  \hfill{\hbox{\rm\cite[section
      4.1.1]{WorldWideWebConsortium2003b}}}
\end{quotation}
On the face of it, OpenMath has very similar goals.
\begin{quotation}\noindent\em
  {\emph{OpenMath}} allows the {\emph{meaning}} of an object to be encoded rather than
  just a visual representation. It is designed to allow the free exchange of mathematical
  objects between software systems and human beings. On the worldwide web it is designed
  to allow mathematical expressions embedded in web pages to be manipulated and computed
  with in a meaningful and correct way. It is designed to be machine-generatable and
  machine-readable, rather than written by
  hand.\hfill\hbox{\rm\cite[Abstract]{OpenMath2004a}}
\end{quotation}
Given the apparent similarity, it is not surprising that attempts are being
made to merge them in MathML 3.0.
\begin{quotation}\noindent\em
  The text of this chapter is generated by filtered extraction from XML Content
  Dictionaries written in accordance with OpenMath. The details of the Content Dictionary
  format have been further specified and the generation procedure improved. It is expected
  that the Content Dictionaries will become a separate joint publication of the W3C and
  OpenMath referenced by the MathML3 specification.
  \hfill\hbox{\rm\cite[Status]{WorldWideWebConsortium2008a}}
\end{quotation}
Nevertheless, this task proves to be harder than might initially be expected.
In this paper, we explain why, motivated by a study of three areas (which in
fact turn out to be inter-related): 
\begin{enumerate}
\item the \verb+<condition>+ element of MathML;
\item the different handling of calculus-related operations in the two;
\item the ``lifting'' of $n$-ary operators, such as $+$ to $\sum$.
\end{enumerate}
\section{MathML}
MathML, starting from version 1.0, had a split into ``presentation'', describing what
mathematics ``looked like''\footnote{Which could include ``sounded like'' (for aural
  rendering) or ``felt like'' (e.g. for Braille), and MathML included a range of symbols
  such as {\texttt{\&InvisibleTimes;}} to help with this task.}, and ``content'',
describing what it ``meant''. MathML version 1 had a fairly limited vision of what might
be in ``content''.
\begin{quotation}\noindent\em
  % It would be an enormous job to systematically codify most of mathematics --- a task
  % which can never be complete. Instead, MathML makes explicit a relatively small number
  % of commonplace mathematical constructs, chosen carefully to be sufficient in a large
  % number of applications. In addition, it provides a mechanism for associating semantics
  % with new notational constructs. In this way, mathematical concepts that are not in the
  % base collection of tags can still be encoded (see section 4.2.6).
  The base set of content elements are chosen to be adequate for simple coding of most of
  the formulas used from kindergarten to the end of high school in the United States, and
  probably beyond through the first two years of college, that is up to A-Level or
  Baccalaureate level in
  Europe.\hfill{\hbox{\rm\cite[4.1.2]{WorldWideWebConsortium1999}\footnote{Repeated in
        \cite[4.1.2]{WorldWideWebConsortium2003b}, save that ``are'' becomes ``is''.}}}
\end{quotation}
\section{Conditions in MathML}
Having said the above, generally known as the ``K--12'' definition, MathML found that it
needed more sophisticated concepts, such as bound variables, to express the concepts that
are manipulated {\it informally\/} at that level.  Having introduced \verb+bvar+ with
\verb+lambda+ and \verb+int+ elements, \cite[4.2.1.8]{WorldWideWebConsortium1999} goes on
to say the following.
\begin{quotation}\noindent\em 
  The \verb+bvar+ qualifier element is also used in another important MathML
  construction. The \verb+condition+ element is used to place conditions on bound
  variables in other expressions. This allows MathML to define sets by rule, rather than
  enumeration, for example. The following markup, for instance, encodes the set $\{x | x <
  1\}$:
\end{quotation}
\begin{lstlisting}[language=MathML2]
    <set>
      <bvar><ci> x </ci></bvar>
      <condition>
        <reln><lt/>
          <ci> x </ci>
          <cn> 1 </cn>
        </reln>
      </condition>
    </set>
\end{lstlisting}
\par\noindent
Here (with the benefit of a great deal of hindsight, it should be pointed out) we can see
the start of the problem. What would we have meant if we had changed the
second\footnote{Changing both of them would have essentially been an $\alpha$-conversion
  \cite[Definition 2.1.11]{Barendregt1984}, though MathML does not analyse the concept.}
\verb+x+ to \verb+y+? We would, of course, have written the MathML equivalent of $\{x | y
< 1\}$, and the MathML would be as meaningless as that set of symbols.
%In other words, the variable in \verb+bvar+ is meant to bind the variable in the condition as well as the variables in the rest of the construction.
We therefore deduce the following (undocumented) rule.
\begin{Rule}[MathML 1]
  Variables in \verb+bvar+ constructions bind the corresponding variable occurrences in
  the scope of the parent of the \verb+bvar+.
\end{Rule}
\par
\cite[4.2.5]{WorldWideWebConsortium1999} tries to be more detailed about the meaning of
\verb+condition+.
\begin{quotation}\noindent\em
  The \verb+condition+ element is used to define the ``such that'' construct in
  mathematical expressions. Condition elements are used in a number of contexts in
  MathML. They are used to construct objects like sets and lists by rule instead of by
  enumeration. They can be used with the \verb+forall+ and \verb+exists+ operators to form
  logical expressions. And finally, they can be used in various ways in conjunction with
  certain operators. For example, they can be used with and \verb+int+ element to specify
  domains of integration, or to specify argument lists for operators like \verb+min+ and
  \verb+max+.

  The \verb+condition+ element is almost always used together with one or more \verb+bva+r
  elements. The only exception is a special usage with the \verb+min+ and \verb+max+
  operators, where the bound variable may be implied. See section 4.4.3.5 for an example
  of this usage.

  The exact interpretation depends on the context, but generally speaking, the
  \verb+condition+ element is used to restrict the permissible values of a bound variable
  appearing in another expression to those which satisfy the relations contained in the
  \verb+condition+. Similarly, when the \verb+condition+ element contains a \verb+set+,
  the values of the bound variables are restricted to that set.
  % \par
  % A condition element contains a single child which is typically a reln element, but may
  % also be an apply or a set element. The apply element is allowed so that several
  % relations can be combined by applying logical operators.
\end{quotation}
\par\noindent
Various examples are given:
``there exists $x$ such that $x^5 < 3$'' \mylst{ex:exists};
``for all $x$,$y$ such that $x^y < 1$ and $y^x < x + y$, $x < Q(y)$''
\mylst{ex:forall};
and %third example shows the use of quantifiers with condition. The following
%JHD: I don't see why this one is about quantifiers when the first already is
%markup encodes
``there exists $x < 3$ such that $x^2 = 4$'' \mylst{ex:exists2}.

The clause referred to \cite[4.4.3.5]{WorldWideWebConsortium1999} states the
following.
\begin{quotation}\noindent
  The elements to be compared may also be described using bound variables with a condition
  element and an expression to be maximised, as in: $\min_{x\notin B} x^2$
  \mylst{ex:connotin}. Note that the bound variable may be implicit:
  (Listing~\ref{lst:max}).
\end{quotation}

\begin{lstlisting}[language=MathML2,label=lst:max,
  caption=The ``implicit'' {\texttt{bvar}} of {\texttt{max}}]
    <apply><max/>
      <condition>
        <apply><and/>
          <reln><in/><ci>x</ci><ci type="set">B</ci></reln>
          <reln><notin/><ci>x</ci><ci type="set">C</ci></reln>
        </apply>
      </condition>
    </apply>
\end{lstlisting}
\subsection{A special case}\ednote{JHD to MK: we might minimise this, as it
was clearly a bug, and is now deprecated.}
This usage is described in \cite[4.2.3.4, {\texttt{min,max}}]{WorldWideWebConsortium1999}
as follows.
\begin{quotation}\noindent\em
  The \verb+min+ and \verb+max+ functions are unique in that they provide the only context
  in which the \verb+bvar+ element is optional when using a \verb+condition+; if a
  \verb+condition+ element containing a single variable is given by itself following a
  \verb+min+ or \verb+max+ operator, the variable is implicitly assumed to be bound, and
  the expression to be maximized or minimized is assumed to be the identity.
\end{quotation}
The example in Listing~\ref{lst:max} does not actually fit this prescription, since it
contains three variables, \verb+B+, \verb+C+, and \verb+x+. Of course, \verb+x+ is the
only ``genuine'' variable, but formal languages should not do this, even though
mathematicians often do. It could be argued that, since \verb+x+ has no \verb+type+
qualifier, it is scalar, and therefore the only one suitable for a \verb+max+ operator,
but this argument falls foul of the fact that \cite[4.4.1.2]{WorldWideWebConsortium1999}
the default type is {\it complex\/} scalar, and thus again not suitable for a \verb+max+
operator.
\begin{quotation}\noindent\em
  To motivate the $\lambda$-notation, consider the everyday mathematical epxression
  `$x-y$'. This can be thought of as defining either a function $f$ of $x$ or $g$ of $y$
  \dots{} And there is need for a notation that gives $f$ and $g$ different names in some
  systematic way. In practice mathematicians usually avoid this need by various `ad hoc'
  special notations, but these can get very clumsy when higher-order functions are
  involved.\hfill\hbox{\rm\cite[p. 1]{HindleySeldin2008}}
\end{quotation}
It could be argued that the ``K--12'' brief of MathML--1 rules out higher-order functions,
but we can see here the need, in a purely first-order case, to resort to ``well, you know
what I mean''. This usage is deprecated in MathML 2 \cite[4.2.3.3 and
4.4.3.4.2]{WorldWideWebConsortium2003b}, stating ``{\emph{Note that the bound variable
    must be stated even if it might be implicit in conventional notation}}'' --- the
conventional notation given is
\begin{equation}
\max\{x\in B \land x\notin C\},
\end{equation}
which some, including the first author\ednote{So JHD is committed. What about MK?}, would argue is poor notation in the
first place.
\subsection{Condition in MathML 2}
While not changing the description of \verb+condition+, MathML 2 introduced
23 examples of its usage, described in Table \ref{table4}, and a further 31
in Appendix C, described in Table \ref{Ctable}. These can be
roughly categorised as follows (where $a+b$ means ``$a$ in Chaper 4 and $b$ in
Appendix C'').
\begin{description}
\item[5+14]are used to encode $\exists n\in \N$ or $\forall n\in \N$ (or
equivalents). Strictly speaking, these usages are not necessary, because of
the equivalences below.
\begin{eqnarray}
\exists v\in S\quad p(v)&\Leftrightarrow \exists v\quad (v\in S)\land p(v)\\
\forall v\in S\quad p(v)&\Leftrightarrow \forall v\quad (v\in S)\Rightarrow p(v)
\end{eqnarray}
However, in practice, it would be better to have a convenient shorthand for
these, hence the proposal in \cite{DavenportKohlhase2009c} for OpenMath
symbols \verb+existsrestricted+ and \verb+forallrestricted+.
\item[6+4]can be replaced by the OpenMath \verb+suchthat+ construct: the
difference being that OpenMath requires a larger set to be specified (to avoid
Russell's paradox) so that MathML's $\{x|x<1\}$ has to be encoded as
$\{x\in\R|x<1\}$. See section \ref{4218}\iffull\else{} of \cite{Full}\fi.
\item[2+2]are solved by the use of \verb+map+ in OpenMath.
\end{description}
\section{Calculus Issues}
MathML and OpenMath have rather different views of calculus, which goes back
to a fundamental confusion in mathematics. These can, simplistically, be
regarded as:
\begin{itemize}
\item what one learned in calculus, which {\it we\/} will write as $\Ded$: the
``differentiation of $\epsilon$--$\delta$ analysis''. Also $\fracDed{}{x}$,
and its inverse $\Ied$;
\item what is taught in differential algebra, which {\it we\/} will write as
$\DDA$: the ``differentiation of differential algebra''. Also $\fracDDA{}{x}$,
and its inverse $\IDA$.
\end{itemize}
{\it Roughly speaking\/}, the MathML encoding corresponds more closely to
$\Ded$ and the OpenMath one to $\DDA$. If we were to look at the derivative of
$x^2$ as in Listing \ref{lst:deriv}, we might be tempted to see only trivial
syntactic differences.
\begin{lstlisting}[label=lst:deriv,caption= MathML and OpenMath differentiation compared]
<apply>                     <OMA>
  <diff/>                     <OMS cd="calculus1" "name="diff"/>
                              <OMBIND>
                                <OMS cd="fns1" name="lambda"/>
  <bvar><ci>x</ci></bvar>       <OMBVAR><OMV name="x"/></OMBVAR>
  <apply>                       <OMA>
    <power/>                      <OMS cd="arith1" name="power"/>
    <ci>x</ci>                    <OMV name="x"/>
    <cn>2</cn>                    <OMI>2</OMI>
  </apply>                      </OMA>
                              </OMBIND>
</apply>                    </OMA>
\end{lstlisting}
The difference, though, is that OpenMath can encode $\sin'=\cos$, as
\begin{lstlisting}
<OMA>
  <OMS cd="relation1" "name="eq"/>
  <OMA>
    <OMS cd="calculus1" "name="diff"/>
    <OMS cd="transc1" name="sin"/>
  </OMA>
  <OMS cd="transc1" name="cos"/>
</OMA>
\end{lstlisting}
whereas it is generally thought that MathML (1 and 2) can only directly encode
$(\sin x)'=\cos x$. However
\cite[4.4.5.2]{WorldWideWebConsortium2003b}\footnote{This text is new since
\cite{WorldWideWebConsortium1999}.} states
\begin{quotation}\noindent
[\verb+diff+] \em may be applied directly to an actual function such as sine or
cosine, thereby denoting a function which is the derivative of the original
function, or it can be applied to an expression involving a single variable
such as $\sin(x)$, or $\cos(x)$. or a polynomial in $x$.
\end{quotation}
Given the history\ednote{Do we want to put this here?} of the two standards,
this difference of encoding is not surprising, since $\DDA$
is what computer algebra systems do (and what humans do, most of the time,
even while interpreting the symbols as $\Ded$), whereas human beings generally
think they are doing $\Ded$.
\par
Partial differentiation sees the two drift further apart. For
$\frac{d^{m+n}}{dx^m dy^n}f(x,y)$, MathML would use
\begin{lstlisting}[language=MathML2]
<apply>
  <partialdiff/>
    <bvar><ci>x</ci><degree><ci>m</ci></degree></bvar>
    <bvar><ci>y</ci><degree><ci>n</ci></degree></bvar>
    <degree><apply><plus/><ci>m</ci><ci>n</ci></apply></degree>
    <apply><ci type="function">f</ci><ci>x</ci><ci>y</ci></apply>
</apply>
\end{lstlisting}
whereas OpenMath uses 
\begin{lstlisting}[language=OpenMath]
<OMA>
  <OMS cd="calculus1" name="partialdiff"/>
  <OMA><OMS cd="list1" name="list"><OMV name="m"/><OMV name="m"/></OMA>
  <OMBIND>
    <OMS cd="fns1" name="lambda"/>
    <OMBVAR><OMV name="x"/><OMV name="y"/></OMBVAR>
    <OMA><OMV name="f"><OMV name="x"/><OMV name="y"/></OMA>
  </OMBIND>
</OMA>
\end{lstlisting}
(For the problems caused by wishing to represent $\frac{d^k}{dx^m
dy^n}f(x,y)$, see \cite{Kohlhase2008}, and the proposed solution in
\cite{DavenportKohlhase2009d}.)
\par
Integration is more problematic. MathML interprets integration as an operator on
expressions in one bound variable and presents as paradigmatic examples the 
three expressions in Table \ref{tab:MMLint}, which differ in which ways the
bound variables are handled.
\begin{table}
\caption{Various MathML2 integrals}\label{tab:MMLint}
\begin{center}\lstset{frame=none,numbers=none,lineskip=-.7ex,aboveskip=-.5em,belowskip=-1em,language=MathML2}
% JHD first was 5.9, others 4.4
\begin{tabular}{|p{5.6cm}|p{4.3cm}|p{4.3cm}|}\hline
  a: $\int_0^af(x) dx$ & b: $\int_{x\in D}f(x) dx$ & c: $\int_Df(x)dx$\\\hline
\begin{lstlisting}
<apply>
  <int/>
  <bvar><ci>x</ci></bvar>
  <lowlimit><cn>0</cn></lowlimit>
  <uplimit><ci>a</ci></uplimit>
  <apply><ci>f</ci>
    <ci>x</ci>
  </apply>
</apply>
\end{lstlisting}
&
\begin{lstlisting}[language=MathML2]
<apply>
  <int/>
  <bvar><ci>x</ci></bvar>
  <condition>
    <apply><in/>
      <ci>x</ci>
      <ci>D</ci>
    </apply>
  </condition>
  <apply><ci>f</ci>
    <ci>x</ci>
  </apply>
</apply>
\end{lstlisting}
&
\begin{lstlisting}[language=MathML2]
<apply>
  <int/>
  <bvar><ci>x</ci></bvar>
  <domainofapplication>
    <ci>D</ci>
  </domainofapplication>
  <apply><ci>f</ci>
    <ci>x</ci>
  </apply>
</apply>
\end{lstlisting}
\\\hline
\end{tabular}
\end{center}
\end{table}
OpenMath can model usages (a) and (c) easily enough, via its \verb+defint+
operator: in fact usage (a) is modeled on the lines of (c), as
$\int_{[0,a]}f(x) dx$, which means that we need to give an
eccentric\footnote{Along the lines of ``the set $[b,a]$ is the same as $[a,b]$
except that, where it appears as a range of integration, we should negate the
value of the integral''! \cite{Kohlhase2008}. It is possible to regard
`backwards integration' as an ``idiom'' in the sense of
\cite{LuoCallaghan1999} and (\ref{eq:backint}) as the explanation of that
idiom, but this seems circular.} meaning to `backwards'
intervals in order to encode the traditional mathematical statement
\begin{equation}\label{eq:backint}
\int_a^bf(x) dx=-\int_b^af(x) dx.
\end{equation}
A more logical view is to regard the two notations as different, and define
$\Ied_{[a,b]}$ (via limits of Riemann sums, or whatever other definition is
appropriate), and then
\begin{equation}\label{eq:backinted}
\Ied_a^bf=%\begin{cases}
\left\{
\begin{array}{lr}
\Ied_{[a,b]}f & a\le b\\
-\Ied_{[b,a]}f & a>b\\
\end{array}\right.,
%\end{cases} 
\end{equation}
whereas 
\begin{equation}\label{eq:backintda} 
\IDA_a^bf = \left(\IDA f\right)(b)-\left(\IDA f\right)(a)
\end{equation}
by definition.
\par
Usage (b) might not worry us too much at first, since it is apparently only a
variant of (c). The challenge comes when we move to multidimensional
integration (in the $\Ied$ sense). 
\cite[p. 189]{BorweinErdelyi1995}
has a real integral over a curve in the complex plane, 
\begin{equation}\label{eq:bounds2} 
\frac1{2\pi}\int_{|t|=R}\left|\frac{f(t)}{t^{n+1}}\right| |dt|
\end{equation}
whereas 
\cite[p. 413, exercise 4, slightly reformulated]{Apostol1967}
has an integral where we clearly want to connect the variables in the
integrand to the variables defining the set:
\begin{equation}\label{eq:bounds1} 
\mathop{\int\int\int}_{\{\frac{x^2}{a^2}+\frac{y^2}{b^2}+\frac{z^2}{c^2}\le1\}}
\left(\frac{x^2}{a^2}+\frac{y^2}{b^2}+\frac{z^2}{c^2}\right)dxdydz
\end{equation}
[the original formulation was ``$\displaystyle\mathop{\int\int\int}_S\ldots$ where
$S=\cdots$'', which brings in the (important) issue of connecting bindings
betwen different formula, which is one that neither OpenMath nor MathML have
attempted to solve].\ednote{JHD to MK: do you want an OMDoc reference here?}
\section{Lifting Associative Operators}
\section{Conclusion}
We have listed three areas where MathML (1--2) and OpenMath have taken
different routes to the expressivity of mathematical meaning. In the case of
the calculus operations, this reflected a genuine split in the approaches to
the calculus operations, whether one viewed them as algebraic or analytic
operations. Since neither is `wrong', but the two {\it are\/} different (for
example the ``Fundamental Theorem of Calculus'' is a theorem from the anaytic
point of view, but a definition in the algebraic view), a converged view at
MathML/OpenMath 3 should incorporate both.
\iffull
\begin{lstlisting}[label=ex:exists,language=MathML2,
  caption={MathML-1 for ``there exists $x$ such that $x^5 < 3$''}]
    <apply><exists/>
      <bvar><ci> x </ci></bvar>
      <condition>
        <reln><lt/>
          <apply><power/>
            <ci>x</ci>
            <cn>5</cn>
          </apply>
          <cn>3</cn>
        </reln>
      </condition>
    </apply>
\end{lstlisting}

\begin{lstlisting}[label=ex:forall,language=MathML2, 
  caption={MathML-1 for ``for all $x$,$y$ such that $x^y < 1$ and $y^x < x + y$,$x < Q(y)$''}]
    <apply><forall/>
      <bvar><ci>x</ci></bvar>
      <bvar><ci>y</ci></bvar>
      <condition>
        <apply><and/>
          <reln>
            <lt/>
            <apply><power/>
              <ci>x</ci>
              <ci>y</ci>
            </apply>
            <cn>1</cn>
          </reln>
          <reln>
            <lt/>
            <apply><power/>
              <ci>y</ci>
              <ci>x</ci>
            </apply>
            <apply><plus/>
              <ci>y</ci>
              <ci>x</ci>
            </apply>
          </reln>
        </apply>
      </condition>
      <reln><lt/>
         <ci> x </ci>
         <apply>
           <fn><ci> x </ci></fn>
           <ci> y </ci>
         </apply>
      </reln>
    </apply>
\end{lstlisting}

\begin{lstlisting}[label=ex:exists2,language=MathML2,
  caption={MathML-1 for ``there exists $x < 3$ such that $x^2=4$''}]
    <apply>
      <exists/>
      <bvar><ci> x </ci></bvar>
      <condition>
        <reln><lt/><ci>x</ci><cn>3</cn></reln>
      </condition>
      <reln>
        <eq/>
        <apply>
          <power/><ci>x</ci><cn>2</cn>
        </apply>
        <cn>4</cn>
      </reln>
    </apply>
\end{lstlisting}
\fi
\iffull

\begin{lstlisting}[label=ex:connotin,language=MathML2,
  caption={MathML-1 for ``there exists $x$ such that $x^5 < 3$''}]
    <apply><min/>
      <bvar><ci>x</ci></bvar>
      <condition>
        <reln><notin/><ci> x </ci><ci type="set"> B </ci></reln>
      </condition>
      <apply>
        <power/>
        <ci> x </ci>
        <cn> 2 </cn>
      </apply>
    </apply>
\end{lstlisting}
\fi

\begin{table}[h]
\caption{{\tt condition} in Chapter  4\label{table4}}
\begin{tabular}{llll}
MathML 2&heading&\iffull This\else\cite{Full}\fi&resolution\\
Chapter 4&&\iffull document\fi&\\
4.2.1.8&qualifiers&\ref{42181}&{\tt suchthat}\\
&&\ref{42182}&{\tt map}\\
4.2.3.2&operators with&\ref{42321}&A calculus issue\\
       &qualifiers&\ref{42322}&{\tt suchthat}\\
       &          &\ref{42323}&no solution\\
       &          &\ref{42324}&{\tt forallrestricted}${}^1$\\
4.2.5  &Conditions&\ref{4251}&none needed\\
       &          &\ref{4252}&{\tt forall/existsrestricted}\\
       &          &\ref{4253}&{\tt existsrestricted}${}^1$\\
4.4.2.7.2&Conditions&\ref{442723}&{\tt suchthat}\\
4.4.3.4&Maximum&\ref{44341}&{\tt map}?${}^2$\\
&&\ref{44342}&{\tt suchthat}\\
4.4.3.17&{\tt forall}&\ref{443171}&{\tt forallrestricted}${}^3$\\
        &            &\ref{443172}&{\tt forallrestricted}${}^3$\\
4.4.5.1&{\tt int}&\ref{4451}&Needs work\\
4.4.5.6.1&{\tt bvar}&\ref{44561}&{\tt suchthat}\\
4.4.5.6.2&{\tt bvar}&\ref{44562}&Needs work\\
4.4.6.1.2&{\tt set}&\ref{44612}&{\tt suchthat}\\
4.4.6.2.2&{\tt list}&\ref{44622}&{\tt integer\_interval}\\
4.4.6.7&Subset&\ref{4467}&loose\\
4.4.7.1&Sum&\ref{4471}&None needed\\
4.4.7.2&Product&\ref{4472}&None needed\\
4.4.7.3&Limit&\ref{4473}&Use {\tt limit1} CD\\
\end{tabular}
\begin{center}
Notes
\end{center}
\begin{enumerate}\itemsep=0pt
\setlength\itemsep{0pt}
\item or nothing special.
\item the query is caused by the fact that this fragment is
close to meaningless.
\item but {\tt forallrestricted} doesn't do a perfect job.
\end{enumerate}
\end{table}
\begin{table}[h]
\caption{{\tt condition} in Appendix C\label{Ctable}}
\begin{tabular}{llll}
MathML 2&heading&\iffull This\else\cite{Full}\fi&resolution\\
Appendix C&&\iffull document\fi&\\
C.2.2.4&{\tt interval}&section \ref{C224}&{\tt suchthat}; ??\\
C.2.2.5&{\tt inverse}&section \ref{C225}&{\tt forallrestricted}\\
C.2.2.7&{\tt condition}&section \ref{C2271}&{\tt suchthat}\\
C.2.2.7&{\tt condition}&section \ref{C2272}&Needs work\\
C.2.2.14&{\tt image}&section \ref{C2214}&{\tt forallrestricted}${}^1$\\
C.2.2.15&{\tt domainofapplication}&section \ref{C2215}&none needed\\
C.2.3.1&{\tt quotient}&section \ref{C231}&{\tt forallrestricted}${}^1$\\
C.2.3.2&{\tt factorial}&section \ref{C232}&{\tt forallrestricted}${}^1$\\
C.2.3.3&{\tt divide}&section \ref{C233}&{\tt forallrestricted}${}^1$\\
C.2.3.4&{\tt max}&section \ref{C234}&{\tt map}\\
C.2.3.5&{\tt min}&section \ref{C235}&{\tt map}\\
C.2.3.7&{\tt plus}&section \ref{C237}&{\tt forallrestricted}${}^1$\\
C.2.3.8&{\tt power}&section \ref{C238}&{\tt forallrestricted}${}^1$\\
C.2.3.9&{\tt quotient}&section \ref{C239}&As section \ref{C231}\\
C.2.3.10&{\tt times}&section \ref{C2310}&No formal MathML;\\
&&&As section \ref{C237}\\
C.2.3.18&{\tt times}&section \ref{C2318}&Apparently meaningless\\
C.2.3.23&{\tt real}&section \ref{C2323}&{\tt forallrestricted}${}^1$\\
C.2.5.1&{\tt int}&section \ref{C251}&As section \ref{C2272}\\
C.2.5.6&{\tt bvar}&section \ref{C256}&{\tt forallrestricted}${}^1$\\
C.2.5.8&{\tt divergence}&section \ref{C258}&Use {\tt limit1} CD\\
C.2.6.1&{\tt set}&section \ref{C261}&{\tt suchthat}\\
C.2.6.2&{\tt list}&section \ref{C262}&{\tt suchthat/list1}\\
C.2.6.7&{\tt subset}&section \ref{C267}&Apparently meaningless\\
C.2.7.1&{\tt sum}&section \ref{C271}&{\tt sum}\\
C.2.7.2&{\tt product}&section \ref{C272}&{\tt product}\\
C.2.7.3&{\tt limit}&section \ref{C273}&{\tt limit}\\
C.2.10.2&{\tt matrix}&section \ref{C2102}&No equivalent\\
C.2.11.3&{\tt rational}&section \ref{C2113}&{\tt forallrestricted}\\
C.2.11.6&{\tt primes}&section \ref{C2116}&{\tt forallrestricted}${}^1$\\
C.2.11.15&{\tt infinity}&section \ref{C211151}&{\tt forallrestricted}${}^1$\\
C.2.11.15&{\tt infinity}&section \ref{C211152}&Use {\tt limit1} CD\\
\end{tabular}
\begin{center}
${}^1$ or nothing special
\end{center}
\end{table}

\bibliography{jhd}
\vfill
\pagebreak
\iffull
\centerline{--- Old version from here ---}
%\begin{abstract}\noindent
%JHD's attempt to look at the uses of \verb+condition+ in the MathML2
%specification. Chapter 4 and Appendix C are now complete, in the sense that
%every use of \verb+condition+ in them has been analysed. The conclusion is
%that almost all of the can be represented in terms of current OpenMath, or
%(better but not strictly necessary) current OpenMath with two additional
%symbols: \verb+forallrestricted+ and \verb+existsrestricted+. 
%\par
%This version updated following the flashmeeting
%(\url{http://fm.ea-tel.eu/fm/flashmeeting.php?pwd=356a23-15076}) on 6 November
%2008.
%\end{abstract}
%\section{Introduction}
%This note follows from the OpenMath tele-conference at 15:00
%GMT\footnote{16:00 BST, 17:00 CET.} on 10/10\discretionary{/}{/}{/}2008. See
%Michael Kohlhase (\url{m.kohlhase@jacobs-university.de})'s mail
%\url{48F05077.7010004@jacobs-university.de}.
%%Date: Sat, 11 Oct 2008 09:06:31 +0200
%\par
%The summary reads as follows.
%\begin{quotation}\noindent
%James was tasked to make sense of the integration/differentiation
%examples from the MathML2 spec and make concrete suggestions for the
%expression-based calculus CD.
%\end{quotation}
%The detailed record shows that it should be a wider look at {\it all\/} uses
%of condition, and this version is an attempt to look at all occurrences of
%\verb+condition+ in the MathML2 specification.
\section{The proposal}
This proposal comes out of the flashmeeting
(\url{http://fm.ea-tel.eu/fm/flashmeeting.php?pwd=356a23-15076}) on 6 November
2008.
As JHD understands it, the proposal is to remove \verb+condition+ elements
during the pragmatic$\rightarrow$strict conversion, in the following way.
Consider the pragmatic MathML in table \ref{pragmatic} (here \verb+X+ stands
for any number of variables in the \verb+bvar+ construct).
\begin{table}[h]
\caption{Pragmatic MathML\label{pragmatic}}
\begin{lstlisting}[language=MathML2,mathescape]
<apply>
  $W$
  <bvar> $X$ </bvar>
  <condition>$Y$</condition>
  $Z$
<apply>
\end{lstlisting}
\end{table}
Let us first assume that \verb+W+ is a normal MathML symbol, say \verb+vvv+.
Then the strict equivalent of this becomes\footnote{In the FlashMeeting, JHD
used {\tt mark2} as the suffix. {\tt withcond} seems more appropriate.} the
following.
\begin{table}[h]
\caption{Strict MathML\label{strict}}
\begin{lstlisting}
<apply>
  $vvvwithcond$
  <bvar>$X$ </bvar>
  <apply>
    <vvvcond/>
    $Y$
    $Z$
  </apply>
<apply>
\end{lstlisting}
\end{table}
\subsection{A worked example --- {\tt forall}}
Section \ref{42324} below gives an exampe of pragmatic MathML that would
become the following.
\begin{lstlisting}[language=MathML2]
<apply>
  <forallwithcond/>
  <bvar><ci> x </ci></bvar>
  <apply>
    <forallcond>
    <apply><lt/>
      <ci> x </ci><cn> 9 </cn>
    </apply>
    <apply><lt/>
      <ci> x </ci><cn> 10 </cn>
    </apply>
  </apply>
</apply>
\end{lstlisting}
It is then {\it always\/} legitimate to replace \verb+forallcond+ by
\verb+implies+, which could be done by the translator, or via an OpenMath
Formal Mathematical Property (FMP). We should note, though, that this a
property of \verb+forall+: \verb+existscond+ should be replaced by
\verb+and+. Note also that it is not strictly necessary to have separate symbols
\verb+forallcond+ and \verb+existscond+: there could\footnote{As DPC argued at
the FlashMeeting.} just be one symbol
\verb+cond+, but then the replacement rules for simple cases would become much
more complicated, and could not be a simple replacement rule.
\subsection{Ongoing work}
The following cases still need resolving.
\begin{enumerate}
\item When the first child of \verb+apply+ is a \verb+csymbol+
\item When the first child of \verb+apply+ is a \verb+ci+
\item When the first child of \verb+apply+ is a more complex expression.
\item Uses such as \verb+set+ (see section \ref{42181}) where there is no
\verb+apply+.
\end{enumerate}
I suggest that cases 1--3 are resolved uniformly, generating the strict MathML
in table \ref{strict2}.
\begin{table}[h]
\caption{Strict MathML for hard cases\label{strict2}}
\begin{lstlisting}[language=MathML2]
<apply>
  <apply>
    <genericwithcond>
    $W$
  </apply>
  <bvar>$X$</bvar>
  <apply>
    <genericcond>
    $Y$
    $Z$
  </apply>
<apply>
\end{lstlisting}
\end{table}
\section{The current state of MathML2}
From table \ref{table4} (page \pageref{table4}) we see that the first dubious
case in the body of the standard is described in our section \ref{42321},
where the alternative in MK's note \cite{Kohlhase2008} seems not to conform to OpenMath
in practice. This example also appears in sections \ref{4451} and \ref{44562}.
%\par
From table \ref{Ctable} (page \pageref{Ctable}) we see that the only dubious
case in Appendix C is in our section \ref{C2272}, which is essentially the same
example.
\par
Our section \ref{42323} illustrates a problem with multivariate definite
integrals that OpenMath cannot represent directly, and that MathML requires a
non-intuitive correspondence of order of variables to express.
\par
We note that our section \ref{4252} is an excellent tribute to the power of the
proposed new symbols \verb+forallrestricted+ and \verb+existsrestricted+. JHD
has been trying to find the origin of these symbols, but it seems not to be in
his archive. Memory is that MK suggested \verb+forallrestricted+. The point is
that the head of an \verb+OMBIND+ need for be a symbol, but can be a compound
expression, so using 
\begin{lstlisting}
<OMA>
  <OMS name="forallrestricted" cd="quant2"/>
  <OMV name="S"/>
</OMA>
\end{lstlisting}
as the head is the same as using \verb+<OMS name="forall" cd="quant1"/>+
except that the variable(s) bound are restricted to range over $S$. Note that
the bound variables do {\it not\/} explicitly appear in the restriction, so
this does not fall foul of OpenMath's requirement highlighed on page
\pageref{OMbound}.
We should note what \verb+forallrestricted+ does, and does not, encode.
\begin{description}
\item[does]$\forall n\in\N$, $\forall m,n\in\N$, $\forall x\in[0,1]$, $\forall
x\in(0,\infty)$ (but not the equivalent $\forall x>0$).
\item[does not]$\forall n\in\N,x \in\R$ (this needs two nested
\verb+forallrestricted+s), $\forall n>2$, $\forall m<n\in\N$.
\end{description}
It is a legitimate argument that this \verb+forallrestricted+ symbol is
privileging the use of $\in$ in what MathML called \verb+condition+s.
\fi
\iffull
\section{Chapter 4}
This chapter is the substantive specification of MathML (Content).
\subsection{4.2.1.8 The use of qualifier elements}\label{4218}
This section introduces the {\tt condition} element, in what might be called a
``proof by example'' style.
\subsubsection{4.2.1.8(1) The use of qualifier elements}\label{42181}
This section contains the following quotation.
\begin{quotation}\noindent
A {\tt condition} element can be used to place restrictions directly on the
bound variable. This allows MathML to define sets by rule, rather than
enumeration. The following markup, for instance, encodes the set $\{x | x <
1\}$: 
\end{quotation}
\begin{lstlisting}[language=MathML2]
<set>
  <bvar><ci> x </ci></bvar>
  <condition>
    <apply>
      <lt/>
      <ci> x </ci>
      <cn> 1 </cn>
    </apply>
  </condition>
  <ci> x </ci>
</set>
\end{lstlisting}
This can be converted into OpenMath by means of \verb+suchthat+: here is an
OpenMath example from \verb+set1+ reworked to encode the same mathematics.
\begin{lstlisting}
<OMOBJ xmlns="http://www.openmath.org/OpenMath" version="2.0"
       cdbase="http://www.openmath.org/cd">
  <OMA>
    <OMS cd="set1" name="suchthat"/>
    <OMS cd="setname1" name="R"/>
    <OMBIND>
      <OMS cd="fns1" name="lambda"/>
      <OMBVAR> <OMV name="x"/> </OMBVAR>
      <OMA>
        <OMS cd="relation1" name="lt"/>
        <OMV name="x"/>
        <OMI> 1 </OMI>
      </OMA>
    </OMBIND>
  </OMA>
</OMOBJ>
\end{lstlisting}
We note that the OpenMath makes it explicit that it is $(-\infty,1)$ that is
meant, not, say, $[0,1)$, and equally that it is $\R$, not $\Z$ or some other
set, that is the base type.
\subsubsection{4.2.1.8(2) The use of qualifier elements}\label{42182}
This section contains the following quotation.
\begin{quotation}\noindent
Another typical use is the "lifting" of $n$-ary operators to "big operators",
for instance the $n$-ary {\tt union} operator to the {\tt union} operator over
sets, as the union of the $U$-complements over a family $F$ of sets in this
construction.
\end{quotation} 
\begin{lstlisting}[language=MathML2]
<apply>
  <union/>
  <bvar><ci>S</ci></bvar>
  <condition>
    <apply><in/><ci>S</ci><ci>F</ci></apply>
  </condition>
  <apply><setdiff/><ci>U</ci><ci>S</ci></apply>
</apply>
\end{lstlisting}
This falls foul of the ambiguity in MathML's \verb+union+
constructor\footnote{At the end of section 4.2.3 of the MathML specification,
we read
\begin{quotation}\noindent
If qualifiers are used, they should be followed by a single child element
representing a function or an expression in the bound variables specified in
the {\tt bvar} qualifiers.
Mathematically the operation is then taken to be over the arguments generated
by this function ranging over the specified domain of application, rather than
over an explicit list of arguments as is the case when qualifier schemata are
not used. 
\end{quotation}
A purist might objection that the presence of the qualifier is changing the
fundamental semantics of the $n$-ary operator.}
highlighted in \cite[especially slide 14]{JHD-JEM}. The best OpenMath
translation would seem to be on the following lines.
\begin{lstlisting}
<OMA>
  <OMS name="big_union" cd="set3"/>
  <OMA>
    <OMS name="map" cd="set1"/>
    <OMBIND>
      <OMS cd="fns1" name="lambda"/>
      <OMBVAR> <OMV name="S"/> </OMBVAR>
      <OMA>
        <OMS name="setdiff" cd="set1"/>
        <OMV name="U"/>
        <OMV name="S"/>
      </OMA>
    </OMBIND>
    <OMV name="F"/>
  </OMA>
</OMA>
\end{lstlisting}
\subsection{4.2.2.2 Constructors}\label{4222}
This contains the sentence
\begin{quotation}\noindent
For example, a {\tt bvar} and a {\tt condition} element can be used to define
lists where membership depends on satisfying certain conditions. 
\end{quotation}
No example is given here, but the example in 4.2.1.8 (our section \ref{42181})
could be regarded as typical.
\subsection{4.2.3.2 Operators taking Qualifiers}\label{4232}
This section lists {\tt condition} among the qualifiers, and the operators
taking qualifiers (not necessarily {\tt condition}) as follows (* indicates
that section 4.2.3 of the MathML specification states that they do not take
{\tt condition} as a qualifier).
\begin{description}
\item[operators]{\tt int}\footnote{Sections \ref{42321}, \ref{C2272},
\ref{C2215}, \ref{C251}.}, {\tt sum}\footnote{Sections \ref{4471} and \ref{C271}.}, {\tt
product}\footnote{Section \ref{C272}.}, {\tt root, diff}*, {\tt partialdiff}*,
{\tt limit}\footnote{Sections \ref{4473}, \ref{C258}, \ref{C273}, \ref{C211152}.}, {\tt log}*,
{\tt moment}*, {\tt forall}\footnote{Sections
\ref{C225}, \ref{C2214}, \ref{C231}, \ref{C232}, \ref{C233}, \ref{C237},
\ref{C238}, \ref{C2310}, \ref{C2318}, \ref{C2323}. \ref{C256}, \ref{C2113},
\ref{C2116}, \ref{C211151}.}, {\tt exists}.
\item[$n$-ary operators]{\tt plus, times, max}\footnote{Sections \ref{442723},
\ref{C234}.},
{\tt min}\footnote{Sections \ref{44341}, \ref{C235}.}, {\tt gcd, lcm, mean, sdev,
variance, median, mode, and, or, xor, union}\footnote{Section \ref{42182}.},
{\tt intersect, cartesianproduct, compose, eq, leq, lt, geq, gt}.
\item[user defined operators]{\tt csymbol, ci}.
\item[missing](or not regarded as operators in MathML) {\tt
set}\footnote{Sections \ref{42181}, \ref{C261}.}, {\tt
list}\footnote{Section \ref{C262}.}, {\tt interval}\footnote{Section
\ref{C224}} {\tt matrix}\footnote{Section \ref{C2102}.}.
\item[spuriously absent]{\tt subset}\footnote{Sections \ref{4467} and
\ref{C267} --- note that we can make no sense of this last example.} and in
theory all other $n$-ary relations --- see section \ref{4467}.
\end{description}
\begin{quotation}\noindent
The ({\tt lowlimit},{\tt uplimit}) pair, the {\tt interval} and the {\tt
condition} are all shorthand notations specifying a particular domain of
application and should not be used if {\tt domainofapplication} is used.
\end{quotation}
It is not clear to the current author how the example in section \ref{C224}
can be cast in this mould, though.
\subsubsection{4.2.3.2 Operators taking Qualifiers (1)}\label{42321}
{\tt condition} has the following example.
\begin{lstlisting}[language=MathML2]
<apply>
  <int/>
  <bvar><ci>x</ci></bvar>
  <condition>
    <apply><in/><ci>x</ci><ci type="set">C</ci></apply>
  </condition>
  <apply><sin/><ci>x</ci></apply>
</apply>
\end{lstlisting}
This is a special case of the example discussed in section \ref{C2272}
\subsubsection{4.2.3.2 Operators taking Qualifiers (2)}\label{42322}
It is also stated that
\begin{lstlisting}[language=MathML2]
<apply>
  <int/>
  <bvar><ci>x</ci></bvar>
  <lowlimit><cn>0</cn></lowlimit>
  <uplimit><cn>1</cn></uplimit>
  <apply><power/><ci>x</ci><cn>2</cn></apply>
</apply>
\end{lstlisting}
(whose OpenMath equivalent in terms of {\tt calculus1} is
\begin{lstlisting}
<OMA>
  <OMS cd="calculus1" name="defint"/>
  <OMA>
    <OMS name="interval_cc" cd="interval1"/>
    <OMI> 0 </OMI>
    <OMI> 1 </OMI>
  </OMA>
  <OMBIND>
    <OMS cd="fns1" name="lambda"/>
    <OMBVAR> <OMV name="x"/> </OMBVAR>
    <OMA>
      <OMS cd="arith1" name="power"/>
      <OMV name="x"/>
      <OMI> 2 </OMI>
    </OMA>
  </OMBIND>
</OMA>
\end{lstlisting}
) can be written as
\begin{lstlisting}[language=MathML2]
<apply>
  <int/>
  <bvar><ci>x</ci></bvar>
  <domainofapplication>
    <set>
      <bvar><ci>t</ci></bvar>
      <condition>
        <apply>
          <and/>
          <apply><leq/><cn>0</cn><ci>t</ci></apply>
          <apply><leq/><ci>t</ci><cn>1</cn></apply>
        </apply>
      </condition>
      <ci>t</ci>
    </set>
  </domainofapplication>
  <apply><power/><ci>x</ci><cn>2</cn></apply>
</apply>
\end{lstlisting} 
The OpenMath equivalent of this would probably be the following.
\begin{lstlisting}
<OMA>
  <OMS cd="calculus1" name="defint"/>
  <OMA>
    <OMS cd="set1" name="suchthat"/>
    <OMS name="R" cd="setname1"/>
    <OMBIND>
      <OMS cd="fns1" name="lambda"/>
      <OMBVAR> <OMV name="t"/> </OMBVAR>
      <OMA>
        <OMS name="and" cd="logic1"/>
        <OMA>
          <OMS cd="relation1" name="gt"/>
          <OMV name="t"/>
          <OMI> 0 </OMI>
        </OMA>
        <OMA>
          <OMS cd="relation1" name="lt"/>
          <OMV name="t"/>
          <OMI> 1 </OMI>
        </OMA>
      </OMA>
    </OMBIND>
  </OMA>
  <OMBIND>
    <OMS cd="fns1" name="lambda"/>
    <OMBVAR> <OMV name="x"/> </OMBVAR>
    <OMA>
      <OMS cd="arith1" name="power"/>
      <OMV name="x"/>
      <OMI> 2 </OMI>
    </OMA>
  </OMBIND>
</OMA>
\end{lstlisting}
\subsubsection{4.2.3.2 Operators taking Qualifiers (3)}\label{42323}
The example continues as follows.
\begin{quotation}\noindent
This use extends to multivariate domains by using extra bound variables and a
domain corresponding to a cartesian product as in:
\end{quotation}
\begin{lstlisting}[language=MathML2]
<apply>
  <int/>
  <bvar><ci>x</ci></bvar>
  <bvar><ci>y</ci></bvar>
  <domainofapplication>
    <set>
      <bvar><ci>t</ci></bvar>
      <bvar><ci>u</ci></bvar>
      <condition>
        <apply>
          <and/>
          <apply><leq/><cn>0</cn><ci>t</ci></apply>
          <apply><leq/><ci>t</ci><cn>1</cn></apply>
          <apply><leq/><cn>0</cn><ci>u</ci></apply>
          <apply><leq/><ci>u</ci><cn>1</cn></apply>
        </apply>
      </condition>
      <list><ci>t</ci><ci>u</ci></list>
    </set>
  </domainofapplication>
  <apply>
    <times/>
    <apply><power/><ci>x</ci><cn>2</cn></apply>
    <apply><power/><ci>y</ci><cn>3</cn></apply>
  </apply>
</apply>
\end{lstlisting}
\begin{quotation}\noindent
Note that the order of bound variables of the integral must correspond to the
order in the {\tt list} used by the {\tt set} constructor in the {\tt
domainofapplication}. 
\end{quotation}
OpenMath 2 as it (and its CDs) exists has no immediate answer to this, since
{\tt defint} from {\tt calculus1} explicitly only integrates unary functions.
Obviously one could replace it by two nested unary integrations. Trying to
represent it directly would fall foul of the potential ambiguity referred to
in the quotation immediately above. It seems to the author that one really
wants some way of representing the following expression:
\begin{equation}
\int_{x=0}^1\int_{y=0}^1x^2y^3{\rm d}x{\rm d}y,
\end{equation}
i.e. explicitly linking the variables to the bounds.
\subsubsection{4.2.3.2 Operators taking Qualifiers (4)}\label{42324}
The text on \verb+forall+ gives the following example.
\begin{lstlisting}[language=MathML2]
<apply>
  <forall/>
  <bvar><ci> x </ci></bvar>
  <condition>
    <apply><lt/>
      <ci> x </ci><cn> 9 </cn>
    </apply>
  </condition>
  <apply><lt/>
    <ci> x </ci><cn> 10 </cn>
  </apply>
</apply>
\end{lstlisting}
This could be solved with the forall/implies encoding, as in
\begin{lstlisting}
<OMBIND>
  <OMS name="forall" cd="quant1"/>
  <OMBVAR> <OMV name="x"/> </OMBVAR>
  <OMA>
    <OMS name="implies" cd="logic1"/>
    <OMA>
      <OMS name="lt" cd="relation1"/>
      <OMV name="x"/> 
      <OMI> 9 </OMI>
    </OMA>
    <OMA>
      <OMS name="lt" cd="relation1"/>
      <OMV name="x"/> 
      <OMI> 10 </OMI>
    </OMA>
  </OMA>
</OMBIND>
\end{lstlisting}
Alternatively, we could use a new symbol \verb+forallrestricted+,  as in the
following.
\begin{lstlisting}
<OMBIND>
  <OMA>
    <OMS name="forallrestricted" cd="quant2"/>
    <OMA>
      <OMS name="suchthat" cd="set1"/>
      <OMS cd="setname1" name="R"/>
      <OMBIND>
        <OMS cd="fns1" name="lambda"/>
        <OMBVAR> <OMV name="x"/> </OMBVAR>
        <OMA>
          <OMS name="lt" cd="relation1"/>
          <OMV name="x"/> 
          <OMI> 9 </OMI>
        </OMA>
      </OMBIND>
    </OMA>
  </OMA>
  <OMA>
    <OMS name="lt" cd="relation1"/>
    <OMV name="x"/> 
    <OMI> 10 </OMI>
  </OMA>
</OMBIND>
\end{lstlisting}
\subsection{4.2.5 Conditions}\label{425}
\subsubsection{First example}\label{4251}
\begin{lstlisting}[language=MathML2]
<apply>
  <exists/>
  <bvar><ci> x </ci></bvar>
  <condition>
    <apply><lt/>
      <apply>
        <power/>
        <ci>x</ci>
        <cn>5</cn>
      </apply>
      <cn>3</cn>
    </apply>
    </condition>
    <true/>
</apply>
\end{lstlisting}
The author's first reaction was ``why a condition: isn't this MathML equivalent
(and shorter)?''
\begin{lstlisting}[language=MathML2]
<apply>
  <exists/>
  <bvar><ci> x </ci></bvar>
  <apply><lt/>
    <apply>
      <power/>
      <ci>x</ci>
      <cn>5</cn>
    </apply>
    <cn>3</cn>
  </apply>
</apply>
\end{lstlisting}
Certainly this would be the obvious OpenMath encoding.
\begin{lstlisting}
<OMBIND>
  <OMS name="exists" cd="quant1"/>
  <OMBVAR> <OMV name="x"/> </OMBVAR>
  <OMA>
    <OMS name="lt" cd="relation1"/>
    <OMA>
      <OMS name="power" cd="arith1"/>
      <OMV name="x"/>
      <OMI> 5 </OMI>
    </OMA>
    <OMI> 3 </OMI>
  </OMA>
</OMBIND>
\end{lstlisting}
\subsubsection{Second example}\label{4252}
Moved to OM2009 paper.
\subsubsection{Third example}\label{4253}
\begin{lstlisting}[language=MathML2]
<apply>
  <exists/>
  <bvar><ci> x </ci></bvar>
  <condition>
    <apply><lt/><ci>x</ci><cn>3</cn></apply>
  </condition>
  <apply>
    <eq/>
    <apply>
      <power/><ci>x</ci><cn>2</cn>
    </apply>
    <cn>4</cn>
  </apply>
</apply>
\end{lstlisting}
This works well with \verb+existsrestricted+.
\begin{lstlisting}
<OMBIND>
  <OMA>
    <OMS name="existsrestricted" cd="quant2"/>
    <OMA>
      <OMS name="suchthat" cd="set1"/>
      <OMS cd="setname1" name="R"/>
      <OMBIND>
        <OMS cd="fns1" name="lambda"/>
        <OMBVAR> <OMV name="x"/> </OMBVAR>
        <OMA>
          <OMS name="lt" cd="relation1"/>
          <OMV name="x"/> 
          <OMI> 3 </OMI>
        </OMA>
      <OMBIND>
    </OMA>
  </OMA>
  <OMA>
    <OMS name="eq" cd="relation1"/>
    <OMA>
      <OMS name="power" cd="arith1"/>
      <OMV name="x"/>
      <OMI> 2 </OMI>
    </OMA>
    <OMI> 4 </OMI>
  </OMA>
</OMBIND>
\end{lstlisting}
\subsection{4.4.2.7.2 Examples (of condition)}\label{44272}
\subsubsection{First example}\label{442721}
\begin{lstlisting}[language=MathML2]
<condition>
  <apply><in/><ci> x </ci><ci type="set"> A </ci></apply>
</condition>
\end{lstlisting}
As a free-standing piece of code, it is hard to see what this means.
\subsubsection{Second example}\label{442722}
\begin{lstlisting}[language=MathML2]
<condition>
  <apply>
    <and/>
    <apply><gt/><ci> x </ci><cn> 0 </cn></apply>
    <apply><lt/><ci> x </ci><cn> 1 </cn></apply>
  </apply>
</condition>
\end{lstlisting}
As a free-standing piece of code, it is hard to see what this means, but it
crops up as a fragment of the next example.
\subsubsection{Third example}\label{442723}
\begin{lstlisting}[language=MathML2]
<apply>
  <max/>
  <bvar><ci> x </ci></bvar>
  <condition>
    <apply> <and/>
      <apply><gt/><ci> x </ci><cn> 0 </cn></apply>
      <apply><lt/><ci> x </ci><cn> 1 </cn></apply>
    </apply>
  </condition>
  <apply>
    <minus/>
    <ci> x </ci>
    <apply>
      <sin/>
      <ci> x </ci>
    </apply>
  </apply>
</apply>
\end{lstlisting}
As OpenMath does not have a \verb+max+ operator acting on functions,
the nearest translation would seem to be the following.
\begin{lstlisting} 
<OMA>
  <OMS name="max" cd="minmax1"/>
  <OMA>
    <OMS name="map" cd="set1"/>
    <OMBIND>
      <OMS name="lambda" cd="fns1"/>
      <OMBVAR> <OMV name="x"/> </OMBVAR>
      <OMA>
        <OMS name="minus" cd="arith1"/>
        <OMV name="x"/>
        <OMA>
          <OMS name="sin" cd="transc1"/>
          <OMV name="x"/>
        </OMA>
      </OMA>
    </OMBIND>
    <OMA>
      <OMS name="interval_oo" cd="interval1"/>
      <OMI> 0 </OMI>
      <OMI> 1 </OMI>
    </OMA>
  </OMA>
</OMA>
\end{lstlisting}
\subsection{4.4.3.4 Maximum and minimum ({\tt max}, {\tt min})}\label{4434}
This contains the following examples.
\subsubsection{4.4.3.4 {\tt max}, {\tt min} (1)}\label{44341}
\begin{lstlisting}[language=MathML2]
<apply>
  <min/>
  <bvar><ci>x</ci></bvar>
  <condition>
    <apply><notin/><ci> x </ci><ci type="set"> B </ci></apply>
  </condition>
  <apply>
      <power/>
    <ci> x </ci>
    <cn> 2 </cn>
  </apply>
</apply>
\end{lstlisting}
This is somewhat hard to interpret: if $x\notin B$, what {\it can\/} we say
about $x$? In general terms, though, a solution such as in our section
\ref{C234} seems appropriate.
\subsubsection{4.4.3.4 {\tt max}, {\tt min} (2)}\label{44342}
\begin{lstlisting}[language=MathML2]
<apply>
  <max/>
  <bvar><ci>x</ci></bvar>
  <condition>
    <apply><and/>
      <apply><in/><ci>x</ci><ci type="set">B</ci></apply>
      <apply><notin/><ci>x</ci><ci type="set">C</ci></apply>
    </apply>
  </condition>
  <ci>x</ci>
</apply>
\end{lstlisting}
This is a clear case of \verb}suchthat}.
\begin{lstlisting}
<OMA>
  <OMS name="max" cd="minmax1"/>
  <OMA>
    <OMS name="suchthat" cd="set1"/>
    <OMV name="B"/>
    <OMBIND>
      <OMS cd="fns1" name="lambda"/>
      <OMBVAR> <OMV name="x"/> </OMBVAR>
      <OMA>
        <OMS name="notin" cd="set1"/>
        <OMV name="x"/>
        <OMV name="C"/>
      </OMA>
    </OMBIND>
  </OMA>
</OMA>
\end{lstlisting}
\subsection{4.4.3.17 Universal quantifier ({\tt forall})}\label{44317}
This contains the following examples.
\subsubsection{4.4.3.17 {\tt forall} (1)}\label{443171}
\begin{lstlisting}[language=MathML2]
<apply>
  <forall/>
  <bvar><ci> p </ci></bvar>
  <bvar><ci> q </ci></bvar>
  <condition>
    <apply><and/>
      <apply><in/><ci> p </ci><rationals/></apply>
      <apply><in/><ci> q </ci><rationals/></apply>
      <apply><lt/><ci> p </ci><ci> q </ci></apply>
    </apply>
  </condition>
  <apply><lt/>
      <ci> p </ci>
      <apply>
          <power/>
        <ci> q </ci>
        <cn> 2 </cn>
      </apply>
  </apply>
</apply>
\end{lstlisting}
This will clearly succumb to the forall/implies encoding.
\verb+forallrestricted+ will cope with the
\verb+<apply><in/><ci> p </ci><rationals/></apply>+ clauses, but not, as
currently suggested, with the $p<q$ clause. This would give the following.
\begin{lstlisting}
<OMBIND>
  <OMA>
    <OMS name="forallrestricted" cd="quant2"/>
    <OMS name="Q" cd="setname1"/>
  </OMA>
  <OMBVAR> <OMV name="p"/> <OMV name="q"/> </OMBVAR>
  <OMA>
    <OMS name="implies" cd="logic1"/>
    <OMA>
      <OMS name="lt" cd="logic1"/>
      <OMV name="p"/>
      <OMV name="q"/> 
    </OMA>
    <OMA>
      <OMS name="lt" cd="logic1"/>
      <OMV name="p"/>
      <OMA>
        <OMS name="power" cd="arith1"/>
        <OMV name="q"/> 
        <OMI> 2 </OMI>
      </OMA>
    </OMA>
  </OMA>
</OMBIND>
\end{lstlisting}
\subsubsection{4.4.3.17 {\tt forall} (2)}\label{443172}
\begin{lstlisting}[language=MathML2]
<apply>
  <forall/>
  <bvar><ci> n </ci></bvar>
  <condition>
    <apply><and/>
      <apply><gt/><ci> n </ci><cn> 0 </cn></apply>
      <apply><in/><ci> n </ci><integers/></apply>
    </apply>
  </condition>
  <apply>
    <exists/>
    <bvar><ci> x </ci></bvar>
    <bvar><ci> y </ci></bvar>
    <bvar><ci> z </ci></bvar>
    <condition>
      <apply><and/>
        <apply><in/><ci> x </ci><integers/></apply>
        <apply><in/><ci> y </ci><integers/></apply>
        <apply><in/><ci> z </ci><integers/></apply>
      </apply>
    </condition>
    <apply>
      <eq/>
      <apply>
        <plus/>
        <apply><power/><ci> x </ci><ci> n </ci></apply>
        <apply><power/><ci> y </ci><ci> n </ci></apply>
      </apply>
      <apply><power/><ci> z </ci><ci> n </ci></apply>
    </apply>
  </apply>
</apply>
\end{lstlisting}
Again, the restricted quantifiers help a great deal. but not perfectly.
\begin{lstlisting}
<OMBIND>
  <OMA>
    <OMS name="forallrestricted" cd="quant2"/>
    <OMS name="Z" cd="setname1"/>
  </OMA>
  <OMBVAR> <OMV name="n"/> </OMBVAR>
  <OMA>
  <OMS name="implies" cd="logic1"/>
    <OMA>
      <OMS name="gt" cd="relation1"/>
      <OMV name="n"/>
      <OMI> 0 </OMI>
    </OMA>
    <OMBIND>
      <OMA>
        <OMS name="existsrestricted" cd="quant2"/>
        <OMS name="Z" cd="setname1"/>
      </OMA>
      <OMBVAR> <OMV name="x"/> <OMV name="y"/> <OMV name="z"/> </OMBVAR>
      <OMA>
        <OMS name="eq" cd="relation1"/>
        <OMA>
          <OMS name="plus" cd="arith1"/>
          <OMA>
            <OMS name="power" cd="arith1"/>
            <OMV name="x"/>
            <OMV name="n"/>
          </OMA>
          <OMA>
            <OMS name="power" cd="arith1"/>
            <OMV name="y"/>
            <OMV name="n"/>
          </OMA>
        </OMA>
        <OMA>
          <OMS name="power" cd="arith1"/>
          <OMV name="z"/>
          <OMV name="n"/>
        </OMA>
      </OMA>
    </OMBIND>
  </OMA>
</OMBIND>
\end{lstlisting}
\subsection{4.4.5.1 Integral ({\tt int})}\label{4451}
This contains the following example.
\begin{lstlisting}[language=MathML2]
<apply>
  <int/>
  <bvar><ci> x </ci></bvar>
  <condition>
    <apply><in/>
      <ci> x </ci>
      <ci type="set"> D </ci>
    </apply>
  </condition>
  <apply><ci type="function"> f </ci>
    <ci> x </ci>
  </apply>
</apply>
\end{lstlisting}
Up to renaming, this is identical to the example in section \ref{C2272}, and
the same comments apply.
\subsection{4.4.5.6.1 Bound variable ({\tt bvar})}\label{44561}
This contains the following example, which is mainly meant to illustrate the
use of the \verb+id=+ construct.
\begin{lstlisting}[language=MathML2]
<set>
  <bvar><ci id="var-x"> x </ci></bvar>
  <condition>
    <apply>
      <lt/>
      <ci definitionURL="#var-x"> x </ci>
      <cn> 1 </cn>
    </apply>
  </condition>
</set>
\end{lstlisting}
From our point of view, this looks like a \verb+suchthat+, except that the
universe is unspecified in the MathML. Taking a guess for this, we end up with
the following OpenMath.
\begin{lstlisting}
<OMA>
  <OMS cd="set1" name="suchthat"/>
  <OMS cd="setname1" name="R"/>
  <OMBIND>
    <OMS cd="fns1" name="lambda"/>
    <OMBVAR> <OMV name="x"/> </OMBVAR>
    <OMA>
      <OMS cd="relation1" name="lt"/>
      <OMV name="x"/>
      <OMI> 1 </OMI>
    </OMA>
  </OMBIND>
</OMA>
\end{lstlisting}
XML \verb+id=+ tags could probably be used here.\ednote{Anyone wish to do so?}
This might also be an occasion for {\tt OMR}.
\subsection{4.4.5.6.2 Bound variable ({\tt bvar})}\label{44562}
Up to renaming, this is identical to the example in section \ref{C2272}, and
the same comments apply.  
\begin{lstlisting}[language=MathML2]
<apply>
  <int/>
  <bvar><ci> x </ci></bvar>
  <condition>
    <apply><in/><ci> x </ci><ci> D </ci></apply>
  </condition>
  <apply><ci type="function"> f </ci>
    <ci> x </ci>
  </apply>
</apply>
\end{lstlisting}
\subsection{4.4.6.1.2 Set ({\tt set})}\label{44612}
This contains the following example.
\begin{lstlisting}[language=MathML2]
<set>
  <bvar><ci> x </ci></bvar>
  <condition>
    <apply><and/>
      <apply><lt/>
        <ci> x </ci>
        <cn> 5 </cn>
      </apply>
      <apply><in/>
        <ci> x </ci>
        <naturalnumbers/>
      </apply>
    </apply>
  </condition>
  <ci> x </ci>
</set>
\end{lstlisting}
Again, this fits very naturally as a \verb+suchthat+, indeed probably more
naturally than the original MathML.
\begin{lstlisting}
<OMA>
  <OMS cd="set1" name="suchthat"/>
  <OMS cd="setname1" name="N"/>
  <OMBIND>
    <OMS cd="fns1" name="lambda"/>
    <OMBVAR> <OMV name="x"/> </OMBVAR>
    <OMA>
      <OMS cd="relation1" name="lt"/>
      <OMV name="x"/>
      <OMI> 5 </OMI>
    </OMA>
  </OMBIND>
</OMA>
\end{lstlisting}
\subsection{4.4.6.2.2 List ({\tt list})}\label{44622}
This contains the following example.
\begin{lstlisting}[language=MathML2]
<list order="numeric">
  <bvar><ci> x </ci></bvar>
  <condition>
    <apply><lt/>
      <ci> x </ci>
      <cn> 5 </cn>
    </apply>
  </condition>
  <ci> x </ci>
</list>
\end{lstlisting}
This has no direct equivalent in OpenMath, not least because there is no
equivalent of \verb+order="numeric"+, but also because one has to assume that
the list is being selected from $\N$, because if it were selected from $\Z$ or
$\R$ is would have no infimum.
\par
The best translation is probably the following.
\begin{lstlisting}
<OMA>
  <OMS name="integer_interval" cd="interval1"/>
  <OMI> 0 </OMI>
  <OMI> 4 </OMI>
</OMA>
\end{lstlisting}
The reader may protest that \verb+integer_interval+ is ``special'', but surely
no more so than implicitly assuming selection from $\N$.
\subsection{4.4.6.7 Subset ({\tt subset})}\label{4467}
This contains the following sentence.
\begin{quotation}\noindent
The subset element is an $n$-ary set relation (see Section 4.2.4 Relations).
As an $n$-ary operator, its operands may also be generated as described in
[$n$-ary operators] Therefore it may take qualifiers.
\end{quotation} 
This appears to justify the following example.
\begin{lstlisting}[language=MathML2]
<apply>
  <subset/>
  <bvar><ci type="set">S</ci></bvar>
  <condition>
    <apply><in/>
      <ci>S</ci>
      <ci type="list">T</ci>
    </apply>
  </condition>
  <ci>S</ci>
</apply>
\end{lstlisting}
As far as the present author can see, this states that the list $T$ is
linearly ordered by $\subseteq$. OpenMath does not, and seems to have decided
that it will not\footnote{MK posed this question in e-mail of 21/9/2008,
tracked as {\tt http://wiki.openmath.org/?title=cd\%3Arelation1}.}, have
$n$-ary relations of this form.
\par
Any translation has therefore to be loose: we propose the
following\footnote{Which is conditional on assuming that this is the right way
to select elements from a {\tt list} --- a debate which we have had, but I
cannot remember the resolution.}.
\begin{lstlisting}
<OMBIND>
  <OMA>
    <OMS name="forallrestricted" cd="quant2"/>
    <OMS name="N" cd="setname1"/>
  </OMA>
  <OMBVAR> <OMV name="i"/> <OMV name="j"/> </OMBVAR>
  <OMS name="implies" cd="logic1"/>
    <OMA>
      <OMS name="lt" cd="relation1"/>
      <OMV name="i"/>
      <OMV name="j"/>
    <OMA>
      <OMS name="subset" cd="set1"/>
      <OMA>
        <OMV name="T"/>
        <OMV name="i"/>
      </OMA>
      <OMA>
        <OMV name="T"/>
        <OMV name="j"/>
      </OMA>
    </OMA>
    </OMA>
</OMBIND> 
\end{lstlisting}
\subsection{4.4.7.1 Sum ({\tt sum}) }\label{4471}
This contains the following example.
\begin{lstlisting}[language=MathML2]
<apply>
  <sum/>
  <bvar><ci> x </ci></bvar>
  <condition>
    <apply> <in/>
      <ci> x </ci>
      <ci type="set"> B </ci>
    </apply>
  </condition>
  <apply><ci type="function"> f </ci>
    <ci> x </ci>
  </apply>
</apply>
\end{lstlisting}
This has an obvious translation into OpenMath, in which $x$ isn't even needed.
It could, of course, be supplied by replaced $f$ by $\lambda x.f(x)$.
\begin{lstlisting}
<OMA>
  <OMS name="sum" cd="arith1"/>
  <OMV name="B"/>
  <OMV name="f"/>
</OMA>
\end{lstlisting}
\subsection{4.4.7.2 Product ({\tt product}) }\label{4472}
This contains the following example.
\begin{lstlisting}[language=MathML2]
<apply>
  <product/>
  <bvar><ci> x </ci></bvar>
  <condition>
    <apply> <in/>
      <ci> x </ci>
      <ci type="set"> B </ci>
    </apply>
  </condition>
  <apply><ci type="function"> f </ci>
    <ci> x </ci>
  </apply>
</apply>
\end{lstlisting}
The same remarks as in the previous section apply.
\subsection{4.4.7.3 Limit ({\tt limit}) }\label{4473}
This contains the following example.
\begin{lstlisting}[language=MathML2]
<apply>
  <limit/>
  <bvar><ci> x </ci></bvar>
  <condition>
    <apply>
      <tendsto type="above"/>
      <ci> x </ci>
      <ci> a </ci>
    </apply>
  </condition>
  <apply><sin/>
     <ci> x </ci>
  </apply>
</apply>
\end{lstlisting}
This can be translated into OpenMath as follows.
\begin{lstlisting}
<OMA>
  <OMS cd="limit1" name="limit"/>
  <OMV name="a"/>
  <OMS cd="limit1" name="above"/>
  <OMBIND>
    <OMS cd="fns1" name="lambda"/>
    <OMBVAR> <OMV name="x"/> </OMBVAR>
    <OMA>
      <OMS name="sin" cd="transc1"/>
      <OMV name="x"/>
    </OMA>
  </OMBIND>
</OMA>
\end{lstlisting}
\section{Appendix C}
\subsection{C.2.2.4 MMLdefinition: {\tt interval}}\label{C224}
This contains the following example.
\begin{lstlisting}[language=MathML2]
<interval>
  <bvar><ci>x</ci></bvar>
  <condition>
    <apply><lt/><cn>0</cn><ci>x</ci></apply>
    </condition>
</interval>
\end{lstlisting}
Presumably this represents $(0,\infty)$. Equally, this is presumably intended
to close over \verb+<interval>+, i.e. $x$ is bound in this expression, and
freely $\alpha$-convertible. However, it seems to JHD to be purely ``luck''
that this defines an interval. What about the following?
\begin{lstlisting}[language=MathML2]
<interval>
  <bvar><ci>x</ci></bvar>
  <condition>
    <apply><lt/><cn>1</cn>
       <apply> <power/>
         <ci>x</ci>
         <cn>2</cn>
         </apply>
      </apply>
    </condition>
</interval>
\end{lstlisting}
By the same logic, this is $(-\infty,-1)\cup(1,\infty)$.  As far as JHD can
see, this use of \verb+interval+ is basically a declaration of a set, coupled
with an assertion that that set is in fact an interval, which assertion is, in
fact, true in the first case, but not the second.
\par
The declaration of the set can be done with \verb+suchthat+, the assertion is
a problem OpenMath has not really addressed.
\par
It is also not clear that this fragment is in fact legal. The specification
says elsewhere [4.4.2.4.1] that
\begin{quotation}\noindent
The interval element expects two child elements that evaluate to real numbers. 
\end{quotation} 
\subsection{C.2.2.5 MMLdefinition: {\tt inverse}}\label{C225}
This contains the following example (our formatting).
\begin{lstlisting}[language=MathML2]
<apply><forall/>
  <bvar><ci>y</ci></bvar>
  <bvar><ci type="function">f</ci></bvar>
  <condition>
    <apply><in/>
      <ci>y</ci>
      <apply>
        <csymbol definitionURL="domain">
          <mtext>Domain</mtext></csymbol>
        <apply><inverse/><ci type="function">f</ci></apply>
      </apply>
    </apply>
  </condition>
  <apply><eq/>
    <apply><ci type="function">f</ci>
      <apply><apply><inverse/><ci type="function">f</ci></apply>
        <ci>y</ci>
      </apply>
    </apply>
    <ci>y</ci>
  </apply>
</apply>
\end{lstlisting}
The associated textual description is 
\begin{lstlisting}
ForAll( y, such y in domain( f^(-1) ), f( f^(-1)(y) ) = y
\end{lstlisting}
which does not include the fact that $f$ is in the scope of $\forall$.
\par
This usage seems to be basically a ``typed quantifier'', and as such could
look like the following.
\begin{lstlisting}
<OMBIND>
  <OMA>
    <OMS name="forallrestricted" cd="quant2"/>
    <OMA>
      <OMS name="domain" cd="fns1"/>
      <OMA>
        <OMS name="inverse" cd="fns1"/>
        <OMV name="f"/>
      </OMA>
    </OMA>
  </OMA>
  <OMBVAR> <OMV name="y"/> </OMBVAR>
  <OMA>
    <OMS name="eq" cd="relation1"/>
    <OMA>
      <OMV name= "f"/>
      <OMA>
        <OMA>
          <OMS name="inverse" cd="fns1"/>
          <OMV name= "f"/>
        </OMA>
        <OMV name= "y"/>
      </OMA>
    </OMA>
    <OMV name= "y"/>
  </OMA>
</OMBIND>
\end{lstlisting}
This resembles the ``vernacular'', and does not bind $f$. If we wanted to do
so, along the lines of the MathML, we would have to wrap the whole thing in
another \verb+forall+. The two cannot be combined, as we want to be able to
$\alpha$-convert $f$ in the argument of \verb+forallrestricted+, and, as MK's
note of the tele-conference reads:
\begin{quotation}\noindent
In particular, we do not
      want to accept the occurrence of the bound variable in the
      (complex) binding operator. In particular, the OpenMath2 standard
      restricts alpha-conversion to the second and third children of the
      OMBIND, which is consistent with this view.
\end{quotation} 
See section \ref{sec:defintcond} below for an example of where this rule
bites.
\subsection{C.2.2.7 MMLdefinition: {\tt condition}}
This section contains two examples.
\subsubsection{C.2.2.7(1) MMLdefinition: {\tt condition}}\label{C2271}
\begin{lstlisting}[language=MathML2]
<condition>
  <apply><lt/>
    <apply><power/><ci>x</ci><cn>5</cn></apply>
    <cn>3</cn>
  </apply>
</condition>
\end{lstlisting}
It is hard to understand the meaning of this fragment in isolation (and
presumably the reader was not intended to do so). If it was intended to be
part of an encoding of a set (mathematically, representing the interval
$(-\infty,3^{1/5})$, or possibly $[0,3^{1/5})$), then \verb+suchthat+ from
\verb+set1+ is appropriate. Here is the example from \verb+set1+ reworked to
match the MathML example (as $(-\infty,3^{1/5})$: if one wanted $[0,3^{1/5})$
one would have to add $x\ge0$, or change from $\bf R$ as the base set).
\begin{lstlisting}
<OMOBJ xmlns="http://www.openmath.org/OpenMath" version="2.0"
       cdbase="http://www.openmath.org/cd">
  <OMA>
    <OMS cd="set1" name="suchthat"/>
    <OMS cd="setname1" name="R"/>
    <OMBIND>
      <OMS cd="fns1" name="lambda"/>
      <OMBVAR> <OMV name="x"/> </OMBVAR>
      <OMA>
        <OMS cd="relation1" name="lt"/>
        <OMA>
          <OMS cd="arith1" name="power"/>
          <OMV name="x"/>
          <OMI> 5 </OMI>
        </OMA>
        <OMI> 3 </OMI>
      </OMA>
    </OMBIND>
  </OMA>
</OMOBJ>
\end{lstlisting}
We should note that the OpenMath makes it clear that $x$ is bound by the
(\verb+OMBIND+ whose first child is the)
\verb+lambda+, whereas the MathML, being only a fragment, does not state the
scope.
\subsubsection{C.2.2.7(2) MMLdefinition: {\tt
condition}}\label{sec:defintcond}\label{C2272}
(Our formatting.)
\begin{lstlisting}[language=MathML2]
<apply><int/>
  <bvar><ci>x</ci></bvar>
  <condition>
    <apply><in/><ci>x</ci><ci type="set">C</ci></apply>
  </condition>
  <apply><ci type="function">f</ci><ci>x</ci></apply>
</apply>
\end{lstlisting}
This seems, to JHD, to be typical of the confusion that can arise over
\verb+condition+s. Let us look at this expression in the vernacular: 
\begin{equation}\label{defintcond}
\int_{x\in C}f(x){\rm d} x,
\end{equation}
and observe that this is probably equivalent to $\int_{C}f(x){\rm d} x$.
\par
In terms of \verb+calculus1+ (calculus with functions \cite{JHD})
this can be expressed as 
\begin{lstlisting}
<OMA>
  <OMS cd="calculus1" name="defint"/>
  <OMV name="C"/>
  <OMV name="f"/>
</OMA>
\end{lstlisting}
and $x$ doesn't appear at all. If one wants it to, then one writes
\begin{lstlisting}
<OMA>
  <OMS cd="calculus1" name="defint"/>
  <OMV name="C"/>
  <OMBIND>
    <OMS cd="fns1" name="lambda"/>
    <OMBVAR>
      <OMV name="x"/>
    </OMBVAR>
    <OMA>
      <OMV name="f"/>
      <OMV name="x"/>
    </OMA>
  </OMBIND>
</OMA>
\end{lstlisting}
as in the example in \verb+calculus1+.
\par
MK's note \cite{Kohlhase2008} on \verb+calculus3+ (calculus with expressions
\cite{JHD}), suggests (in MathML syntax)
\begin{lstlisting}[language=MathML2]
<bind>
  <apply>
    <csymbol cd="calculus3">defintset</csymbol>
    <ci>C</ci>
  </apply>
  <bvar><ci>x</ci></bvar>
  <apply><ci>f</ci><ci>x</ci></apply>
</bind>
\end{lstlisting}
which is a precise translation of the previous one from the language of
functions to expressions.
In OpenMath syntax it would be the following\footnote{In the circulated
version of {\tt calculus3}, there appear to be two symbols called {\tt
defint}. The second should presumably be {\tt defintset}.}.
\begin{lstlisting}
<OMBIND>
  <OMA>
    <OMS cd="calculus3" name="defintset"/>
    <OMV name="C"/>
  </OMA>
  <OMBVAR>
    <OMV name="x"/>
  </OMBVAR>
  <OMA>
    <OMV name="f"/>
    <OMV name="x"/>
  </OMA>
</bind>
\end{lstlisting}
MK's note also suggests an alternative way of expressing (\ref{defintcond}),
which in MathML syntax would be the following.
\begin{lstlisting}[language=MathML2]
<bind>
  <apply>
    <csymbol cd="calciulus3">defintcond</csymbol>
    <apply><in/>
      <ci>x</ci>
      <ci>C</ci>
    </apply>
  </apply>
  <bvar><ci>x</ci></bvar>
  <apply><ci>f</ci><ci>x</ci></apply>
</bind> 
\end{lstlisting}
In OpenMath syntax it would be the following.
\begin{lstlisting}
<OMBIND>
  <OMA>
    <OMS cd="calculus3" name="defintcond"/>
    <OMA>
      <OMS name="in" cd="set1"/>
      <OMV name="x"/>
      <OMV name="C"/>
    </OMA>
  </OMA>
  <OMBVAR>
    <OMV name="x"/>
  </OMBVAR>
  <OMA>
    <OMV name="f"/>
    <OMV name="x"/>
  </OMA>
</bind>
\end{lstlisting}
This, however, falls foul of %OpenMath's rule that 
\begin{quotation}\noindent\label{OMbound}
the OpenMath2 standard restricts alpha-conversion to the second and third
children of the OMBIND
\end{quotation}
and hence JHD does not see how \verb+defintcond+ as postulated can make its
way into OpenMath.
\subsection{C.2.2.14 MMLdefinition: {\tt image}\label{C2214}}
This section contains the following example.
\begin{lstlisting}[language=MathML2]
<apply><forall/>
  <bvar><ci>x</ci></bvar>
  <condition>
    <apply><in/>
      <ci>x</ci>
      <apply><image/><ci>f</ci></apply>
    </apply>
  </condition>
  <apply><in/>
    <ci>x</ci>
    <apply><codomain/><ci>f</ci></apply>
  </apply>
</apply>
\end{lstlisting}
The following remarks could be made.
\begin{enumerate}
\item This cries out for \verb"forallrestricted".
\begin{lstlisting}
<OMBIND>
  <OMA>
    <OMS name="forallrestricted" cd="quant2"/>
    <OMA>
      <OMS name="image" cd="fns1"/>
      <OMV name="f"/>
    </OMA>
  </OMA>
  <OMBVAR> <OMV name="x"/> </OMBVAR>
  <OMA>
    <OMS name="in" cd="set1"/>
    <OMV name="x"/> 
    <OMA>
      <OMS name="codomain" cd="fns1"/>
      <OMV name= "f"/>
    </OMA>
  </OMA>
</OMBIND>
\end{lstlisting}
\item It is a pretty good case for the ``forall with implies'' trick.
\begin{lstlisting}
<OMBIND>
  <OMS name="forall" cd="quant1"/>
  <OMBVAR> <OMV name="x"/> </OMBVAR>
  <OMA>
    <OMS name="implies" cd="logic1"/>
    <OMA>
      <OMS name="in" cd="set1"/>
      <OMV name="x"/> 
      <OMA>
        <OMS name="image" cd="fns1"/>
        <OMV name= "f"/>
      </OMA>
    </OMA>
    <OMA>
      <OMS name="in" cd="set1"/>
      <OMV name="x"/> 
      <OMA>
        <OMS name="codomain" cd="fns1"/>
        <OMV name= "f"/>
      </OMA>
    </OMA>
  </OMA>
</OMBIND>
\end{lstlisting}
\item Why use quantifiers at all?
\begin{lstlisting}
<OMA>
  <OMS name="subset" cd="set1"/>
  <OMA>
    <OMS name="image" cd="fns1"/>
    <OMV name= "f"/>
  </OMA>
  <OMA>
    <OMS name="codomain" cd="fns1"/>
    <OMV name= "f"/>
  </OMA>
</OMBIND>
\end{lstlisting}
Indeed there could be (possibly even ought to be, since in ZF it is the
{\it definition\/} of $\subset$) a FMP of \verb+subset+ that made this
equivalent to expression 2.
\end{enumerate}
\subsection{C.2.2.15 MMLdefinition: {\tt domainofapplication}}\label{C2215}
This contains the interesting statement
\begin{quotation}\noindent
Special cases of this qualifier can be abbreviated using one of interval
{\tt condition} or an ({\tt lowlimit},{\tt uplimit}) pair.
\end{quotation}
The examples given are
\begin{lstlisting}[language=MathML2]
<apply><int/>
  <domainofapplication><ci>C</ci></domainofapplication>
  <ci>f </ci>
</apply>
\end{lstlisting}
(which is a fairly straight-forward definite integral) and
\begin{lstlisting}[language=MathML2]
<apply><int/>
      <domainofapplication>
        <set>
          <bvar><ci>t</ci></bvar>
          <condition>
            <apply><in/>
              <ci>t</ci>
              <ci type="set">C</ci>
            </apply>
          </condition>
        </set>
      </domainofapplication>
      <ci>f</ci>
</apply>
\end{lstlisting}
which seems to this author to be a redundant variant. 
\subsection{C.2.3.1 MMLdefinition: {\tt quotient}}\label{C231}
This has the following property (presumably meant to be the equivalent of
OpenMath's FMP\footnote{OpenMath's FMP for {\tt quotient} in {\tt integer1} is
currently missing the proviso $b\ne0$.}).
\begin{lstlisting}[language=MathML2]
ForAll( [a,b], b != 0, a = b*quotient(a,b) + rem(a,b) )

<apply><forall/>
  <bvar><ci>a</ci></bvar>
  <bvar><ci>b</ci></bvar>
  <condition><apply><neq/><ci>b</ci><cn>0</cn></apply></condition>
  <apply><eq/>
    <ci>a</ci>
    <apply><plus/>
      <apply><times/>
          <ci>b</ci>
          <apply><quotient/><ci>a</ci><ci>b</ci></apply>
      </apply>
  <apply><rem/><ci>a</ci><ci>b</ci></apply>
    </apply>
  </apply>
</apply>
\end{lstlisting}
Again, this seems to be a pretty good case for the ``forall with implies''
trick, though \verb+forallrestricted+ could be used.
\begin{lstlisting}
<OMBIND>
  <OMS name="forall" cd="quant1"/>
  <OMBVAR> <OMV name="a"/> <OMV name="b"/> </OMBVAR>
  <OMA>
    <OMS name="implies" cd="logic1"/>
    <OMA>
      <OMS name="neq" cd="relation1"/>
      <OMV name="b"/> 
      <OMS name="zero" cd="arith1"/>
    </OMA>
    <OMA>
      <OMS name="eq" cd="relation1"/>
      <OMV name="a"/> 
      <OMA>
        <OMS name="plus" cd="arith1"/>
        <OMA>
          <OMS name="times" cd="arith1"/>
          <OMV name="b"/> 
          <OMA>
            <OMS name="quotient" cd="integer1"/>
            <OMV name="a"/> 
            <OMV name="b"/> 
          </OMA>
        </OMA>
        <OMA>
          <OMS name="remainder" cd="integer1"/>
          <OMV name="a"/> 
          <OMV name="b"/> 
        </OMA>
      </OMA>
    </OMA>
  </OMA>
</OMBIND>
\end{lstlisting}
\subsection{C.2.3.2 MMLdefinition: {\tt factorial}}\label{C232}
This has the following property (presumably meant to be the equivalent of
OpenMath's FMP).
\begin{lstlisting}[language=MathML2]
ForAll( n, n \gt 0, n! = n*(n-1)! )

              
<apply><forall/>
  <bvar><ci>n</ci></bvar>
  <condition><apply><gt/><ci>n</ci><cn>0</cn></apply></condition>
  <apply><eq/>
    <apply><factorial/><ci>n</ci></apply>
    <apply><times/>
      <ci>n</ci>
      <apply><factorial/>
        <apply><minus/><ci>n</ci><cn>1</cn></apply>
      </apply>
    </apply>
  </apply>
</apply>
\end{lstlisting}
Again, this seems to be a pretty good case for the ``forall with implies''
trick, though \verb+forallrestricted+ could be used.
\subsection{C.2.3.3 MMLdefinition: {\tt divide}}\label{C233}
This has the following property (presumably meant to be the equivalent of
OpenMath's FMP).
\begin{lstlisting}[language=MathML2]
ForAll( a, a!= 0, a/a = 1 ) 
          
<apply><forall/>
  <bvar><ci>a</ci></bvar>
  <condition><apply><neq/><ci>a</ci><cn>0</cn></apply></condition>
  <apply><eq/>
    <apply><divide/><ci>a</ci><ci>a</ci></apply>
    <cn>1</cn>
  </apply>
</apply>
\end{lstlisting}
Again, this seems to be a pretty good case for the ``forall with implies''
trick, though \verb+forallrestricted+ could be used.
\subsection{C.2.3.4 MMLdefinition: {\tt max}}\label{C234}
This contains the interesting statement
\begin{quotation}\noindent
The elements may be listed explicitly or they may be described by a {\tt
domainofapplication}, for example, the maximum over all $x$ in the set $A$.
The {\tt domainofapplication} is often abbreviated by placing a {\tt
condition} directly on a bound variable. 
\end{quotation}
The example given is the following (our layout).
\begin{lstlisting}[language=MathML2]
<apply>
  <max/>
  <bvar><ci>y</ci></bvar>
  <condition>
    <apply>
      <in/>
      <ci>y</ci>
      <interval><cn>0</cn><cn>1</cn></interval>
    </apply>
  </condition>
  <apply><power/><ci> y</ci><cn>3</cn></apply>
</apply>
\end{lstlisting}
As OpenMath does not have a \verb+max+ operator acting on functions,
the nearest translation would seem to be the following.
\begin{lstlisting} 
<OMA>
  <OMS name="max" cd="minmax1"/>
  <OMA>
    <OMS name="map" cd="set1"/>
    <OMBIND>
      <OMS name="lambda" cd="fns1"/>
      <OMBVAR> <OMV name="y"/> </OMBVAR>
      <OMA>
        <OMS name="power" cd="arith1"/>
        <OMV name="y"/>
        <OMI> 3 </OMI>
      </OMA>
    </OMBIND>
    <OMA>
      <OMS name="interval_cc" cd="interval1"/>
      <OMI> 0 </OMI>
      <OMI> 1 </OMI>
    </OMA>
  </OMA>
</OMA>
\end{lstlisting}
We note that OpenMath seems to require us to be precise about the species of
interval we want to use.
\subsection{C.2.3.5 MMLdefinition: {\tt min}}\label{C235}
Nothing new need be said here.
\subsection{C.2.3.7 MMLdefinition: {\tt plus}}\label{C237}
The property here is the following.
\begin{lstlisting}[language=MathML2]
    Commutativity 
    <apply><forall/>
      <bvar><ci>a</ci></bvar>
      <bvar><ci>b</ci></bvar>
      <condition>
        <apply><and/>
          <apply><in/><ci>a</ci><reals/></apply>
          <apply><in/><ci>b</ci><reals/></apply>
        </apply>
      </condition>
      <apply><eq/>
        <apply><plus/><ci>a</ci><ci>b</ci></apply>
        <apply><plus/><ci>b</ci><ci>a</ci></apply>
      </apply>
    </apply>
\end{lstlisting}
Again, this seems to be a pretty good case for the ``forall with implies''
trick, though \verb+forallrestricted+ could be used.
\subsection{C.2.3.8 MMLdefinition: {\tt power}}\label{C238}
The property here is the following.
\begin{lstlisting}[language=MathML2]
ForAll( a, a!=0, a^0=1 ) 
<apply><forall/>
  <bvar><ci>a</ci></bvar>
  <condition><apply><neq/><ci>a</ci><cn>0</cn></apply></condition>
  <apply><eq/>
    <apply><power/><ci>a</ci><cn>0</cn></apply>
    <cn>1</cn>
  </apply>
</apply>
\end{lstlisting}
Again, this seems to be a pretty good case for the ``forall with implies''
trick, though \verb+forallrestricted+ could be used.
\subsection{C.2.3.9 MMLdefinition: {\tt rem}}\label{C239}
This has the same property, and solution, as {\tt quotient} (section
\ref{C231}).
\subsection{C.2.3.10 MMLdefinition: {\tt times}}\label{C2310}
The property here is the following.
\begin{lstlisting} 
ForAll( [a,b], condition(in({a,b}, Commutative)), a*b=b*a )
\end{lstlisting}
However, no formal translation is given, and it is not clear what one would
be.
\par
Later on we see the following property, to which the same remark applies as in
section \ref{C237}.
\begin{lstlisting}[language=MathML2]
<apply><forall/>
  <bvar><ci>a</ci></bvar>
  <bvar><ci>b</ci></bvar>
  <condition>
    <apply><and/>
      <apply><in/><ci>a</ci><reals/></apply>
      <apply><in/><ci>b</ci><reals/></apply>
    </apply>
  </condition>
  <apply><eq/>
    <apply><times/><ci>a</ci><ci>b</ci></apply>
    <apply><times/><ci>b</ci><ci>a</ci></apply>
  </apply>
</apply> 
\end{lstlisting}
\subsection{C.2.3.18 MMLdefinition: {\tt forall}}\label{C2318}
This contains the following example (our formatting).
\begin{lstlisting}[language=MathML2]
<apply>
  <forall/>
  <bvar><ci> x </ci></bvar>
  <condition>
    <apply><lt/><ci> x </ci><cn> 0 </cn></apply>
  </condition>
  <ci> x </ci>
</apply>
\end{lstlisting}
This seems to be $\forall x:x<0 x$. Since this last $x$ is not a Boolean, the
author respectfully submits that this example is ill-typed. In any case
\verb+forallrestricted+ would solve the issue. 
\subsection{C.2.3.23 MMLdefinition: {\tt real}}\label{C2323}
This contains the following property,
\begin{lstlisting}[language=MathML2]
<apply><forall/>
  <bvar><ci>x</ci></bvar>
  <bvar><ci>y</ci></bvar>
  <condition>
    <apply><and/>
      <apply><in/><ci>x</ci><reals/></apply>
      <apply><in/><ci>y</ci><reals/></apply>
    </apply>
  </condition>
  <apply><eq/>
    <apply><real/>
      <apply><plus/>
        <ci> x </ci>
        <apply><times/><imaginaryi/><ci>y</ci></apply>
      </apply>
    </apply>
    <ci> x </ci>
  </apply>
</apply>
\end{lstlisting}
Again, this seems to be a pretty good case for the ``forall with implies''
trick, though \verb+forallrestricted+ could be used.
\subsection{C.2.5.1 MMLdefinition: {\tt int}}\label{C251}
This contains the following example.
\begin{lstlisting}[language=MathML2]
<apply><int/>
  <bvar><ci> x </ci></bvar>
  <condition>
    <apply><in/><ci> x </ci><ci type="set"> D </ci></apply>
  </condition>
  <apply><ci type="function"> f </ci><ci> x </ci></apply>
</apply>
\end{lstlisting}
The discussion in section \ref{C2272} is appropriate here.
\subsection{C.2.5.6 MMLdefinition: {\tt bvar}}\label{C256}
This contains the following example (our formatting).
\begin{lstlisting}[language=MathML2]
<apply><forall/><bvar><ci>x</ci></bvar>
  <condition><apply><in/><ci>x</ci><reals/></apply></condition>
  <apply>
    <eq/>
    <apply><minus/><ci>x</ci><ci>x</ci></apply>
    <cn>0</cn>
  </apply>
</apply>
\end{lstlisting}
Again, this seems to be a pretty good case for the ``forall with implies''
trick, though c\verb+forallrestricted+ ould be used.
\subsection{C.2.5.8 MMLdefinition: {\tt divergence}}\label{C258}
This contains the following example.
\begin{lstlisting}[language=MathML2]
<apply>
  <eq/>
  <apply><divergence/><ci type="vectorfield">a</ci></apply>
  <apply>
    <limit/>
    <bvar><ci> V </ci></bvar>
    <condition>
      <apply>
        <tendsto/>
        <ci> V </ci>
        <cn> 0 </cn>
      </apply>
    </condition>
    <apply>
      <divide/>
      <apply>
        <int encoding="text" definitionURL="SurfaceIntegrals.htm"/>
        <bvar><ci> S</ci></bvar>
        <ci> a </ci>
      </apply>
      <ci> V </ci>
    </apply>
  </apply>
</apply>
\end{lstlisting}
Here the relevant part is the limit, which could be expressed (as it is in the
\verb+limit1+ CD) as the following.
\begin{lstlisting}
<OMA>
  <OMS cd="limit1" name="limit"/>
  <OMI> 0 </OMI>
  <OMS cd="limit1" name="both_sides"/>
  <OMBIND>
    <OMS cd="fns1" name="lambda"/>
      <OMBVAR>
      <OMV name="V"/>
      </OMBVAR>
      ...
  </OMBIND>
</OMA>
\end{lstlisting}
\subsection{C.2.6.1 MMLdefinition: {\tt set}}\label{C261}
This contains the following example.
\begin{lstlisting}[language=MathML2]
<set>
  <bvar><ci> x </ci></bvar>
  <condition>
    <apply><lt/>
      <ci> x </ci>
      <cn> 5 </cn>
    </apply>
  </condition>
  <ci>x</ci>
</set>
\end{lstlisting}
It is not clear what this means, but a plausible stab would seem to be the
following.
\begin{lstlisting}
<OMA>
  <OMS cd="set1" name="suchthat"/>
  <OMS cd="setname1" name="N"/>
  <OMBIND>
    <OMS cd="fns1" name="lambda"/>
    <OMBVAR>
      <OMV name="x"/>
    </OMBVAR>
    <OMA>
      <OMS cd="relation1" name="lt"/>
    <OMV name="x"/>
      <OMI> 5 </OMI>
    </OMA>
  </OMBIND>
</OMA>
\end{lstlisting}
\subsection{C.2.6.2 MMLdefinition: {\tt list}}\label{C262}
This contains the following example.
\begin{lstlisting}[language=MathML2]
<list order="numeric">
  <bvar><ci> x </ci></bvar>
  <condition>
    <apply><lt/>
      <ci> x </ci>
      <cn> 5 </cn>
    </apply>
  </condition>
</list>
\end{lstlisting}
There is no direct translation into OpenMath for reasons other than
\verb+condition+, but \verb+sucthat+ in \verb+list1+ seems an obvious tool to
use.
\subsection{C.2.6.7 MMLdefinition: {\tt subset}}\label{C267}
This contains the following example.
\begin{lstlisting}[language=MathML2]
<apply>
  <subset/>
  <subset/>
  <bvar><ci type="set">S</ci></bvar>
  <condition>
    <apply><in/>
      <ci>S</ci>
      <ci type="list">T</ci>
    </apply>
  </condition>
  <ci>S</ci>
</apply>
\end{lstlisting}
Even assuming the second \verb+<subset/>+ to be a mistake, the present author
can make no sense of this.
\subsection{C.2.7.1 MMLdefinition: {\tt sum}}\label{C271}
This contains the following example (our formatting).
\begin{lstlisting}[language=MathML2]
<apply><sum/>
  <bvar><ci> x </ci></bvar>
  <condition>
    <apply> <in/><ci> x </ci><ci type="set">B</ci></apply>
  </condition>
  <apply><ci type="function"> f </ci><ci> x </ci></apply>
</apply>
\end{lstlisting}
This translates straightforwardly.
\begin{lstlisting}
<OMA>
  <OMS name="sum" cd="arith1"/>
  <OMV name="S"/>
  <OMV name="f"/>
</OMA>
\end{lstlisting}
We note that there is no need to name the dummy variable at all.
\subsection{C.2.7.2 MMLdefinition: {\tt product}}\label{C272}
The example, and its OpenMath translation, are essentially identicalto the
previous section.
\subsection{C.2.7.3 MMLdefinition: {\tt limit}}\label{C273}
This contains the following example (our formatting).
\begin{lstlisting}[language=MathML2]
<apply><limit/>
  <bvar><ci>x</ci></bvar>
  <condition>
    <apply><tendsto/><ci>x</ci><cn>0</cn></apply>
  </condition>
  <apply><sin/><ci>x</ci></apply>
</apply>
\end{lstlisting}
The equivalent OpenMath would be the following.
\begin{lstlisting}
<OMA>
  <OMS cd="limit1" name="limit"/>
  <OMI> 0 </OMI>
  <OMS cd="limit1" name="both_sides"/>
  <OMS cd="transc1" name="sin"/>
  </OMBIND>
</OMA>
\end{lstlisting}
We again note that there is no need to name the dummy variable at all.
\subsection{C.2.10.2 MMLdefinition: {\tt matrix}}\label{C2102}
This contains the following example (our formatting).
\begin{lstlisting}[language=MathML2]
<matrix>
  <bvar><ci type="integer">i</ci></bvar>
  <bvar><ci type="integer">j</ci></bvar>
  <condition>
    <apply><and/>
      <apply><in/>
        <ci>i</ci>
        <interval><ci>1</ci><ci>5</ci></interval>
      </apply>
      <apply><in/>
        <ci>j</ci>
        <interval><ci>5</ci><ci>9</ci></interval>
      </apply>
    </apply>
  </condition>
  <apply><power/>
    <ci>i</ci>
    <ci>j</ci>
  </apply>
</vector>
\end{lstlisting}
We can assume that this should end \verb+</matrix>+ instead of
\verb+</vector>+, but this has no equivalent in OpenMath.
\subsection{C.2.11.3 MMLdefinition: {\tt rational}}\label{C2113}
This contains the following property (our formatting).
\begin{lstlisting}[language=MathML2]
    for all z where z is a rational, there exists 
      integers p and q with p/q = z
<apply><forall/>
  <bvar><ci>z</ci></bvar>
  <condition>
    <apply><in/><ci>z</ci><rationals/></apply>
    </condition>
  <apply><exists/>
    <bvar><ci>p</ci></bvar>
    <bvar><ci>q</ci></bvar>
    <apply><and/>
      <apply><in/><ci>p</ci><integers/></apply>
      <apply><in/><ci>q</ci><integers/></apply>
      <apply><eq/>
        <apply><divide/><ci>p</ci><ci>q</ci></apply>
        <ci>z</ci>
      </apply>
    </apply>
  </apply> 
\end{lstlisting}
\verb+forallrestricted+ seems the obvious solution, though the implies trick
could also be used.
\subsection{C.2.11.6 MMLdefinition: {\tt primes}}\label{C2116}
This contains the following property (our formatting).
\begin{lstlisting}[language=MathML2]
ForAll( [d,p], p is prime, Implies( d | p , d=1 or d=p ) ) 
<apply><forall/>
  <bvar><ci>d</ci></bvar>
  <bvar><ci>p</ci></bvar>
  <condition>
    <apply><and/>
    <apply><in/><ci>p</ci><primes/></apply>
    <apply><in/><ci>d</ci><naturalnumbers/></apply>
    </apply>
  </condition>
  <apply><implies/>
    <apply><factorof/><ci>d</ci><ci>p</ci></apply>
    <apply><or/>
      <apply><eq/><ci>d</ci><cn>1</cn></apply>
      <apply><eq/><ci>d</ci><ci>p</ci></apply>
    </apply>
  </apply>
</apply>
\end{lstlisting}
This could be encoded with the forall/implies trick, except that the result
would have two implication signs --- perfectly correct, but possibly harder to
read. We would need to nest \verb+forallrestricted+, as in the following.
\begin{lstlisting}
<OMBIND>
  <OMA>
    <OMS name="forallrestricted" cd="quant2"/>
    <OMS name="N" cd="setname1"/>
  </OMA>
  <OMBVAR> <OMV name="d"/> </OMBVAR>
  <OMBIND>
    <OMA>
      <OMS name="forallrestricted" cd="quant2"/>
      <OMS name="P" cd="setname1"/>
    </OMA>
    <OMBVAR> <OMV name="p"/> </OMBVAR>
    ...
  </OMBIND>
</OMBIND> 
\end{lstlisting}
\subsection{C.2.11.15 MMLdefinition: {\tt infinity}}\label{C21115}
This contains the following property and example (our formatting).
\subsubsection{C.2.11.15(1) MMLdefinition: {\tt infinity}}\label{C211151}
\begin{lstlisting}[language=MathML2] 
    for all reals x, x \lt infinity
<apply><forall/>
  <bvar><ci>n</ci></bvar>
  <condition><apply><in/><ci>n</ci><reals/></apply></condition>
  <apply><lt/><ci>n</ci><infinity/></apply>
</apply>
\end{lstlisting}
\verb+forallrestricted+ seems the obvious solution, though the implies trick
could also be used.
\subsubsection{C.2.11.15(2) MMLdefinition: {\tt infinity}}\label{C211152}
\begin{lstlisting}[language=MathML2]
<apply><eq/>
  <apply><limit/>
    <bvar><ci>x</ci></bvar>
    <condition>
      <apply><tendsto/><ci>x</ci><infinity/></apply>
      </condition>
    <apply><divide/><cn>1</cn><ci>x</ci></apply>
  </apply>
  <cn>0</cn>
</apply>
\end{lstlisting}
From OpenMath's point of view, this is a straightforward limit.
\begin{lstlisting} 
<OMA>
  <OMA>
    <OMS cd="limit1" name="limit"/>
    <OMS name="infinity" cd="nums1"/>
    <OMS cd="limit1" name="below"/>
    <OMBIND>
      <OMS cd="fns1" name="lambda"/>
      <OMBVAR> <OMV name="x"/> </OMBVAR>
      <OMA>
        <OMS name="divide" cd="arith1"/>
        <OMI> 1 </OMI>
        <OMV name="x"/>
      </OMA>
    </OMBIND>
  </OMA>
  <OMS cd="alg1" name="zero"/>
</OMA>
\end{lstlisting}
\begin{thebibliography}{9}
\bibitem{JHD-JEM}
Davenport,J.H.,
OpenMath in a (Semantic) Web. 
Presentation at third Joining Educational Mathematics Workshop
\url{http://www.jem-thematic.net/files_private/Barcelona.pdf},
February 1, 2008.
\bibitem{JHD}
Davenport,J.H.,
OpenMath and MathML:
Differentiating between analysis and algebra.
\url{http://staff.bath.ac.uk/masjhd/differentiate.html},
October 4, 2008.
%\bibitem{MK}
%Kohlhase,M.,
%OpenMath3 without conditions: A Proposal for a
%MathML3/OM3 Calculus Content Dictionary.
%\url{http://svn.openmath.org/OpenMath3/doc/blue/noconds/note.tex},
%September 6, 2008.
\end{thebibliography}
\fi
\end{document}
$$
2\pi\phi=\left\{\int_\delta^{2\pi-\delta}+\int_{-\delta}^0+\int_0^\delta\right\}
\frac{a^2-r^2}{a^2-2ar\cos \vartheta+r^2}f(\theta+\vartheta)d\vartheta
$$
\cite[(8) p.~435]{JeffreysJeffreys1956}
\par
\begin{quotation}
If $Q=[0,1]\times[0,\pi/2]$, evaluate
${\int\int}_midsub Q (x\sin y -ye^x)dxdy$. \cite[p. 360, Example
1]{Apostol1967}
\pause
\begin{itemize}
\end{itemize}
\begin{lstlisting}
<OMA>
</OMA>
\end{lstlisting}
\begin{enumerate}
\end{enumerate}
