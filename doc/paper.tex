\documentclass{llncs}
\usepackage[show]{ed}
\usepackage{calbf}
\usepackage{stex-logo}
\usepackage{amstext,amsmath,amssymb}
\usepackage{xspace}
\usepackage{listings}\lstset{basicstyle=\sf,columns=fullflexible}

% \usepackage{mdframed}
% \newenvironment{boxedquote}{\begin{mdframed}[leftmargin=1cm,rightmargin=1cm]}{\end{mdframed}}

\usepackage{wrapfig,paralist}
\usepackage[hyperref,style=alphabetic]{biblatex}
% \addbibresource{kwarcpubs.bib}
% \addbibresource{extpubs.bib}
% \addbibresource{kwarccrossrefs.bib}
% \addbibresource{extcrossrefs.bib}
\addbibresource{kwarc.bib}
\addbibresource{rest.bib}
\def\latexml{{\LaTeX}ML\xspace}
\pagestyle{plain}
\usepackage{tikz}\usetikzlibrary{docicon}

\usepackage{hyperref}

 \title{System Description: A Semantics-Aware {\LaTeX}-to-Office Converter}
 \author{Lukas Kohlhase and Michael Kohlhase}
\institute{Mathematics/Computer Science\\
  Jacobs University Bremen}
\begin{document} 
\maketitle
\begin{abstract}
  We present a {\LaTeX}-to-Office conversion plugin for \latexml that can bridge the
  divide between publication practices in the theoretical disciplines (\LaTeX) and the
  applied ones (predominantly Office). The advantage of this plugin over other converters
  is that \latexml conserves enough of the document- and formula structure, that the
  transformed structures can be edited and processed further.
\end{abstract}

\section{Problem \& State of the Art}\label{sec:intro}

Technical documents from the STEM fields (\underline{S}cience, \underline{T}echnology,
\underline{E}ngineering, and \underline{M}athematics) augment the text with structured
objects -- images, mathematical/chemical formulae, diagrams, and tables -- that carry
essential parts of the information. There are two camps with different techniques for
authoring documents. The more theoretical disciplines (Mathematics, Physics, and Computer
Science) prefer {\LaTeX}, while the more applied ones (e.g. Life Sciences, Chemistry,
Engineering) use Office Suites almost exclusively. Transforming between these two document
formatting approaches is non-trivial: The {\TeX/\LaTeX} paradigm relies on in-document
macros to ``program'' documents, empowering authors to automate document aspects and
leading to community-supplied domain-specific extensions via {\LaTeX} packages. Office
suites rely on document styles that adapt visual parameters of the underlying document
markup either document-wide or for individual elements.

This incompatibility of document preparation approaches causes friction in cross-paradigm
collaboration as each camp deems their approach vastly superior and the other's
insufferable. In this paper, we will discuss the transformation from {\TeX/\LaTeX} to
Office documents. The converse direction would also be useful, but uses different methods.

\begin{figure}[ht]\centering\vspace*{-1em}
  \begin{tabular}{|c|c|}\hline%|
    copy from PDF & paste (libreoffice)\\\hline
    \includegraphics[width=6cm]{mathsnippet} & 
    \includegraphics[width=5cm]{mathsnippet-libreoffice}\\\hline
  \end{tabular}
\caption{Copy \& Paste in Word Processors}\label{fig:cnp}\vspace*{-1em}
\end{figure}

There are several methods to transform papers from {\LaTeX} to an office word
processor. The first method is to just generate a PDF file and then open this file in
Word/LibreOffice or copy/paste a fragment. This achieves the goal of looking like the
desired PDF document, just in Office. There are two problems with this route:
\begin{inparaenum}[\em i\rm)]
\item mathematical formulae are not preserved (see Figure~\ref{fig:cnp})
\item even if the result looks OK the results have lost their links (e.g. for
  citations/references or label/ref), or become difficult to edit, because they do not
  conform to the styling system of the word processor.
\end{inparaenum}
The fundamental problem is that this process converts only the appearance of the document
and loses all meaning that was encoded markup macros that were expanded in the PDF
generation. This is especially blatant when looking at the math in a document, which is
either treated as text or images and cannot be edited/processed further. The same holds
true for references, they are essentially treated as parts of text with a linked number in
front of them, complicating adding new references substantially.

The other way of transforming {\LaTeX} to Word, by transforming the {\LaTeX} source file
directly, avoids these problems. \texttt{latex2rtf}~\cite{latex2rtf:on} is a widely used
system that uses a custom parser to convert a non-trivial fragment of {\LaTeX} to the RTF
format understood by most office systems. The system works well, but coverage is limited
by the {\LaTeX} parser and the aging RTF format.  TeX4ht~\cite{tex4ht:online}, which uses
the {\TeX} parser itself and seeds the output with custom directives that are parsed to
create HTML has a post-processor that generates ODF. Its coverage of {\LaTeX} is unlimited,
but the intermediate format HTML somewhat limits the range of document fragments that can
be generated. 

Here we present a similar approach, only that we extend the backend of the \latexml
system~\cite{Miller:latexml:online} to generate WML -- the file format of MS Word -- and
ODT -- that of Libre- and OpenOffice. Like \texttt{latex2rdf}, the \latexml system
directly parses {\LaTeX} source files. The main difference to TeX4ht is that \latexml
generates an XML representation that is structurally near to the {\LaTeX} sources and thus
preserves the author-supplied semantics for further processing. Coverage for {\TeX}
primitives is complete, semantics-preserving \latexml bindings are available for most
commonly used {\LaTeX} packages.

\section{The Office Formats}\label{sec:target}

WML and ODT follow the same architectural paradigm: they are both zip-packaged directories
of XML files that contain document content, metadata, and styling. We will use WML in our
presentation here and point out differences in ODT as we go along.

The main content of a WML document -- text, document structure, placement of images,
tables etc. -- is represented by special content markup elements in an XML file
\texttt{document.xml}. All elements contain styling information, usually by referencing a
style element in the file \texttt{style.xml}, which can be modified by adding local
settings in children of the \texttt{properties} child. The other important kind of file
are the \texttt{.rels} files, which are again XML. These files contain
\texttt{relationship} elements, which detail the relations between elements in
\texttt{document.xml} and external resources (e.g. for hyperlinks) or resources in the WML
package (e.g. the image data files). The WML package additionally contains miscellaneous
XML files; e.g. \texttt{settings.xml}, which is used to make the state of the word
processor applications persistent and \texttt{fonttable.xml}, which contains extra
information about fonts.

\begin{wrapfigure}r{4.8cm}\vspace*{-3em}
\lstinputlisting[basicstyle=\sf\scriptsize,language=XML]{wmlmath.xml}
\vspace*{-3em}
\end{wrapfigure}
Of special interest here is the representation of mathematical formulae. WML uses a
proprietary XML format for presentation markup together with a variant of {\TeX} markup
that is used for user input. The expression of the left is the --slightly abridged --
representation of $1.5\times 10^7$.  The ODT format treats formulae as external objects;
every single one has a subdirectory in the package which contains a presentation MathML
file (for external communication), a user input file in the venerable StarOffice format,
and an image of the formula (for display in the word processor).

\section{Transformation}\label{sec:trans}

\begin{wrapfigure}r{5.3cm}\scriptsize\vspace*{-3em}
\begin{tikzpicture}[yscale=1.2,xscale=1]
\tikzstyle{doc}=[draw,thick,align=center,color=black,
                 shape=document,minimum width=10mm,minimum height=7mm]
\node[doc] (p) at (-1,3.5) {\texttt{paper.tex}};
\node[doc] (b) at (1,3.5) {\texttt{group.bib}};
\node[doc,dashed] (px) at (-1,2.2) {\texttt{paper.tex.xml}};
\node[doc,dashed] (bx) at (1,2.67) {\texttt{group.tex.xml}};
\draw[->,thick] (p) -- node[left,near end] (l) {\latexml} (px);
\draw[->,thick] (b) -- (bx);
\draw[->,thick] (bx) -- (l);
\node[doc,dashed] (d) at (-2,1) {\texttt{document.xml}};
\node[doc,dashed] (r) at (0.2,1) {\texttt{relations.xml}};
\node[doc] (s) at (1,.3) {\texttt{styles.xml}};
\draw[->,thick] (px) -- node[left]{XSLT}(d);
\draw[->,thick] (px) -- node[right]{XSLT}(r); 
\node[inner sep=0pt,outer sep=0pt] (z) at (-1,.3) {};
\draw[thick] (d) -- (z);
\draw[thick] (r) -- (z);
\draw[thick] (s) -- (z);
\node[doc] (dx) at (0-1,-.4) {\texttt{paper.docx}};
\draw[->,thick] (z) -- node[left] {zip} (dx);
\draw[dotted] (-3.1,-.1) rectangle (1.9,1.9);
\node at (1.6,1.8) {post};
\end{tikzpicture}
\caption{The Transformation Process}\label{fig:arch}\vspace*{-1.5em}
\end{wrapfigure}
To create the WML/ODT files we first transform the \texttt{.tex} file to an intermediate
XML-based \textsf{LTXML} format using \latexml. Then we use an XSLT stylesheet to generate
\texttt{document.xml}. For \textsf{LTXML} elements that do not have a direct counterpart
in WML we adapt existing WML elements. For instance, a {\LaTeX} \texttt{quote} environment
is represented by a WML \texttt{p} (``paragraph'') element with a special style
``\texttt{quote}'' we added to \texttt{styles.xml}. This which allows the user to later
semantically work with the document, e.g. by changing all quotes to red.  For WML
formulae, we use a stylesheet supplied by Microsoft to transform the MathML generated by
\latexml to the WML math format, for ODT formulae we make use of MathML and image
generation in \latexml. The file \texttt{relations.xml} is generated by XSLT from
\texttt{.tex.xml} and is placed into the directory structure the by the \latexml
post-processor together with other supporting files such as images and some static files
that are independent of the input document. The main file of interest here is
\texttt{styles.xml}, which contains the style information of the document. This had to be
adapted manually recreate the visual appearance of the PDF files generated by
{\LaTeX}. Finally the \latexml post-processor zips documents into the final WML/ODT file.

The user does not see all these transformation, generation, and packaging steps: given a
{\LaTeX} paper, all she has to do is type
\begin{quote}\tt
latexmlc paper.tex --destination=paper.docx
\end{quote}
A transformation to ODT can be specified by choosing the destination \texttt{paper.odt}.

\section{Conclusion}\label{sec:concl}
\begin{wrapfigure}r{5.5cm}\vspace*{-2.5em}
  \includegraphics[width=5.5cm]{convertedmathsnippet}\vspace*{-1.5em}
  \caption{Converted Formula in MS Word}\vspace*{-1.5em}
\end{wrapfigure}
We have presented a \latexml plugin that transforms {\LaTeX} papers into Office documents
in a one-line system call. The result of converting the the formula from
Figure~\ref{fig:cnp} to MS Word is on the right. With the recent web front-end of
\latexml, it will be simple to extend this to a web service. The \latexml Word Processing
plugin is public domain and is available from GitHub at~\cite{LaTeX2Office:github:on}. The
conversion makes crucial use of the fact that \latexml preserves more of the document and
formula semantics than other systems that process {\LaTeX} documents, this ensures that
the core process in the transformation -- the translation of \latexml XML to Office XML
(WML or ODF) has enough information to generate the respective target document
structures. The biggest limitations of the current transformation are that
\begin{inparaenum}[\em i\rm)]
\item we cannot currently generate the text-based input format (StarMath or the WML {\TeX}
  variant) and 
\item citations and references are only partially converted into the ``semantic'' formats.
\end{inparaenum}
This makes it difficult to edit formulae/references in the respective word processors
after transformation. For ODF formulae, we want to make use of the TeXMaths plugin for
Libreoffice, which uses {\LaTeX} instead of StarMath for user input of formulae -- but
hides it in the comment area of the images which makes handling more difficult.

In the future we want to develop an ``office package'' for \LaTeX and a corresponding
\latexml binding, which allows the direct markup of higher-level structures --
e.g. document metadata in {\LaTeX} documents, so that it can be transferred to the office
documents. Similarly, we want to extend the transformation to carry over even more
semantics from the \stex format into semantically extended office formats like CPoint or
CWord; this would finally give us a way to cleanly interface the currently {\LaTeX}-based
document methods in the KWARC group to applied STEM disciplines.

\printbibliography
\end{document}


%  LocalWords:  maketitle ednote hline libreoffice includegraphics mathsnippet cnp Hier
%  LocalWords:  einen einfuegen glaube ich docx impl odt Latexml erlaerung eigentlich px
%  LocalWords:  nicht zufrieden hiermit concl printbibliography cience echnology texttt
%  LocalWords:  ngineering athematics latex2rdf tikzpicture yscale xscale tikzstyle bx tt
%  LocalWords:  paper.ltxml group.ltxml ltmxl ltxml stex CPoint CWord ist Trennung und rm
%  LocalWords:  zwischen theoretisch praktisch ganz recht angewandte mathematiker bzw
%  LocalWords:  Physiker benutzen doch auch Libre rels fonttable.xml wrapfigure vspace
%  LocalWords:  lstinputlisting basicstyle scriptsize wmlmath.xml textsf latexmlc
%  LocalWords:  inparaenum TeXMaths convertedmathsnippet
