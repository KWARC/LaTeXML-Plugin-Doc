\documentclass{llncs}
\usepackage[show]{ed}
\usepackage{calbf}
\usepackage{stex-logo}
\usepackage{amstext,amsmath,amssymb}
\usepackage{xspace}

\usepackage{mdframed}
\newenvironment{boxedquote}{\begin{mdframed}[leftmargin=1cm,rightmargin=1cm]}{\end{mdframed}}

\usepackage{wrapfig,paralist}
\usepackage[hyperref,style=alphabetic,backend=bibtex]{biblatex}
% \addbibresource{kwarcpubs.bib}
% \addbibresource{extpubs.bib}
% \addbibresource{kwarccrossrefs.bib}
% \addbibresource{extcrossrefs.bib}
\addbibresource{kwarc.bib}
\addbibresource{rest.bib}
\def\latexml{{\LaTeX}ML\xspace}
\pagestyle{plain}
\usepackage{tikz}\usetikzlibrary{docicon}

\usepackage{hyperref}
\title{System Description: A Semantics-Aware {\LaTeX}-to-DOCX/ODF Converter}
\author{Lukas Kohlhase and Michael Kohlhase}
\institute{Mathematics/Computer Science\\
  Jacobs University Bremen}
\begin{document}
\maketitle
\begin{abstract}
  We present a {\LaTeX}-to-Office conversion plugin for \latexml ML that can bridge the
  divide between publication practices in the theoretical disciplines (\LaTeX) and the
  applied ones (predominantly Office). The advantage of this plugin over other converters
  is that \latexml conserves enough of the document- and formula structure, that the
  transformed structures can be edited and processed further.
\end{abstract}

\section{Problem \& State of the Art}\label{sec:intro}

Technical documents from the STEM (\underline{S}cience, \underline{T}echnology,
\underline{E}ngineering, and \underline{M}athematics) augment the text with structured
objects -- images, mathematical/chemical formulae, diagrams, and tables -- that carry
essential parts of the information. There are two camps with different techniques for
authoring documents. The more theoretical disciplines (Mathematics, Physics, and Computer
Science) almost exclusively {\LaTeX}, while the more applied ones (e.g. Life Sciences,
Chemistry, Engineering) use Office Suites almost exclusively. Transforming between these
two document formatting approaches is non-trivial: The {\TeX/\LaTeX} paradigm relies on
in-document macros to ``program'' documents, empowering authors to automate document
aspects and leading to community-supplied domain-specific extensions via \LaTeX
packages. Office suites rely on document-styles that adapt visual parameters of the
underlying document markup either document-wide or for individual elements.

This incompatibility of document preparation approaches causes friction in cross-paradigm
collaboration as each camp deems their approach vastly superior and the other's
insufferable. In this paper, we will discuss the transformation from {\TeX/\LaTeX} to
Office documents.

\begin{figure}[ht]
  \begin{tabular}{|c|c|}\hline%|
    copy from PDF & paste (libreoffice)\\\hline
    \includegraphics[width=6cm]{mathsnippet} & 
    \includegraphics[width=5cm]{mathsnippet-libreoffice}\\\hline
  \end{tabular}
\caption{Copy \& Paste in Word Processors}\label{fig:cnp}
\end{figure}

There are several methods to transform papers from {\LaTeX} to an office word
processor. The first method is to just generate a PDF file and then open this file in
Word/LibreOffice. This achieves the goal of looking like the desired PDF document, just in
Office. There are two problems with this route: 
\begin{enumerate}
\item mathematical formulae are not preserved (see Figure~\ref{fig:cnp})
\item even if the result looks OK the results have lost their links (e.g. for
  citations/references or label/ref), or become difficult to edit, because they do not
  conform to the styling system of the word processor.
\end{enumerate}
The fundamental problem is that it converts the appearance of the document and loses
meaning due to macro expansion. This is especially blatant when looking at the math in a
document. Either it is treated as text, with no meaningful way to distinguish between math
and formatted text that happens to contain some mathematical symbols, making automatic
treatment of this kind of math difficult, or it is represented by an image of the relevant
formulae, which makes editing extremely impractical if not impossible. The same holds true
for references, they are essentially treated as parts of text with a linked number in
front of them, complicating adding new references substantially.

The other way of transforming {\LaTeX} to Word, by transforming the {\LaTeX} source file
directly, avoids these problems. \texttt{latex2rtf}~\cite{latex2rtf:on} is a widely used
system that uses a custom parser to convert a non-trivial fragment of {\LaTeX} to the RTF
format understood by most office systems. The system works well, but coverage is limited
by the {\LaTeX} parser and the aging RTF format.  TeX4ht~\cite{tex4ht:online}, which uses
the {\TeX} parser itself and seeds the output with custom directives that are parsed to
create HTML has a post-processor that generates ODF. Its coverage of {\LaTeX} is unlimited,
but the intermediate format HTML somewhat limits the range of document fragments that can
be generated. 

Here we present a similar approach, only that we extend the backend of the \latexml system
to generate DOCX and ODF. Like \texttt{latex2rdf} \latexml directly parses {\LaTeX} source
files, but the coverage of \TeX is complete (including macro definitions) and
semantics-preserving bindings for the most important {\LaTeX} packages are provided. The
main difference to TeX4ht is that \latexml generates an XML representation that is
structurally near to the {\LaTeX} sources and preserves the author-supplied semantics for
further processing.

\section{Implementation}\label{sec:impl}

\begin{figure}[ht]\centering
\begin{tikzpicture}[yscale=1.5,xscale=1.2]
\tikzstyle{doc}=[draw,thick,align=center,color=black,
                 shape=document,minimum width=10mm,minimum height=8mm]
\node[doc] (p) at (-1,3.5) {\texttt{paper.tex}};
\node[doc] (b) at (1,3.5) {\texttt{group.bib}};
\node[doc,dashed] (px) at (-1,2.2) {\texttt{paper.ltxml}};
\node[doc,dashed] (bx) at (1,2.67) {\texttt{group.ltxml}};
\draw[->,thick] (p) -- node[left,near end] (l) {\latexml} (px);
\draw[->,thick] (b) -- (bx);
\draw[->,thick] (bx) -- (l);
\node[doc,dashed] (d) at (-2,1) {\texttt{document.xml}};
\node[doc,dashed] (r) at (0.2,1) {\texttt{relations.xml}};
\node[doc] (s) at (1,.3) {\texttt{styles.xml}};
\draw[->,thick] (px) -- node[left]{XSLT}(d);
\draw[->,thick] (px) -- node[right]{XSLT}(r); 
\node[inner sep=0pt,outer sep=0pt] (z) at (-1,.3) {};
\draw[thick] (d) -- (z);
\draw[thick] (r) -- (z);
\draw[thick] (s) -- (z);
\node[doc] (dx) at (0-1,-.4) {\texttt{paper.docx}};
\draw[->,thick] (z) -- node[left] {zip} (dx);
\draw[dotted] (-3.1,-.1) rectangle (1.9,1.9);
\node at (1.6,1.8) {post};
\end{tikzpicture}
\caption{The Transformation Process}\label{fig:arch}
\end{figure}

Both docx and odt files share a very similar structure and are almost interchangeable,
except for slight differences in syntax and different names. They both consist of zipped up XML files. The main content, such as text, placement of images, tables etc., is written in document.xml. The other important file is relations.xml, which contains information about where in the docx/odt file other supplementary files such as images are contained. Finally the archive contains various other objects such as style files, setting files and images. \\



To create the .odt/.docx files we first transform the .tex file to an intermediate
XML-based format using{\LaTeX}ml. \ednote{Papa Latexml erlaerung einfuegen}.

Then we use an XSLT stylesheet to generate document.xml from the .ltmxl file. For Word files, we use a Microsoft stylesheet to transform the MathML generated by {\LaTeX}ML to the docx math format. The other file we generate from the ltxml file using XSLT is relations.xml.The other supporting files such as images are placed into the correct file structure the post-processor. As the penultimate step some static files, that don't change depending on the input document, are also placed into the correct directories.The main file of interest here is styles.xml, which contains the style information of the document. We had to create this ourselves to recreate the feel of the PDF files generated by {\LaTeX}. Finally the document is zipped to create the docx/odt file. \\ 


\ednote{Screenshot einfuegen}
\ednote{Bin eigentlich nicht zufrieden hiermit TT}

\section{Conclusion}\label{sec:concl}
We have presented a \latexml plugin that transforms {\LaTeX} papers into Word/Office
documents in a one-line system call. With the recent web front-end of \latexml, it will be
simple to extend this to a web service. The \latexml Word Processing plugin is public
domain and is available from GitHub at~\cite{LaTeX2Office:github:on}. The conversion makes
crucial use of the fact that \latexml preserves more of the document and formula semantics
than other systems that process \LaTeX documents, this ensures that the core process in
the transformation -- the translation of \latexml XML to Office XML (DOCX or ODF) has
enough information to generate the respective target document structures.

In the future we want to develop an office package for \LaTeX, which allows the direct
markup of higher-level structures -- e.g. document metadata in {\LaTeX} documents, so that
it can be transferred to the office documents. Similarly, we want to extend the
transformation to carry over even more semantics from the \stex format into semantically
extended office formats like CPoint or
CWord~\cite{Kohlhase:SemanticInteractionDesignDiss:biblatex}; this would finally give us a
way to cleanly interface the currently {\LaTeX}-based document methods in the KWARC group
to applied STEM disciplines.

\printbibliography
\end{document}



%  LocalWords:  maketitle ednote hline libreoffice includegraphics mathsnippet cnp Hier
%  LocalWords:  einen einfuegen glaube ich docx impl odt Latexml erlaerung eigentlich px
%  LocalWords:  nicht zufrieden hiermit concl printbibliography cience echnology texttt
%  LocalWords:  ngineering athematics latex2rdf tikzpicture yscale xscale tikzstyle bx
%  LocalWords:  paper.ltxml group.ltxml ltmxl ltxml stex CPoint CWord
